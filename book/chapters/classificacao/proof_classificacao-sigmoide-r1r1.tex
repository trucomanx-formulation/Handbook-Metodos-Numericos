
%%%%%%%%%%%%%%%%%%%%%%%%%%%%%%%%%%%%%%%%%%%%%%%%%%%%%%%%%%%%%%%%%%%%%%%%%%%%%%%%%%%%%%%
%%%%%%%%%%%%%%%%%%%%%%%%%%%%%%%%%%%%%%%%%%%%%%%%%%%%%%%%%%%%%%%%%%%%%%%%%%%%%%%%%%%%%%%
\begin{myproofT}[Relativa ao Teorema \ref{theo:reglogr1r1:1}]\label{proof:theo:reglogr1r1}
Este procedimento é semelhante à Prova \ref{proof:theo:reglogr1r1poly} quando
na função $f_{\VECTOR{c}}(x)$,
\begin{equation}
y\equiv f_{\VECTOR{c}}(x)= \frac{1}{1+e^{-h_{\VECTOR{c}}(x) }},
\end{equation}
escolhemos uma função $h_{\VECTOR{c}}(x)= c_1 +c_2 x$;
é dizer um valor $M=1$ na Eq. (\ref{eq:proof:theo:reglogr1r1poly:1}).
\end{myproofT}

