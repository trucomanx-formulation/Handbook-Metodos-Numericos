
%%%%%%%%%%%%%%%%%%%%%%%%%%%%%%%%%%%%%%%%%%%%%%%%%%%%%%%%%%%%%%%%%%%%%%%%%%%%%%%%%%%%%%%
%%%%%%%%%%%%%%%%%%%%%%%%%%%%%%%%%%%%%%%%%%%%%%%%%%%%%%%%%%%%%%%%%%%%%%%%%%%%%%%%%%%%%%%
\begin{myproofT}[Relativa ao Teorema \ref{theo:reglogr1r1poly:1}]\label{proof:theo:reglogr1r1poly}
Dados,
os escalares $x \in \mathbb{R}$, $y \in \mathbb{R}$ e $c_m \in \mathbb{R}$,
uma função $f_{\VECTOR{c}}:\mathbb{R} \rightarrow \mathbb{R}$, 
uma função polinomial $h_{\VECTOR{c}}:\mathbb{R} \rightarrow \mathbb{R}$, de grau $M$, e 
definidas as seguintes equações,
\begin{equation}\label{eq:proof:theo:reglogr1r1poly:1}
y\equiv f_{\VECTOR{c}}(x)= \frac{1}{1+e^{-h_{\VECTOR{c}}(x) }},
\quad h_{\VECTOR{c}}(x)\equiv \sum_{m=0}^{M}c_{m+1} x^m\equiv \VECTOR{a}_M(x)\VECTOR{c},
\end{equation}
ou seu equivalente: $logit(y)=h_{\VECTOR{c}}(x)$,
onde $\VECTOR{c}=[c_1~ c_2~ ...~ c_m~ ...~ c_{M+1}]^{\transpose} \in \mathbb{R}^{M+1}$ é um vetor coluna e
$\VECTOR{a}_M(x)=\begin{bmatrix} 
1& x& x^2& \hdots & x^m& \hdots& x^M
\end{bmatrix}$ um vetor linha.
Se definimos um erro $e(\VECTOR{c})$ como
\begin{equation}\label{eq:proof:theo:reglogr1r1poly:2}
%e(\VECTOR{c}) = ||h(\VECTOR{x})-\VECTOR{y}||_{\MATRIX{W}}^2 \equiv \sum_{n=1}^{L} w_l||h(x_l)-y_l||^2,
e(\VECTOR{c}) =  \sum_{n=1}^{L} w_l||h_{\VECTOR{c}}(x_l)-logit(y_l)||^2,
\end{equation}
proveniente de avaliar $L$ amostras $x_l$ que pertencem a dois grupos, 
sendo que cada valor $x_l$ tem uma etiqueta de grupo $y_l\in \{A,1-A\}$, 
onde $0<A\ll 0.5$ é escolhido por nós.
Todos estes dados podem ser representados pelos vetores 
$\VECTOR{x}=[x_1~ x_2~$ $...~$ $x_l~$ $...~ x_L]^{\transpose}$ 
e $\VECTOR{z}=[logit(y_1)~$ $logit(y_2)~$ $...~$ $logit(y_l)~$ $...~ logit(y_L)]^{\transpose}$,
ponderados com os pesos $w_l \in \mathbb{R}_+$, 
representados pela matriz diagonal $\MATRIX{W}=\funcdiag([w_1~ w_2~ ...~ w_l~ ...~ w_L]^{\transpose})$.
De modo que podemos rescrever a Eq. (\ref{eq:proof:theo:reglogr1r1poly:2}) como,
\begin{equation}\label{eq:proof:theo:reglogr1r1poly:3}
e(\VECTOR{c}) \equiv ||\MATRIX{A}\VECTOR{c}-\VECTOR{z}||_{\MATRIX{W}}^2 
\end{equation}
onde a matriz $\MATRIX{A}$ é definida como,
\begin{equation}\label{eq:proof:reglogr1r1poly:4}
\MATRIX{A}=\begin{bmatrix}
\VECTOR{a}_M(x_1)\\
\VECTOR{a}_M(x_2)\\
\vdots\\
\VECTOR{a}_M(x_l)\\
\vdots\\
\VECTOR{a}_M(x_L)\\
\end{bmatrix}
\equiv 
\begin{bmatrix}
1      & x_1    & x_1^2  & \hdots  & x_1^M  \\
1      & x_2    & x_2^2  & \hdots  & x_2^M  \\
1      & x_3    & x_3^2  & \hdots  & x_3^M  \\
\vdots & \vdots & \vdots & \ddots  & \vdots \\
1      & x_l    & x_l^2  & \hdots  & x_l^M  \\
\vdots & \vdots & \vdots & \ddots  & \vdots \\
1      & x_L    & x_L^2  & \hdots  & x_L^M  \\ 
\end{bmatrix}.
\end{equation}


Usando o Teorema \ref{theo:minAxbCAxb}, sabemos que o vetor $\VECTOR{c}=\VECTOR{\hat{c}}$,
que minimiza a Eq. (\ref{eq:proof:theo:reglogr1r1poly:3}) pode ser achado usando 
\begin{equation}\label{eq:proof:theo:reglogr1r1poly:5}
\VECTOR{\hat{c}}=[\MATRIX{A}^{\transpose}\MATRIX{W}\MATRIX{A}]^{-1}\MATRIX{A}^{\transpose}\MATRIX{W}\VECTOR{z}.
\end{equation}
\end{myproofT}

