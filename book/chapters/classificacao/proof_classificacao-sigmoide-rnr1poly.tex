
%%%%%%%%%%%%%%%%%%%%%%%%%%%%%%%%%%%%%%%%%%%%%%%%%%%%%%%%%%%%%%%%%%%%%%%%%%%%%%%%%%%%%%%
%%%%%%%%%%%%%%%%%%%%%%%%%%%%%%%%%%%%%%%%%%%%%%%%%%%%%%%%%%%%%%%%%%%%%%%%%%%%%%%%%%%%%%%
\begin{myproofT}[Relativa ao Teorema \ref{theo:reglogrnr1poly:1}:]\label{proof:theo:reglogrnr1poly}
Dados
o vetor coluna $\VECTOR{x} \in \mathbb{R}^{N}$, os escalares $y \in \mathbb{R}$ e $c_m \in \mathbb{R}$,
uma função $f_{\VECTOR{c}}:\mathbb{R}^{N} \rightarrow \mathbb{R}$, 
uma função $h_{\VECTOR{c}}:\mathbb{R}^{N} \rightarrow \mathbb{R}$,  e 
definidas as seguintes equações,
\begin{equation}\label{eq:proof:theo:reglogrnr1poly:1}
y\equiv f_{\VECTOR{c}}(\VECTOR{x})= \frac{1}{1+e^{-h_{\VECTOR{c}}(\VECTOR{x}) }},
\quad h_{\VECTOR{c}}(\VECTOR{x}) =  \VECTOR{a}_M(\VECTOR{x})\VECTOR{c},
\end{equation}
ou seu equivalente: $logit(y)=h_{\VECTOR{c}}(\VECTOR{x})$,
onde $\VECTOR{c}=[c_1~ c_2~ c_3~  ...~ c_{L_M(N)}]^{\transpose} \in \mathbb{R}^{L_M(N)}$ é um vetor coluna
com $L_M(N)=\binom{N+M}{M}$ coeficientes\footnote{Para calcular $L_M(N)$, ver o Lema \ref{theo:reglogrnr1poly:aM}.} 
vinculados a cada elemento em
$\VECTOR{a}_M(\VECTOR{x})$, que  é um vetor com todos os monômios multivariados de grau total menor ou igual a $M$,
\begin{equation}\label{eq:proof:theo:reglogrnr1poly:2}
\VECTOR{a}_M(\VECTOR{x})=
\begin{bmatrix}
\VECTOR{b}_0(\VECTOR{x}) &
\VECTOR{b}_1(\VECTOR{x}) &
\hdots &
\VECTOR{b}_m(\VECTOR{x}) &
\hdots &
\VECTOR{b}_M(\VECTOR{x}) 
\end{bmatrix},
\end{equation}
onde $\VECTOR{b}_m(\VECTOR{x}) \in \mathbb{R}^{\phi(N,m)}$ é um vetor linha 
de $\phi(N,m)=\binom{N+m-1}{m}$ elementos\footnote{Para calcular $\phi(N,m)$, ver o Lema \ref{theo:reglogrnr1poly:bm}.},
com todos os monômios multivariados de grau total igual a $m$
formados a partir dos elementos do vetor $\VECTOR{x}$,
\begin{equation}\label{eq:proof:theo:reglogrnr1poly:3}
\VECTOR{b}_0(\VECTOR{x})=
\begin{bmatrix}
1 
\end{bmatrix},
\quad 
\VECTOR{b}_1(\VECTOR{x})=
\begin{bmatrix}
x_1 & x_2 & ... &  x_N
\end{bmatrix},
\quad 
\VECTOR{b}_2(\VECTOR{x})=
\begin{bmatrix}
x_1^2 & x_2^2 & ... &  x_N^2 & x_1 x_2 &  ... & x_{N-1} x_N
\end{bmatrix},
\end{equation}
\begin{equation}\label{eq:proof:theo:reglogrnr1poly:4}
\VECTOR{b}_m(\VECTOR{x})=
{\bigcup\limits_{\alpha_1+\alpha_2+...+\alpha_N=m}^{\rightarrow}}{x_1^{\alpha_1} x_2^{\alpha_2} ... x_N^{\alpha_N}};
\end{equation}
em geral, a ordem dos elementos (monômios) no vetor $\VECTOR{a}_M(\VECTOR{x})$ 
é arbitrário\footnote{Inclusive poderia se escolher só um número reduzido de monômios;
por exemplo, poderíamos escolher $\VECTOR{a}_3(\VECTOR{x})=[1\quad x_1\quad x_1 x_2^2\quad x_1^2]$,
consequentemente $\VECTOR{c}=[c_1\quad c_2\quad c_3\quad c_4]$,
de modo que $h_{\VECTOR{c}}(\VECTOR{x})=c_1 +c_2 x_1+c_3 x_1 x_2^2+c_4 x_1^2$.}, 
pois uma modificação na ordem só leva a uma modificação na ordem dos elementos no vetor $\VECTOR{c}$.
Assim, podemos definir o erro quadrático $e(\VECTOR{c})$ como
\begin{equation}\label{eq:proof:theo:reglogrnr1poly:5}
%e(\VECTOR{c}) = ||h(\VECTOR{x})-\VECTOR{y}||_{\MATRIX{W}}^2 \equiv \sum_{n=1}^{L} w_l||h(x_l)-y_l||^2,
e(\VECTOR{c}) =  \sum_{n=1}^{L} w_l||h_{\VECTOR{c}}(\VECTOR{x}_l)-logit(y_l)||^2,
\end{equation}
proveniente de avaliar o erro quadrático de $L$ amostras $\VECTOR{x}_l \in \mathbb{R}^{N}$ que pertencem a dois grupos, 
onde cada amostra $\VECTOR{x}_l$ tem uma etiqueta de grupo $y_l\in \{A,1-A\}$, 
 $0<A\ll 0.5$ é escolhido por nós, 
e os erros quadráticos são ponderados com os pesos $w_l \in \mathbb{R}_+$.
Assim, podemos reescrever a Eq. (\ref{eq:proof:theo:reglogrnr1poly:5}) como,
\begin{equation}\label{eq:proof:theo:reglogrnr1poly:6}
e(\VECTOR{c}) \equiv ||\MATRIX{A}\VECTOR{c}-\VECTOR{z}||_{\MATRIX{W}}^2 
\end{equation}
onde
\begin{equation}\label{eq:proof:reglogrnr1poly:7}
\MATRIX{A}=\begin{bmatrix}
\VECTOR{a}_M(\VECTOR{x}_1)\\
\VECTOR{a}_M(\VECTOR{x}_2)\\
\vdots\\
\VECTOR{a}_M(\VECTOR{x}_l)\\
\vdots\\
\VECTOR{a}_M(\VECTOR{x}_L)\\
\end{bmatrix},
\quad
\VECTOR{z}=
\begin{bmatrix}
logit(y_1)  \\
logit(y_2)  \\
\vdots  \\
logit(y_l)  \\
\vdots \\
logit(y_L) \\
\end{bmatrix},
\quad
\MATRIX{W}=
\funcdiag \left(
\begin{bmatrix}
w_1  \\
w_2  \\
\vdots  \\
w_l  \\
\vdots \\
w_L \\
\end{bmatrix} \right).
\end{equation}


Usando o Teorema \ref{theo:minAxbCAxb}, sabemos que o vetor $\VECTOR{c}=\VECTOR{\hat{c}}$,
que minimiza as Eqs. (\ref{eq:proof:theo:reglogrnr1poly:5}) e (\ref{eq:proof:theo:reglogrnr1poly:6}), 
pode ser obtido usando 
\begin{equation}\label{eq:proof:theo:reglogrnr1poly:8}
\VECTOR{\hat{c}}=[\MATRIX{A}^{\transpose}\MATRIX{W}\MATRIX{A}]^{-1}\MATRIX{A}^{\transpose}\MATRIX{W}\VECTOR{z}.
\end{equation}
\end{myproofT}

