\section{Provas dos teoremas}

%%%%%%%%%%%%%%%%%%%%%%%%%%%%%%%%%%%%%%%%%%%%%%%%%%%%%%%%%%%%%%%%%%%%%%%%%%%%%%%%%%%%%%%
%%%%%%%%%%%%%%%%%%%%%%%%%%%%%%%%%%%%%%%%%%%%%%%%%%%%%%%%%%%%%%%%%%%%%%%%%%%%%%%%%%%%%%%
\begin{myproofT}[Relativa ao Teorema \ref{theo:derAx}:]\label{proof:theo:derAx}
Dados
uma matriz $\MATRIX{A}=\left[a_{:1}~ a_{:2}~ \hdots~ a_{:n}~ \hdots~ a_{:N}\right]$ e 
um vetor coluna $\VECTOR{x}=\left[x_{1}~ x_{2}~ \hdots~ x_{n}~ \hdots~ x_{N}\right]^{\transpose}$, 
podemos expressar que:
\begin{equation}
\frac{\partial \MATRIX{A}\VECTOR{x}}{\partial x_n}=
\frac{\partial \MATRIX{A}}{\partial x_n}\VECTOR{x}+\MATRIX{A}\frac{\partial \VECTOR{x}}{\partial x_n}=
\MATRIX{A}\frac{\partial \VECTOR{x}}{\partial x_n}.
\end{equation}
Sabendo que $\frac{\partial \VECTOR{x}}{\partial x_n}$ é igual a um vetor 
com um $1$ na posição $n$ e $0$ em qualquer outra posição, obtemos que
\begin{equation}
\frac{\partial \MATRIX{A}\VECTOR{x}}{\partial x_n}=
\left[a_{:1}~ a_{:2}~ \hdots~ a_{:n}~ \hdots~ a_{:N}\right]\left[
\begin{matrix}
 0\\
 0\\
 \vdots\\
 1\\
 \vdots\\
 0
\end{matrix}
\right]=a_{:n}.
\end{equation}
\end{myproofT}

%%%%%%%%%%%%%%%%%%%%%%%%%%%%%%%%%%%%%%%%%%%%%%%%%%%%%%%%%%%%%%%%%%%%%%%%%%%%%%%%%%%%%%%
%%%%%%%%%%%%%%%%%%%%%%%%%%%%%%%%%%%%%%%%%%%%%%%%%%%%%%%%%%%%%%%%%%%%%%%%%%%%%%%%%%%%%%%
\begin{myproofT}[Relativa ao Teorema \ref{theo:derxAtAx}:]\label{proof:theo:derxAtAx}
Dados
uma matriz $\MATRIX{A}=\left[a_{:1}~ a_{:2}~ \hdots~ a_{:n}~ \hdots~ a_{:N}\right]$ e 
um vetor coluna $\VECTOR{x}=\left[x_{1}~ x_{2}~ \hdots~ x_{n}~ \hdots~ x_{N}\right]^{\transpose}$, 
podemos expressar que:
\begin{equation}\label{eq:proof:derxAtAx1}
\frac{\partial ||\MATRIX{A}\VECTOR{x}||^{2}}{\partial x_n}=
\frac{\partial \left(\MATRIX{A}\VECTOR{x}\right)^{\transpose}\left(\MATRIX{A}\VECTOR{x}\right)}{\partial x_n}=
\left(\frac{\partial \MATRIX{A}\VECTOR{x}}{\partial x_n}\right)^{\transpose}\left(\MATRIX{A}\VECTOR{x}\right)+
\left(\MATRIX{A}\VECTOR{x}\right)^{\transpose} \frac{\partial \MATRIX{A}\VECTOR{x}}{\partial x_n}.
\end{equation}
Pelo visto no Teorema \ref{theo:derAx}, podemos substituir valores na Eq. (\ref{eq:proof:derxAtAx1})
e obter:
\begin{equation}\label{eq:proof:derxAtAx2}
\frac{\partial ||\MATRIX{A}\VECTOR{x}||^{2}}{\partial x_n}=
\left(a_{:n}\right)^{\transpose} \MATRIX{A}\VECTOR{x} +
\left(\MATRIX{A}\VECTOR{x}\right)^{\transpose} a_{:n}.
\end{equation}
Como cada um dos somandos da equação anterior é um escalar, podemos aplicar o operador
transposta ($\transpose$) sobre qualquer somando sem alterar o resultado; 
de modo que temos duas possíveis
formas de expressar a solução:
\begin{equation}
\begin{matrix}
\frac{\partial ||\MATRIX{A}\VECTOR{x}||^2 }{\partial x_n}&=&
2\left(\MATRIX{A}\VECTOR{x}\right)^{\transpose}a_{:n}\\
~&=& 2\left(a_{:n}\right)^{\transpose}\MATRIX{A}\VECTOR{x}.
\end{matrix}
\end{equation}
\end{myproofT}

%%%%%%%%%%%%%%%%%%%%%%%%%%%%%%%%%%%%%%%%%%%%%%%%%%%%%%%%%%%%%%%%%%%%%%%%%%%%%%%%%%%%%%%
%%%%%%%%%%%%%%%%%%%%%%%%%%%%%%%%%%%%%%%%%%%%%%%%%%%%%%%%%%%%%%%%%%%%%%%%%%%%%%%%%%%%%%%
\begin{myproofT}[Relativa ao Teorema \ref{theo:derAxbAxb}:]\label{proof:theo:derAxbAxb}
Dados
uma matriz $\MATRIX{A}=\left[a_{:1}~ a_{:2}~ \hdots~ a_{:n}~ \hdots~ a_{:N}\right]$, 
uma matriz diagonal $\MATRIX{C}\in \mathbb{R}^{M\times M}$, 
um vetor coluna $\VECTOR{x}=\left[x_{1}~ x_{2}~ \hdots~ x_{n}~ \hdots~ x_{N}\right]^{\transpose}$, e 
um vetor coluna $\VECTOR{b}\in \mathbb{R}^{M}$, 
podemos expressar que:
\begin{equation}\label{eq:proof:derAxbAxb1}
\frac{\partial ||\MATRIX{A}\VECTOR{x}-\VECTOR{b}||_{\MATRIX{C}}^2}{\partial x_n} =
\frac{\partial \left(\MATRIX{A}\VECTOR{x}-\VECTOR{b}\right)^{\transpose}\MATRIX{C}\left(\MATRIX{A}\VECTOR{x}-\VECTOR{b}\right)}{\partial x_n}=
 \left(\frac{\partial\MATRIX{A}\VECTOR{x}}{\partial x_n}\right)^{\transpose}\MATRIX{C}\left(\MATRIX{A}\VECTOR{x}-\VECTOR{b}\right)+
 \left(\MATRIX{A}\VECTOR{x}-\VECTOR{b}\right)^{\transpose}\MATRIX{C}\left(\frac{\partial\MATRIX{A}\VECTOR{x}}{\partial x_n}\right).
\end{equation}
Pelo visto no Teorema \ref{theo:derAx}, podemos substituir valores na Eq. (\ref{eq:proof:derAxbAxb1})
e obter:
\begin{equation}\label{eq:proof:derAxbAxb2}
\frac{\partial ||\MATRIX{A}\VECTOR{x}-\VECTOR{b}||_{\MATRIX{C}}^{2}}{\partial x_n}=
\left(a_{:n}\right)^{\transpose}\MATRIX{C}\left( \MATRIX{A}\VECTOR{x}-\VECTOR{b}\right) +
\left(\MATRIX{A}\VECTOR{x}-\VECTOR{b}\right)^{\transpose}\MATRIX{C} a_{:n}.
\end{equation}
Como cada um dos somandos da equação anterior é um escalar, podemos aplicar o operador
transposta ($\transpose$) sobre qualquer somando sem alterar o resultado; de modo que temos duas possíveis
forma de expressar a solução:
\begin{equation}
\begin{matrix}
\frac{\partial ||\MATRIX{A}\VECTOR{x}-\VECTOR{b}||_{\MATRIX{C}}^2 }{\partial x_n}&=&
2\left(\MATRIX{A}\VECTOR{x}-\VECTOR{b}\right)^{\transpose}\MATRIX{C}a_{:n}\\
~&=& 2\left(a_{:n}\right)^{\transpose}\MATRIX{C}\left(\MATRIX{A}\VECTOR{x}-\VECTOR{b}\right).
\end{matrix}
\end{equation}
\end{myproofT}

%%%%%%%%%%%%%%%%%%%%%%%%%%%%%%%%%%%%%%%%%%%%%%%%%%%%%%%%%%%%%%%%%%%%%%%%%%%%%%%%%%%%%%%
%%%%%%%%%%%%%%%%%%%%%%%%%%%%%%%%%%%%%%%%%%%%%%%%%%%%%%%%%%%%%%%%%%%%%%%%%%%%%%%%%%%%%%%
\begin{myproofT}[Relativa ao Teorema \ref{theo:derfxbCfxb0}:]\label{proof:theo:derfxbCfxb0}
Dados
uma função $\VECTOR{f}:\mathbb{R}^{N} \rightarrow \mathbb{R}^{M}$, 
uma matriz diagonal $\MATRIX{C}\in \mathbb{R}^{M\times M}$, 
um vetor coluna $\VECTOR{x}=\left[x_{1}~ x_{2}~ \hdots~ x_{n}~ \hdots~ x_{N}\right]^{\transpose}$, e
um vetor coluna $\VECTOR{b}\in \mathbb{R}^{M}$;
podemos expressar que:
\begin{equation}\label{eq:proof:derfxbCfxb01}
\begin{matrix}
\frac{\partial ||\VECTOR{f}(\VECTOR{x})-\VECTOR{b}||_{\MATRIX{C}}^2}{\partial x_{n}} & = &
\frac{\partial \left( \VECTOR{f}(\VECTOR{x})-\VECTOR{b}\right)^{\transpose} \MATRIX{C} \left( \VECTOR{f}(\VECTOR{x})-\VECTOR{b}\right) }{\partial x_{n}} \\
~ &= &
 \left(  \frac{\partial\VECTOR{f}(\VECTOR{x})  }{\partial x_{n}} \right)^{\transpose} \MATRIX{C} \left( \VECTOR{f}(\VECTOR{x})-\VECTOR{b}\right) +
 \left( \VECTOR{f}(\VECTOR{x})-\VECTOR{b}\right)^{\transpose} \MATRIX{C}  \left(  \frac{\partial\VECTOR{f}(\VECTOR{x})  }{\partial x_{n}} \right)\\
~ &= &
 2 \left(  \frac{\partial\VECTOR{f}(\VECTOR{x})  }{\partial x_{n}} \right)^{\transpose} \MATRIX{C} \left( \VECTOR{f}(\VECTOR{x})-\VECTOR{b}\right).
\end{matrix}
\end{equation}
Assim, usando a Definição \ref{def:deltaver} junto com a Eq. (\ref{eq:proof:derfxbCfxb01}),
obtemos:
\begin{equation}\label{eq:proof:derfxbCfxb02}
\frac{\partial ||\VECTOR{f}(\VECTOR{x})-\VECTOR{b}||_{\MATRIX{C}}^2}{\partial \VECTOR{x}}  = 
2 \MATRIX{J}(\VECTOR{x})^{\transpose} \MATRIX{C} \left( \VECTOR{f}(\VECTOR{x})-\VECTOR{b}\right).
\end{equation}
\end{myproofT}


\begin{myproofT}[Relativa ao Teorema \ref{theo:derfxbCfxb}:]\label{proof:theo:derfxbCfxb}
Dados
uma função $\VECTOR{f}:\mathbb{R}^{N} \rightarrow \mathbb{R}^{M}$, 
uma matriz diagonal $\MATRIX{C}\in \mathbb{R}^{M\times M}$, 
um vetor coluna $\VECTOR{x}\in \mathbb{R}^{N}$, 
um vetor coluna $\VECTOR{b}\in \mathbb{R}^{M}$, 
e considerando a aproximação
$\VECTOR{f}(\VECTOR{x})\approx \VECTOR{f}(\VECTOR{p})+\MATRIX{J}(\VECTOR{p})\left(\VECTOR{x}-\VECTOR{p}\right)$,
obtida a partir da \hyperref[def:taylor]{\textbf{série de Taylor}} para funções multivariáveis;
podemos expressar que:
\begin{equation}\label{eq:proof:derfxbCfxb1}
\frac{\partial ||\VECTOR{f}(\VECTOR{x})-\VECTOR{b}||_{\MATRIX{C}}^2}{\partial \VECTOR{x}} \approx
\frac{\partial ||\MATRIX{J}(\VECTOR{p})\VECTOR{x}-\left[\MATRIX{J}(\VECTOR{p})\VECTOR{p}+\VECTOR{b}-\VECTOR{f}(\VECTOR{p})\right]||_{\MATRIX{C}}^2}{\partial \VECTOR{x}}.
\end{equation}
Pelo visto no Corolário \ref{coro:derAxbAxb2}, podemos substituir os valores,
$\MATRIX{J}(\VECTOR{p})$ e 
$\left[\MATRIX{J}(\VECTOR{p})\VECTOR{p}+\VECTOR{b}-\VECTOR{f}(\VECTOR{p})\right]$,
da Eq. (\ref{eq:proof:derfxbCfxb1}), nas variáveis $\MATRIX{A}$ e $\VECTOR{b}$ 
do Corolário \ref{coro:derAxbAxb2}, respectivamente. Assim obtemos:
\begin{equation}\label{eq:proof:derfxbCfxb2}
\frac{\partial ||\MATRIX{J}(\VECTOR{p})\VECTOR{x}-\left[\MATRIX{J}(\VECTOR{p})\VECTOR{p}+\VECTOR{b}-\VECTOR{f}(\VECTOR{p})\right]||_{\MATRIX{C}}^2}{\partial \VECTOR{x}}  = 
2 \MATRIX{J}(\VECTOR{p})^{\transpose}\MATRIX{C}\left( \MATRIX{J}(\VECTOR{p})\VECTOR{x}-\MATRIX{J}(\VECTOR{p})\VECTOR{p}-\VECTOR{b}+\VECTOR{f}(\VECTOR{p})\right).
\end{equation}
\end{myproofT}


%%%%%%%%%%%%%%%%%%%%%%%%%%%%%%%%%%%%%%%%%%%%%%%%%%%%%%%%%%%%%%%%%%%%%%%%%%%%%%%%%%%%%%%
%%%%%%%%%%%%%%%%%%%%%%%%%%%%%%%%%%%%%%%%%%%%%%%%%%%%%%%%%%%%%%%%%%%%%%%%%%%%%%%%%%%%%%%
\begin{myproofT}[Relativa ao Teorema \ref{theo:der2fxbCfxb0}:]\label{proof:theo:der2fxbCfxb0}
Dados
uma função $\VECTOR{f}:\mathbb{R}^{N} \rightarrow \mathbb{R}^{M}$, 
uma matriz diagonal $\MATRIX{C}\in \mathbb{R}^{M\times M}$, 
um vetor coluna $\VECTOR{x}=\left[x_{1}~ x_{2}~ \hdots~ x_{n}~ \hdots~ x_{N}\right]^{\transpose}$, e
um vetor coluna $\VECTOR{b}\in \mathbb{R}^{M}$;
podemos expressar que:
\begin{equation}\label{eq:proof:der2fxbCfxb01}
\begin{matrix}
\frac{\partial }{\partial x_{n}}\left( \frac{\partial ||\VECTOR{f}(\VECTOR{x})-\VECTOR{b}||_{\MATRIX{C}}^2}{\partial \VECTOR{x}} \right) & = &
\frac{\partial 2 \MATRIX{J}(\VECTOR{x})^{\transpose} \MATRIX{C} \left( \VECTOR{f}(\VECTOR{x})-\VECTOR{b}\right)}{\partial x_{n}} \\
~ & = & 2 \frac{\partial \MATRIX{J}(\VECTOR{x})^{\transpose} }{\partial x_{n}} \MATRIX{C} \left( \VECTOR{f}(\VECTOR{x})-\VECTOR{b}\right)+
2  \MATRIX{J}(\VECTOR{x})^{\transpose}  \MATRIX{C} \frac{\partial \left( \VECTOR{f}(\VECTOR{x})-\VECTOR{b}\right) }{\partial x_{n}}\\
~ & = & 2 \frac{\partial \MATRIX{J}(\VECTOR{x})^{\transpose} }{\partial x_{n}} \MATRIX{C} \left( \VECTOR{f}(\VECTOR{x})-\VECTOR{b}\right)+
2  \MATRIX{J}(\VECTOR{x})^{\transpose}  \MATRIX{C} \frac{\partial \VECTOR{f}(\VECTOR{x}) }{\partial x_{n}}.\\
\end{matrix}
\end{equation}
Assim, usando a Definição \ref{def:deltahor} junto com a Eq. (\ref{eq:proof:der2fxbCfxb01}),
obtemos:
\begin{equation}\label{eq:proof:der2fxbCfxb02}
\frac{\partial }{\partial \VECTOR{x}^{\transpose}}\left( \frac{\partial ||\VECTOR{f}(\VECTOR{x})-\VECTOR{b}||_{\MATRIX{C}}^2}{\partial \VECTOR{x}} \right) =
2 {\bigcup\limits_{n=1}^{ \rightarrow }}^{N}\left\{ \frac{\partial \MATRIX{J}(\VECTOR{x})^{\transpose} }{\partial x_{n}} \MATRIX{C} \left( \VECTOR{f}(\VECTOR{x})-\VECTOR{b}\right) \right\} +
2  \MATRIX{J}(\VECTOR{x})^{\transpose}  \MATRIX{C} \frac{\partial \VECTOR{f}(\VECTOR{x}) }{\partial \VECTOR{x}^{\transpose}}.
\end{equation}
\end{myproofT}

\begin{myproofT}[Relativa ao Teorema \ref{theo:der2fxbCfxb0aprox}:]\label{proof:theo:der2fxbCfxb0aprox}
Dados
uma função $\VECTOR{f}:\mathbb{R}^{N} \rightarrow \mathbb{R}^{M}$, 
uma matriz diagonal $\MATRIX{C}\in \mathbb{R}^{M\times M}$, 
os vetores coluna $\VECTOR{x}\in \mathbb{R}^{N}$ e 
$\VECTOR{b}\in \mathbb{R}^{M}$, e
considerando o Toerema \ref{theo:derfxbCfxb} que usa a aproximação
$\VECTOR{f}(\VECTOR{x})\approx \VECTOR{f}(\VECTOR{p}) + \MATRIX{J}(\VECTOR{p})$ $\left(\VECTOR{x}-\VECTOR{p}\right)$,
podemos expressar que:
\begin{equation}\label{eq:proof:der2fxbCfxb01aprox}
\begin{matrix}
\frac{\partial }{\partial \VECTOR{x}^{\transpose}}\left( \frac{\partial ||\VECTOR{f}(\VECTOR{x})-\VECTOR{b}||_{\MATRIX{C}}^2}{\partial \VECTOR{x}} \right) & \approx & 
\frac{\partial 2 \MATRIX{J}(\VECTOR{p})^{\transpose}\MATRIX{C}\left[\MATRIX{J}(\VECTOR{p})\left(\VECTOR{x} - \VECTOR{p}\right)-\left(\VECTOR{b}-\VECTOR{f}(\VECTOR{p})\right)\right]}{\partial \VECTOR{x}^{\transpose}} \\
~ & \approx & \frac{\partial 2 \MATRIX{J}(\VECTOR{p})^{\transpose}\MATRIX{C} \MATRIX{J}(\VECTOR{p})\VECTOR{x}  }{\partial \VECTOR{x}^{\transpose}}
\end{matrix},
\end{equation}
em que $\VECTOR{p}$ é um ponto fixo no domínio de $\VECTOR{f}(\VECTOR{x})$, 
ao redor do qual é feita  aproximação
da função $\VECTOR{f}(\VECTOR{x})$,
e $\MATRIX{J}(\VECTOR{p})$ é a \hyperref[def:jacobian]{\textbf{matriz Jacobiana}} 
de $\VECTOR{f}(\VECTOR{x})$ avaliado no ponto $\VECTOR{p}$.
Assim, usando o Teorema \ref{theo:derAx} na Eq. (\ref{eq:proof:der2fxbCfxb01aprox}),
obtemos,
\begin{equation}\label{eq:proof:der2fxbCfxb01aprox2}
\frac{\partial }{\partial \VECTOR{x}^{\transpose}}\left( \frac{\partial ||\VECTOR{f}(\VECTOR{x})-\VECTOR{b}||_{\MATRIX{C}}^2}{\partial \VECTOR{x}} \right) \approx 
2 \MATRIX{J}(\VECTOR{p})^{\transpose}\MATRIX{C} \MATRIX{J}(\VECTOR{p}).
\end{equation}
\end{myproofT}

%%%%%%%%%%%%%%%%%%%%%%%%%%%%%%%%%%%%%%%%%%%%%%%%%%%%%%%%%%%%%%%%%%%%%%%%%%%%%%%%%%%%%%%
%%%%%%%%%%%%%%%%%%%%%%%%%%%%%%%%%%%%%%%%%%%%%%%%%%%%%%%%%%%%%%%%%%%%%%%%%%%%%%%%%%%%%%%
\begin{myproofT}[Relativa ao Teorema \ref{theo:derfxbCfxbxqDxq}:]\label{proof:theo:derfxbCfxbxqDxq}
Dados
uma função $\VECTOR{f}:\mathbb{R}^{N} \rightarrow \mathbb{R}^{M}$, 
uma matriz diagonal $\MATRIX{C}\in \mathbb{R}^{M\times M}$, 
uma matriz diagonal $\MATRIX{D}\in \mathbb{R}^{N\times N}$, 
um vetor coluna $\VECTOR{x}\in \mathbb{R}^{N}$, 
um vetor coluna $\VECTOR{b}\in \mathbb{R}^{M}$, e 
um vetor coluna $\VECTOR{q}\in \mathbb{R}^{N}$, 
podemos expressar que:
\begin{equation}\label{eq:proof:derfxbCfxbxqDxq1}
\frac{\partial ||\VECTOR{f}(\VECTOR{x})-\VECTOR{b}||_{\MATRIX{C}}^2 +\alpha||\VECTOR{x}-\VECTOR{q}||_{\MATRIX{D}}^2}{\partial \VECTOR{x}} =
\frac{\partial ||\VECTOR{f}(\VECTOR{x})-\VECTOR{b}||_{\MATRIX{C}}^2}{\partial \VECTOR{x}}+
\alpha \frac{\partial ||\VECTOR{x}-\VECTOR{q}||_{\MATRIX{D}}^2}{\partial \VECTOR{x}}.
\end{equation}
Usando o Corolário \ref{coro:derAxbAxb2} na Eq. (\ref{eq:proof:derfxbCfxbxqDxq1}),
obtemos:
\begin{equation}\label{eq:proof:derfxbCfxbxqDxq2}
\frac{\partial ||\VECTOR{f}(\VECTOR{x})-\VECTOR{b}||_{\MATRIX{C}}^2 +\alpha||\VECTOR{x}-\VECTOR{q}||_{\MATRIX{D}}^2}{\partial \VECTOR{x}} =
\frac{\partial ||\VECTOR{f}(\VECTOR{x})-\VECTOR{b}||_{\MATRIX{C}}^2}{\partial \VECTOR{x}}+
\alpha 2 \MATRIX{D}\left(\VECTOR{x}-\VECTOR{q}\right).
\end{equation}
Usando o Teorema \ref{theo:derfxbCfxb} na Eq. (\ref{eq:proof:derfxbCfxbxqDxq2}),
obtemos:
\begin{equation}\label{eq:proof:derfxbCfxbxqDxq3}
\frac{\partial ||\VECTOR{f}(\VECTOR{x})-\VECTOR{b}||_{\MATRIX{C}}^2 +\alpha||\VECTOR{x}-\VECTOR{q}||_{\MATRIX{D}}^2}{\partial \VECTOR{x}} \approx
2 \MATRIX{J}(\VECTOR{p})^{\transpose}\MATRIX{C}\left[\MATRIX{J}(\VECTOR{p})\left(\VECTOR{x} - \VECTOR{p}\right)-\left(\VECTOR{b}-\VECTOR{f}(\VECTOR{p})\right)\right]+
2 \alpha \MATRIX{D}\left(\VECTOR{x}-\VECTOR{q}\right).
\end{equation}
\end{myproofT}
%
