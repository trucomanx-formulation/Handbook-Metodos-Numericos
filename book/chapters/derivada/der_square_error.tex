\chapterimage{chapter_derivada.pdf} % Chapter heading image

\chapter{Derivada de funções: $\mathbb{R}^{N}$ $\rightarrow$ $\mathbb{R}^{M}$}


%%%%%%%%%%%%%%%%%%%%%%%%%%%%%%%%%%%%%%%%%%%%%%%%%%%%%%%%%%%%%%%%%%%%%%%%%%%%%%%%%%%%%%%
%%%%%%%%%%%%%%%%%%%%%%%%%%%%%%%%%%%%%%%%%%%%%%%%%%%%%%%%%%%%%%%%%%%%%%%%%%%%%%%%%%%%%%%
\section{Derivada de $\MATRIX{A}\VECTOR{x}$}

\begin{theorem}\label{theo:derAx}
Se 
$\VECTOR{x}\in \mathbb{R}^N$ é um vetor coluna com elementos $x_n$ de modo que
$n\in \mathbb{N}$, $1 \leq n \leq N$, e 
$\MATRIX{A} \in \mathbb{R}^{M\times N}$ é uma matriz com elementos $a_{mn}$ de modo que
$m\in \mathbb{N}$, $1 \leq m \leq M$, então se cumpre que:
\begin{equation}
\frac{\partial \MATRIX{A}\VECTOR{x}}{\partial x_n}=a_{:n}
\end{equation}
A demostração pode ser vista na Prova \ref{proof:theo:derAx}.
\end{theorem}

\begin{corollaryT}[Derivada de $\MATRIX{A}\VECTOR{x}$ em relação ao vector $\VECTOR{x}^{\transpose}$]\label{coro:derAx1}
Aplicando a Definição \ref{def:deltahor} junto ao Teorema \ref{theo:derAx}, é
fácil deduzir que:
\begin{equation}
\frac{\partial \MATRIX{A}\VECTOR{x}}{\partial \VECTOR{x}^{\transpose}}=
\MATRIX{A}=
\left[
\begin{matrix}
 a_{:1} &  a_{:2} &  \cdots &  a_{:N}
\end{matrix}
\right]
\end{equation}
\end{corollaryT}

\begin{corollaryT}[Derivada de $\MATRIX{A}\VECTOR{x}$ em relação ao vector $\VECTOR{x}$]\label{coro:derAx2}
Aplicando a Definição \ref{def:deltaver} junto ao Teorema \ref{theo:derAx}, é
fácil deduzir que:
\begin{equation}
\frac{\partial \MATRIX{A}\VECTOR{x}}{\partial \VECTOR{x}}=\funcvec(\MATRIX{A})=
\left[
\begin{matrix}
 a_{:1} \\  a_{:2} \\  \vdots \\  a_{:N}
\end{matrix}
\right]
\end{equation}
\end{corollaryT}

\begin{corollaryT}[Derivada de $\VECTOR{a}^{\transpose}\VECTOR{x}$ em relação ao vector $\VECTOR{x}^{\transpose}$]\label{coro:derAx3}
Aplicando a Definição \ref{def:deltahor} junto ao Teorema \ref{theo:derAx} e sabendo que $\VECTOR{a}^{\transpose}$
é um vetor linha, é
fácil deduzir que:
\begin{equation}
\frac{\partial \VECTOR{a}^{\transpose}\VECTOR{x}}{\partial \VECTOR{x}^{\transpose}}=\VECTOR{a}^{\transpose}
\end{equation}
\end{corollaryT}

\begin{corollaryT}[Derivada de $\VECTOR{a}^{\transpose}\VECTOR{x}$ em relação ao vector $\VECTOR{x}$]\label{coro:derAx4}
Aplicando a Definição \ref{def:deltaver} junto ao Teorema \ref{theo:derAx} e sabendo que $\VECTOR{a}^{\transpose}$
é um vetor linha, é
fácil deduzir que:
\begin{equation}
\frac{\partial \VECTOR{a}^{\transpose}\VECTOR{x}}{\partial \VECTOR{x}}=\VECTOR{a}
\end{equation}
\end{corollaryT}

%%%%%%%%%%%%%%%%%%%%%%%%%%%%%%%%%%%%%%%%%%%%%%%%%%%%%%%%%%%%%%%%%%%%%%%%%%%%%%%%%%%%%%%
%%%%%%%%%%%%%%%%%%%%%%%%%%%%%%%%%%%%%%%%%%%%%%%%%%%%%%%%%%%%%%%%%%%%%%%%%%%%%%%%%%%%%%%
\section{Derivada de $||\MATRIX{A}\VECTOR{x}||^2$ 
}

\begin{theorem}\label{theo:derxAtAx}
Se 
$\VECTOR{x}\in \mathbb{R}^N$ é um vetor coluna com elementos $x_n$ de modo que
$n\in \mathbb{N}$, $1 \leq n \leq N$, e 
$\MATRIX{A} \in \mathbb{R}^{M\times N}$ é uma matriz com elementos $a_{mn}$ de modo que
$m\in \mathbb{N}$, $1 \leq m \leq M$, então se cumpre que:
\begin{equation}
\begin{matrix}
\frac{\partial ||\MATRIX{A}\VECTOR{x}||^2 }{\partial x_n}&=&
\frac{\partial \left(\MATRIX{A}\VECTOR{x}\right)^{\transpose}\left(\MATRIX{A}\VECTOR{x}\right)}{\partial x_n}&=&
2\left(\MATRIX{A}\VECTOR{x}\right)^{\transpose}a_{:n}\\
~&~&~&=& 2\left(a_{:n}\right)^{\transpose}\MATRIX{A}\VECTOR{x}
\end{matrix}
\end{equation}
A demostração pode ser vista na Prova \ref{proof:theo:derxAtAx}.
\end{theorem}

\begin{corollaryT}[Derivada de $||\MATRIX{A}\VECTOR{x}||^2$ em relação ao vector $\VECTOR{x}^{\transpose}$]\label{coro:derxAtAx1}
Aplicando a Definição \ref{def:deltahor} junto ao Teorema \ref{theo:derxAtAx}, é
fácil deduzir que:
\begin{equation}
\frac{\partial ||\MATRIX{A}\VECTOR{x}||^2 }{\partial \VECTOR{x}^{\transpose}}=
2\left(\MATRIX{A}^{\transpose}\MATRIX{A}\VECTOR{x}\right)^{\transpose}
\end{equation}
%\frac{\partial \left(\MATRIX{A}\VECTOR{x}\right)^{\transpose}\left(\MATRIX{A}\VECTOR{x}\right)}{\partial \VECTOR{x}^{\transpose}}=
\end{corollaryT}

\begin{corollaryT}[Derivada de $||\MATRIX{A}\VECTOR{x}||^2$ em relação ao vector $\VECTOR{x}$]\label{coro:derxAtAx2}
Aplicando a Definição \ref{def:deltaver} junto ao Teorema \ref{theo:derxAtAx}, é
fácil deduzir que:
\begin{equation}
\frac{\partial ||\MATRIX{A}\VECTOR{x}||^2 }{\partial \VECTOR{x}}=
2 \MATRIX{A}^{\transpose}\MATRIX{A}\VECTOR{x}
\end{equation}
%\frac{\partial \left(\MATRIX{A}\VECTOR{x}\right)^{\transpose}\left(\MATRIX{A}\VECTOR{x}\right)}{\partial \VECTOR{x}}=
\end{corollaryT}

%%%%%%%%%%%%%%%%%%%%%%%%%%%%%%%%%%%%%%%%%%%%%%%%%%%%%%%%%%%%%%%%%%%%%%%%%%%%%%%%%%%%%%%
%%%%%%%%%%%%%%%%%%%%%%%%%%%%%%%%%%%%%%%%%%%%%%%%%%%%%%%%%%%%%%%%%%%%%%%%%%%%%%%%%%%%%%%
\section{Derivada de $||\MATRIX{A}\VECTOR{x}-\VECTOR{b}||_{\MATRIX{C}}^2$ 
}

\begin{theorem}\label{theo:derAxbAxb}
Se 
$\VECTOR{x}\in \mathbb{R}^N$ é um vetor coluna com elementos $x_n$ de modo que
$n\in \mathbb{N}$, $1 \leq n \leq N$, 
$\VECTOR{b}\in \mathbb{R}^M$ é um vetor coluna com elementos $b_m$ de modo que
$m\in \mathbb{N}$, $1 \leq m \leq M$,  
$\MATRIX{A} \in \mathbb{R}^{M\times N}$ é uma matriz com elementos $a_{mn}$, e
$\MATRIX{C} \in \mathbb{R}^{M\times M}$ é uma matriz diagonal, 
então se cumpre que:
\begin{equation}
\begin{matrix}
\frac{\partial ||\MATRIX{A}\VECTOR{x}-\VECTOR{b}||_{\MATRIX{C}}^2}{\partial x_n}&=&
\frac{\partial \left(\MATRIX{A}\VECTOR{x}-\VECTOR{b}\right)^{\transpose}\MATRIX{C}\left(\MATRIX{A}\VECTOR{x}-\VECTOR{b}\right)}{\partial x_n}&=&
2\left(\MATRIX{A}\VECTOR{x}-\VECTOR{b}\right)^{\transpose}\MATRIX{C}a_{:n}\\
~&~&~&=& 2\left(a_{:n}\right)^{\transpose}\MATRIX{C}\left(\MATRIX{A}\VECTOR{x}-  \VECTOR{b}\right)
\end{matrix}
\end{equation}
A demostração pode ser vista na Prova \ref{proof:theo:derAxbAxb}.
\end{theorem}

\begin{corollaryT}[Derivada de $||\MATRIX{A}\VECTOR{x}-\VECTOR{b}||_{\MATRIX{C}}^2$ em relação ao vector $\VECTOR{x}^{\transpose}$]\label{coro:derAxbAxb1}
Aplicando a Definição \ref{def:deltahor} junto ao Teorema \ref{theo:derAxbAxb}, é
fácil deduzir que:
\begin{equation}
\frac{\partial ||\MATRIX{A}\VECTOR{x}-\VECTOR{b}||_{\MATRIX{C}}^2}{\partial \VECTOR{x}^{\transpose}}=
2\left(\MATRIX{A}\VECTOR{x}- \VECTOR{b} \right)^{\transpose}\MATRIX{C}\MATRIX{A}
\end{equation}
\end{corollaryT}

\begin{corollaryT}[Derivada de $||\MATRIX{A}\VECTOR{x}-\VECTOR{b}||_{\MATRIX{C}}^2$ em relação ao vector $\VECTOR{x}$]\label{coro:derAxbAxb2}
Aplicando a Definição \ref{def:deltaver} junto ao Teorema \ref{theo:derAxbAxb}, é
fácil deduzir que:
\begin{equation}
\frac{\partial ||\MATRIX{A}\VECTOR{x}-\VECTOR{b}||_{\MATRIX{C}}^2}{\partial \VECTOR{x} }=
2 \MATRIX{A}^{\transpose}\MATRIX{C}\left(\MATRIX{A}\VECTOR{x}-\VECTOR{b}\right)
\end{equation}
\end{corollaryT}

%%%%%%%%%%%%%%%%%%%%%%%%%%%%%%%%%%%%%%%%%%%%%%%%%%%%%%%%%%%%%%%%%%%%%%%%%%%%%%%%%%%%%%%
%%%%%%%%%%%%%%%%%%%%%%%%%%%%%%%%%%%%%%%%%%%%%%%%%%%%%%%%%%%%%%%%%%%%%%%%%%%%%%%%%%%%%%%
\section{Derivada de $||\VECTOR{f}(\VECTOR{x})-\VECTOR{b}||_{\MATRIX{C}}^2$  
}

\begin{theorem}[Valor exato:]\label{theo:derfxbCfxb0}
Se 
$\VECTOR{x}\in \mathbb{R}^N$ e, 
$\VECTOR{b}\in \mathbb{R}^M$ são vetores coluna,  
$\VECTOR{f}: \mathbb{R}^{N}\rightarrow \mathbb{R}^{M}$ é uma função de valor vectorial, e
$\MATRIX{C} \in \mathbb{R}^{M\times M}$ é uma matriz diagonal, 
então se cumpre que:
\begin{equation}
\frac{\partial ||\VECTOR{f}(\VECTOR{x})-\VECTOR{b}||_{\MATRIX{C}}^2}{\partial \VECTOR{x}} =
2 \MATRIX{J}(\VECTOR{x})^{\transpose}\MATRIX{C}\left[\VECTOR{f}(\VECTOR{x})-\VECTOR{b}\right],
\end{equation}
onde $\MATRIX{J}(\VECTOR{x})$ é a \hyperref[def:jacobian]{\textbf{matriz Jacobiana}} de $\VECTOR{f}(\VECTOR{x})$.

A demostração pode ser vista na Prova \ref{proof:theo:derfxbCfxb0}.
\end{theorem}

\begin{theorem}[Valor aproximado:]\label{theo:derfxbCfxb}
Se 
$\VECTOR{x}\in \mathbb{R}^N$ é um vetor coluna, 
$\VECTOR{b}\in \mathbb{R}^M$ é um vetor coluna,  
$\VECTOR{f}: \mathbb{R}^{N}\rightarrow \mathbb{R}^{M}$ é uma função de valor vectorial, e
$\MATRIX{C} \in \mathbb{R}^{M\times M}$ é uma matriz diagonal, 
então se cumpre que:
\begin{equation}
\frac{\partial ||\VECTOR{f}(\VECTOR{x})-\VECTOR{b}||_{\MATRIX{C}}^2}{\partial \VECTOR{x}} \approx
2 \MATRIX{J}(\VECTOR{p})^{\transpose}\MATRIX{C}\left[\MATRIX{J}(\VECTOR{p})\left(\VECTOR{x} - \VECTOR{p}\right)-\left(\VECTOR{b}-\VECTOR{f}(\VECTOR{p})\right)\right]
\end{equation}

Onde é considerada a aproximação
$\VECTOR{f}(\VECTOR{x})\approx \VECTOR{f}(\VECTOR{p})+\MATRIX{J}(\VECTOR{p})\left(\VECTOR{x}-\VECTOR{p}\right)$,
usando a \hyperref[def:taylor]{\textbf{serie de Taylor}} para funções multivariáveis. Sendo $\VECTOR{p}$ um ponto fixo no domínio de $\VECTOR{f}(\VECTOR{x})$,  ao redor do qual é feita  aproximação
da função $\VECTOR{f}(\VECTOR{x})$,
e $\MATRIX{J}(\VECTOR{p})$ é a \hyperref[def:jacobian]{\textbf{matriz Jacobiana}} de $\VECTOR{f}(\VECTOR{x})$ avaliado no ponto $\VECTOR{p}$.

A demostração pode ser vista na Prova \ref{proof:theo:derfxbCfxb}.
\end{theorem}



%%%%%%%%%%%%%%%%%%%%%%%%%%%%%%%%%%%%%%%%%%%%%%%%%%%%%%%%%%%%%%%%%%%%%%%%%%%%%%%%%%%%%%%
%%%%%%%%%%%%%%%%%%%%%%%%%%%%%%%%%%%%%%%%%%%%%%%%%%%%%%%%%%%%%%%%%%%%%%%%%%%%%%%%%%%%%%%
\section{Derivada de $||\VECTOR{f}(\VECTOR{x})-\VECTOR{b}||_{\MATRIX{C}}^2+\alpha||\VECTOR{x}-\VECTOR{q}||_{\MATRIX{D}}^2$ 
}

\begin{theorem}[Valor exato:]\label{theo:exact:derfxbCfxbxqDxq}
Se 
$\VECTOR{x}\in \mathbb{R}^N$,
$\VECTOR{q}\in \mathbb{R}^N$ e, 
$\VECTOR{b}\in \mathbb{R}^M$ são vetores coluna,  
$\VECTOR{f}: \mathbb{R}^{N}\rightarrow \mathbb{R}^{M}$ é uma função de valor vectorial, e
$\MATRIX{C} \in \mathbb{R}^{M\times M}$ e $\MATRIX{D} \in \mathbb{R}^{N\times N}$ são matrizes diagonais, 
então se cumpre\footnote{A 
demostração pode ser vista na união da Prova \ref{proof:theo:derfxbCfxb0} e o Corolário \ref{coro:derAxbAxb2}.} que:
\begin{equation}
\frac{\partial ||\VECTOR{f}(\VECTOR{x})-\VECTOR{b}||_{\MATRIX{C}}^2+\alpha||\VECTOR{x}-\VECTOR{q}||_{\MATRIX{D}}^2}{\partial \VECTOR{x}} 
= 2 \MATRIX{J}(\VECTOR{x})^{\transpose}\MATRIX{C}\left[\VECTOR{f}(\VECTOR{x})-\VECTOR{b}\right]
+ 2 \alpha\MATRIX{D}\left(\VECTOR{x}-\VECTOR{q}\right),
\end{equation}
onde $\MATRIX{J}(\VECTOR{x})$ é a \hyperref[def:jacobian]{\textbf{matriz Jacobiana}} de $\VECTOR{f}(\VECTOR{x})$.
\end{theorem}



\begin{theorem}[Valor aproximado:]\label{theo:derfxbCfxbxqDxq}
Se 
$\VECTOR{x}\in \mathbb{R}^N$ é um vetor coluna, 
$\VECTOR{b}\in \mathbb{R}^M$ é um vetor coluna,
$\VECTOR{q}\in \mathbb{R}^N$ é um vetor coluna, 
$\VECTOR{f}: \mathbb{R}^{N}\rightarrow \mathbb{R}^{M}$ é uma função de valor vectorial, 
$\MATRIX{C} \in \mathbb{R}^{M\times M}$ é uma matriz diagonal, e
$\MATRIX{D} \in \mathbb{R}^{N\times N}$ é uma matriz diagonal, 
então se cumpre\footnote{A demostração pode ser vista na Prova \ref{proof:theo:derfxbCfxbxqDxq}.} que:
\begin{equation}
\frac{\partial \left(||\VECTOR{f}(\VECTOR{x})-\VECTOR{b}||_{\MATRIX{C}}^2+\alpha||\VECTOR{x}-\VECTOR{q}||_{\MATRIX{D}}^2\right)}{\partial \VECTOR{x}} \approx
2 \MATRIX{J}(\VECTOR{p})^{\transpose}\MATRIX{C}\left\{\MATRIX{J}(\VECTOR{p})\left[\VECTOR{x} - \VECTOR{p}\right]-\left[\VECTOR{b}-\VECTOR{f}(\VECTOR{p})\right] \right\}
+2\alpha\MATRIX{D}\left[\VECTOR{x}-\VECTOR{q}\right]
\end{equation}

Onde é considerada a aproximação
$\VECTOR{f}(\VECTOR{x})\approx \VECTOR{f}(\VECTOR{p})+\MATRIX{J}(\VECTOR{p})\left(\VECTOR{x}-\VECTOR{p}\right)$,
usando a \hyperref[def:taylor]{\textbf{serie de Taylor}} para funções multivariáveis. Sendo $\VECTOR{p}$ um ponto fixo no domínio de $\VECTOR{f}(\VECTOR{x})$,  ao redor do qual é feita  aproximação
da função $\VECTOR{f}(\VECTOR{x})$,
e $\MATRIX{J}(\VECTOR{p})$ é a matriz \hyperref[def:jacobian]{Jacobiana} \cite{Jacobian} de $\VECTOR{f}(\VECTOR{x})$ avaliado no ponto $\VECTOR{p}$.


\end{theorem}


%%%%%%%%%%%%%%%%%%%%%%%%%%%%%%%%%%%%%%%%%%%%%%%%%%%%%%%%%%%%%%%%%%%%%%%%%%%%%%%%%%%%%%%
%%%%%%%%%%%%%%%%%%%%%%%%%%%%%%%%%%%%%%%%%%%%%%%%%%%%%%%%%%%%%%%%%%%%%%%%%%%%%%%%%%%%%%%
\section{Derivada de segundo ordem de $||\VECTOR{f}(\VECTOR{x})-\VECTOR{b}||_{\MATRIX{C}}^2$ 
}



\begin{theorem}[Valor exato:]\label{theo:der2fxbCfxb0}
Se
$\VECTOR{x}\in \mathbb{R}^N$ e 
$\VECTOR{b}\in \mathbb{R}^M$ são vetores coluna,  
$\VECTOR{f}: \mathbb{R}^{N}\rightarrow \mathbb{R}^{M}$ é uma função de valor vectorial,
$\MATRIX{C} \in \mathbb{R}^{M\times M}$ é uma matriz diagonal, e
definimos a função $e(\VECTOR{x})$ como
\begin{equation}
e(\VECTOR{x})= ||\VECTOR{f}(\VECTOR{x})-\VECTOR{b}||_{\MATRIX{C}}^2.
\end{equation}
Então a \hyperref[def:hessian]{\textbf{matriz Hessiana}} $\MATRIX{H}(\VECTOR{x})$ 
de $e(\VECTOR{x})$ é igual\footnote{A demostração pode ser vista na Prova \ref{proof:theo:der2fxbCfxb0}.} a:
\begin{equation}
\MATRIX{H}(\VECTOR{x}) = \frac{\partial}{\partial \VECTOR{x}^{\transpose}}\left(  
\frac{\partial e(\VECTOR{x}) }{\partial \VECTOR{x}} \right) = 2 \MATRIX{B}(\VECTOR{x})
+2 \MATRIX{J}(\VECTOR{x})^{\transpose}\MATRIX{C} \MATRIX{J}(\VECTOR{x}),
\end{equation}
onde 
\begin{equation}
 \MATRIX{B}(\VECTOR{x})=
{\bigcup\limits_{n=1}^{ \rightarrow }}^{N}\left\{ \frac{\partial \MATRIX{J}(\VECTOR{x})^{\transpose} }{\partial x_{n}} \MATRIX{C} \left( \VECTOR{f}(\VECTOR{x})-\VECTOR{b}\right) \right\},
\end{equation}
\end{theorem}

\begin{theorem}[Valor aproximado:]\label{theo:der2fxbCfxb0aprox}
Se 
$\VECTOR{x}\in \mathbb{R}^N$ é  
$\VECTOR{b}\in \mathbb{R}^M$ são vetores coluna,  
$\VECTOR{f}: \mathbb{R}^{N}\rightarrow \mathbb{R}^{M}$ é uma função de valor vectorial,
$\MATRIX{C} \in \mathbb{R}^{M\times M}$ é uma matriz diagonal, e 
definimos a função $e(\VECTOR{x})$ como
\begin{equation}
e(\VECTOR{x})= ||\VECTOR{f}(\VECTOR{x})-\VECTOR{b}||_{\MATRIX{C}}^2.
\end{equation}
Então, para achar uma forma aproximada da \hyperref[def:hessian]{\textbf{matriz Hessiana}} $\MATRIX{H}(\VECTOR{x})$ de $e(\VECTOR{x})$, 
podemos usar a aproximação linear de $\VECTOR{f}(\VECTOR{x})$ por meio da \hyperref[def:taylor]{\textbf{serie de Taylor}} 
para funções multivariáveis\footnote{ $\VECTOR{f}(\VECTOR{x})\approx \VECTOR{f}(\VECTOR{p}) + \MATRIX{J}(\VECTOR{p})$ $\left(\VECTOR{x}-\VECTOR{p}\right)$,
onde $\VECTOR{p}$ é um ponto fixo no domínio de $\VECTOR{f}(\VECTOR{x})$ ao redor do qual é feita  aproximação
da função $\VECTOR{f}(\VECTOR{x})$,
e $\MATRIX{J}(\VECTOR{p})$ é a \hyperref[def:jacobian]{\textbf{matriz Jacobiana}} de $\VECTOR{f}(\VECTOR{x})$ avaliada no ponto $\VECTOR{p}$.}, obtendo\footnote{A demostração pode ser vista na Prova \ref{proof:theo:der2fxbCfxb0aprox}.} o seguinte resultado,
\begin{equation}
\MATRIX{H}(\VECTOR{x}) = \frac{\partial}{\partial \VECTOR{x}^{\transpose}}\left(  
\frac{\partial e(\VECTOR{x}) }{\partial \VECTOR{x}} \right) \approx 
2 \MATRIX{J}(\VECTOR{p})^{\transpose}\MATRIX{C} \MATRIX{J}(\VECTOR{p}).
\end{equation}


\end{theorem}


