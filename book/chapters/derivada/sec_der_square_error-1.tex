
%%%%%%%%%%%%%%%%%%%%%%%%%%%%%%%%%%%%%%%%%%%%%%%%%%%%%%%%%%%%%%%%%%%%%%%%%%%%%%%%%%%%%%%
%%%%%%%%%%%%%%%%%%%%%%%%%%%%%%%%%%%%%%%%%%%%%%%%%%%%%%%%%%%%%%%%%%%%%%%%%%%%%%%%%%%%%%%
\section{Derivada de $\MATRIX{A}\VECTOR{x}$}

\begin{theorem}\label{theo:derAx}
Se 
$\VECTOR{x}\in \mathbb{R}^N$ é um vetor coluna com elementos $x_n$ de modo que
$n\in \mathbb{N}$, $1 \leq n \leq N$, e 
$\MATRIX{A} \in \mathbb{R}^{M\times N}$ é uma matriz com elementos $a_{mn}$ de modo que
$m\in \mathbb{N}$, $1 \leq m \leq M$, 
então se cumpre\footnote{A demostração pode ser vista na Prova \ref{proof:theo:derAx}.} que:
\begin{equation}
\frac{\partial \MATRIX{A}\VECTOR{x}}{\partial x_n}=a_{:n}
\end{equation}
\end{theorem}
~

\begin{corollary}[Derivada de $\MATRIX{A}\VECTOR{x}$ em relação ao vector $\VECTOR{x}^{\transpose}$]\label{coro:derAx1}
Aplicando a Definição \ref{def:deltahor2} junto ao Teorema \ref{theo:derAx}, é
fácil deduzir que:
\begin{equation}
\frac{\partial \MATRIX{A}\VECTOR{x}}{\partial \VECTOR{x}^{\transpose}}=
\MATRIX{A}=
\left[
\begin{matrix}
 a_{:1} &  a_{:2} &  \cdots &  a_{:N}
\end{matrix}
\right]
\end{equation}
\end{corollary}
~

\begin{corollary}[Derivada de $\MATRIX{A}\VECTOR{x}$ em relação ao vector $\VECTOR{x}$]\label{coro:derAx2}
Aplicando a Definição \ref{def:deltaver} junto ao Teorema \ref{theo:derAx}, é
fácil deduzir que:
\begin{equation}
\frac{\partial \MATRIX{A}\VECTOR{x}}{\partial \VECTOR{x}}=\funcvec(\MATRIX{A})=
\left[
\begin{matrix}
 a_{:1} \\  
a_{:2} \\  
\vdots \\  
a_{:n} \\  
\vdots \\  
a_{:N}
\end{matrix}
\right]
\end{equation}
\end{corollary}

\newpage
\begin{corollary}[Derivada de $\VECTOR{a}^{\transpose}\VECTOR{x}$ em relação ao vector $\VECTOR{x}^{\transpose}$]\label{coro:derAx3}
Aplicando a Definição \ref{def:deltahor} junto ao Teorema \ref{theo:derAx} e sabendo que $\VECTOR{a}^{\transpose}$
é um vetor linha, é
fácil deduzir que:
\begin{equation}
\frac{\partial \VECTOR{a}^{\transpose}\VECTOR{x}}{\partial \VECTOR{x}^{\transpose}}=\VECTOR{a}^{\transpose}
\end{equation}
\end{corollary}

\begin{corollary}[Derivada de $\VECTOR{a}^{\transpose}\VECTOR{x}$ em relação ao vector $\VECTOR{x}$]\label{coro:derAx4}
Aplicando a Definição \ref{def:deltaver} junto ao Teorema \ref{theo:derAx} e sabendo que $\VECTOR{a}^{\transpose}$
é um vetor linha, é
fácil deduzir que:
\begin{equation}
\frac{\partial \VECTOR{a}^{\transpose}\VECTOR{x}}{\partial \VECTOR{x}}=\VECTOR{a}
\end{equation}
\end{corollary}
