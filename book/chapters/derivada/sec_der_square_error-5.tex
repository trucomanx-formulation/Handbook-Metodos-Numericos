
%%%%%%%%%%%%%%%%%%%%%%%%%%%%%%%%%%%%%%%%%%%%%%%%%%%%%%%%%%%%%%%%%%%%%%%%%%%%%%%%%%%%%%%
%%%%%%%%%%%%%%%%%%%%%%%%%%%%%%%%%%%%%%%%%%%%%%%%%%%%%%%%%%%%%%%%%%%%%%%%%%%%%%%%%%%%%%%
\section{Derivada de $||\VECTOR{f}(\VECTOR{x})-\VECTOR{b}||_{\MATRIX{C}}^2+\alpha||\VECTOR{x}-\VECTOR{q}||_{\MATRIX{D}}^2$ 
}

\begin{theorem}[Valor exato:]\label{theo:exact:derfxbCfxbxqDxq}
Se 
$\VECTOR{x}\in \mathbb{R}^N$,
$\VECTOR{q}\in \mathbb{R}^N$ e, 
$\VECTOR{b}\in \mathbb{R}^M$ são vetores coluna,  
$\VECTOR{f}: \mathbb{R}^{N}\rightarrow \mathbb{R}^{M}$ é uma função de valor vetorial, e
$\MATRIX{C} \in \mathbb{R}^{M\times M}$ e $\MATRIX{D} \in \mathbb{R}^{N\times N}$ são matrizes diagonais;
então se cumpre\footnote{A 
demonstração pode ser obtida unindo da Prova \ref{proof:theo:derfxbCfxb0} e 
o Corolário \ref{coro:derAxbAxb2}.} 
a Eq. (\ref{eq:theo:exact:derfxbCfxbxqDxq}),
sendo que $\MATRIX{J}(\VECTOR{x})$ é a \hyperref[def:jacobian]{\textbf{matriz Jacobiana}} de $\VECTOR{f}(\VECTOR{x})$.
\begin{equation}\label{eq:theo:exact:derfxbCfxbxqDxq}
\frac{\partial ||\VECTOR{f}(\VECTOR{x})-\VECTOR{b}||_{\MATRIX{C}}^2+\alpha||\VECTOR{x}-\VECTOR{q}||_{\MATRIX{D}}^2}{\partial \VECTOR{x}} 
= 2 \MATRIX{J}(\VECTOR{x})^{\transpose}\MATRIX{C}\left[\VECTOR{f}(\VECTOR{x})-\VECTOR{b}\right]
+ 2 \alpha\MATRIX{D}\left(\VECTOR{x}-\VECTOR{q}\right),
\end{equation}

\end{theorem}



\begin{theorem}[Valor aproximado:]\label{theo:derfxbCfxbxqDxq}
Se 
$\VECTOR{x}\in \mathbb{R}^N$ é um vetor coluna, 
$\VECTOR{b}\in \mathbb{R}^M$ é um vetor coluna,
$\VECTOR{q}\in \mathbb{R}^N$ é um vetor coluna, 
$\VECTOR{f}: \mathbb{R}^{N}\rightarrow \mathbb{R}^{M}$ é uma função vetorial de valor vetorial, 
$\MATRIX{C} \in \mathbb{R}^{M\times M}$ é uma matriz diagonal, e
$\MATRIX{D} \in \mathbb{R}^{N\times N}$ é uma matriz diagonal, 
então se cumpre\footnote{A demonstração pode ser vista na Prova \ref{proof:theo:derfxbCfxbxqDxq}.} que:
\begin{equation}
\frac{\partial \left(||\VECTOR{f}(\VECTOR{x})-\VECTOR{b}||_{\MATRIX{C}}^2+\alpha||\VECTOR{x}-\VECTOR{q}||_{\MATRIX{D}}^2\right)}{\partial \VECTOR{x}} \approx
2 \MATRIX{J}(\VECTOR{p})^{\transpose}\MATRIX{C}\left\{\MATRIX{J}(\VECTOR{p})\left[\VECTOR{x} - \VECTOR{p}\right]-\left[\VECTOR{b}-\VECTOR{f}(\VECTOR{p})\right] \right\}
+2\alpha\MATRIX{D}\left[\VECTOR{x}-\VECTOR{q}\right].
\end{equation}

Sendo $\VECTOR{p}$ um ponto fixo no domínio de $\VECTOR{f}(\VECTOR{x})$,  ao redor do qual é feita  aproximação
usando a \hyperref[def:taylor]{\textbf{série de Taylor}}, 
$\VECTOR{f}(\VECTOR{x})\approx \VECTOR{f}(\VECTOR{p})+\MATRIX{J}(\VECTOR{p})\left(\VECTOR{x}-\VECTOR{p}\right)$,
em que $\MATRIX{J}(\VECTOR{p})$ é a matriz \hyperref[def:jacobian]{Jacobiana} 
de $\VECTOR{f}(\VECTOR{x})$ avaliada em $\VECTOR{p}$.


\end{theorem}

