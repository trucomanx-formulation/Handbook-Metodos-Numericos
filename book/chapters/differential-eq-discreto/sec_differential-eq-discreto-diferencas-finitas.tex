%%%%%%%%%%%%%%%%%%%%%%%%%%%%%%%%%%%%%%%%%%%%%%%%%%%%%%%%%%%%%%%%%%%%%%%%%%%%%%%%%%%%%%%
%%%%%%%%%%%%%%%%%%%%%%%%%%%%%%%%%%%%%%%%%%%%%%%%%%%%%%%%%%%%%%%%%%%%%%%%%%%%%%%%%%%%%%%
%%%%%%%%%%%%%%%%%%%%%%%%%%%%%%%%%%%%%%%%%%%%%%%%%%%%%%%%%%%%%%%%%%%%%%%%%%%%%%%%%%%%%%%
\section{Diferenças finitas}

O método das diferenças finitas nos brinda com uma ferramenta para a resolução de 
equações diferenciais; com esse fim são utilizadas aproximações das derivadas\footnote{As 
derivadas são diferencias infinitesimais.} 
mediante diferenças finitas.

\begin{definition}[Amostragem de uma função:]
\label{def:diferenças-finitas:0}
Dado um conjunto de amostras de um sinal representado pela função $\VECTOR{f}: \mathbb{R} \rightarrow \mathbb{R}^N$, 
adquiridas nos  tempos $t_n$, $\forall n \in \mathbb{Z}$; 
se essas amostras foram adquiridas de forma regular a cada $\ToS$ segundos,
então podemos definir que
\begin{equation}
\VECTOR{f}(t_n)\equiv \VECTOR{f}(n\tau) \equiv \VECTOR{f}[n].
\end{equation}
\end{definition}

\begin{theorem}[Diferença finita para derivada segunda:]
\label{teo:diferenças-finitas:2}
Usando a \hyperref[def:taylor]{\textbf{série de Taylor}}, podemos aproximar 
a segunda derivada de uma função $\VECTOR{f}(t)$, em que\footnote{A
demonstração da aproximação e o erro do cálculo podem ser visto na Prova \ref{proof:teo:diferenças-finitas:2}.}
\begin{equation}
\left.\frac{d^2\VECTOR{f}(t)}{dt^2}\right|_{t=t_n}\equiv \DDTVECTOR{f}(t_n)
\quad \approx \quad
%\frac{\VECTOR{f}(t_n+\ToS)-2\VECTOR{f}(t_n)+\VECTOR{f}(t_n-\ToS)}{\ToS^2} \equiv 
\frac{\VECTOR{f}[n+1]-2\VECTOR{f}[n]+\VECTOR{f}[n-1]}{\ToS^2}.
\end{equation}
\end{theorem}

\begin{tcbattention}
\begin{itemize}
\item Para um valor $\ToS$, tal que $0<|\ToS|<1$, a aproximação da primeira derivada por diferença finita, 
que produz o menor erro, é a que usa ``diferença central'', pois o erro é proporcional a ${\ToS}^2$ 
(Ver Teorema \ref{teo:diferenças-finitas:1}
e a Prova \ref{proof:teo:diferenças-finitas:1}); porém, 
essa aproximação tem o problema de necessitar de amostras futuras para dar um parecer da amostra atual.
\item As aproximações por diferenças progressivas e regressivas têm erros comparáveis
 (Ver Prova \ref{proof:teo:diferenças-finitas:1});
porém as diferenças regressivas têm como vantagem que, para dar um parecer da amostra atual,
você só precisa de amostras anteriores.
\end{itemize}
\end{tcbattention}


\begin{theorem}[Diferença finita para a primeira derivada:]
\label{teo:diferenças-finitas:1}
Usando a \hyperref[def:taylor]{\textbf{série de Taylor}}, podemos aproximar\footnote{As
demonstrações das aproximações e o erro do cálculo podem ser visto na Prova \ref{proof:teo:diferenças-finitas:1}.} 
a primeira derivada de uma função $\VECTOR{f}(t)$.
\begin{itemize}
\item \textbf{Diferença regressiva ou atrasada:}
\begin{equation}
\left.\frac{d\VECTOR{f}(t)}{dt}\right|_{t=t_n}\equiv \DTVECTOR{f}(t_n)
\quad \approx \quad
%\frac{\VECTOR{f}(t_n)-\VECTOR{f}(t_n-\ToS)}{\ToS} \equiv 
\frac{\VECTOR{f}[n]-\VECTOR{f}[n-1]}{\ToS}.
\end{equation}
\item \textbf{Diferença progressiva:}
\begin{equation}
\left.\frac{d\VECTOR{f}(t)}{dt}\right|_{t=t_n}\equiv \DTVECTOR{f}(t_n)
\quad \approx \quad
%\frac{\VECTOR{f}(t_n)-\VECTOR{f}(t_n-\ToS)}{\ToS} \equiv 
\frac{\VECTOR{f}[n+1]-\VECTOR{f}[n]}{\ToS}.
\end{equation}
\item \textbf{Diferença central:}
\begin{equation}
\left.\frac{d\VECTOR{f}(t)}{dt}\right|_{t=t_n}\equiv \DTVECTOR{f}(t_n)
\quad \approx \quad
%\frac{\VECTOR{f}(t_n)-\VECTOR{f}(t_n-\ToS)}{\ToS} \equiv 
\frac{\VECTOR{f}[n+1]-\VECTOR{f}[n-1]}{2\ToS}.
\end{equation}
\end{itemize}
\end{theorem}



%% polinomios interpoladores
%https://www.ufrgs.br/reamat/CalculoNumerico/livro-sci/dn-obtencao_de_formulas_por_polinomios_interpoladores.html
