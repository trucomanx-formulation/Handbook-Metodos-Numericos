%%%%%%%%%%%%%%%%%%%%%%%%%%%%%%%%%%%%%%%%%%%%%%%%%%%%%%%%%%%%%%%%%%%%%%%%%%%%%%%%%%%%%%%
%%%%%%%%%%%%%%%%%%%%%%%%%%%%%%%%%%%%%%%%%%%%%%%%%%%%%%%%%%%%%%%%%%%%%%%%%%%%%%%%%%%%%%%
%%%%%%%%%%%%%%%%%%%%%%%%%%%%%%%%%%%%%%%%%%%%%%%%%%%%%%%%%%%%%%%%%%%%%%%%%%%%%%%%%%%%%%%
\section{Difrenças finitas}

O método das diferenças finitas nos brinda uma ferramenta para a resolução de 
equações diferenciais, com este fim são utilizadas aproximações das derivadas\footnote{As 
derivadas são diferencias infinitesimais.} 
mediante diferenças finitas.

\begin{definition}[Amostragem de uma função:]
\label{def:diferenças-finitas:0}
dado um conjunto de amostras da sinal $\VECTOR{f}(t): \mathbb{R} \rightarrow \mathbb{R}^N$, 
adquiridas nos  tempos $t_n$, $\forall n \in \mathbb{Z}$, 
se estas amostras foram adquiridas de forma regular a cada $\ToS$ segundos,
então podemos definir que,
\begin{equation}
\VECTOR{f}(t_n)\equiv \VECTOR{f}(n\tau) \equiv \VECTOR{f}[n]
\end{equation}
\end{definition}

\begin{theorem}[Diferença finita para derivada segunda:]
\label{teo:diferenças-finitas:2}
Usando a \hyperref[def:taylor]{\textbf{serie de Taylor}} podemos aproximar 
a segunda derivada de uma função $\VECTOR{f}(t)$, onde\footnote{A
demostração da aproximação e o erro do cálculo pode ser visto na Prova \ref{proof:teo:diferenças-finitas:2}}
\begin{equation}
\left.\frac{d^2\VECTOR{f}(t)}{dt^2}\right|_{t=t_n}\equiv \DDTVECTOR{f}(t_n)
\quad \approx \quad
%\frac{\VECTOR{f}(t_n+\ToS)-2\VECTOR{f}(t_n)+\VECTOR{f}(t_n-\ToS)}{\ToS^2} \equiv 
\frac{\VECTOR{f}[n+1]-2\VECTOR{f}[n]+\VECTOR{f}[n-1]}{\ToS^2}  
\end{equation}
\end{theorem}

\begin{tcbattention}
\begin{itemize}
\item Para um valor $\ToS$, tal que $0<|\ToS|<1$, a aproximação da primeira derivada, por diferença finita, 
que produz menor erro é a que usa ``diferença central'' (Ver Teorema \ref{teo:diferenças-finitas:1}); porém, 
esta aproximação tem o problema de necessitar amostras futuras para dar um parecer da amostra atual.
\item As aproximações por diferenças progressivas e regressivas têm erros comparáveis
 (Ver Prova \ref{proof:teo:diferenças-finitas:1});
porém as diferenças regressivas têm a vantagem que para dar um parecer da amostra atual só precisa de amostras anteriores.
\end{itemize}
\end{tcbattention}


\begin{theorem}[Diferença finita para a primeira derivada:]
\label{teo:diferenças-finitas:1}
Usando a \hyperref[def:taylor]{\textbf{serie de Taylor}} podemos aproximar 
a primeira derivada de uma função $\VECTOR{f}(t)$, onde\footnote{As
demostrações das aproximações e o erro do cálculo podem ser visto na Prova \ref{proof:teo:diferenças-finitas:1}.}
\begin{itemize}
\item \textbf{Diferença regressiva ou atrasada:}
\begin{equation}
\left.\frac{d\VECTOR{f}(t)}{dt}\right|_{t=t_n}\equiv \DTVECTOR{f}(t_n)
\quad \approx \quad
%\frac{\VECTOR{f}(t_n)-\VECTOR{f}(t_n-\ToS)}{\ToS} \equiv 
\frac{\VECTOR{f}[n]-\VECTOR{f}[n-1]}{\ToS}
\end{equation}
\item \textbf{Diferença progressiva:}
\begin{equation}
\left.\frac{d\VECTOR{f}(t)}{dt}\right|_{t=t_n}\equiv \DTVECTOR{f}(t_n)
\quad \approx \quad
%\frac{\VECTOR{f}(t_n)-\VECTOR{f}(t_n-\ToS)}{\ToS} \equiv 
\frac{\VECTOR{f}[n+1]-\VECTOR{f}[n]}{\ToS}
\end{equation}
\item \textbf{Diferença central:}
\begin{equation}
\left.\frac{d\VECTOR{f}(t)}{dt}\right|_{t=t_n}\equiv \DTVECTOR{f}(t_n)
\quad \approx \quad
%\frac{\VECTOR{f}(t_n)-\VECTOR{f}(t_n-\ToS)}{\ToS} \equiv 
\frac{\VECTOR{f}[n+1]-\VECTOR{f}[n-1]}{2\ToS}
\end{equation}
\end{itemize}
\end{theorem}



%% polinomios interpoladores
%https://www.ufrgs.br/reamat/CalculoNumerico/livro-sci/dn-obtencao_de_formulas_por_polinomios_interpoladores.html
