%%%%%%%%%%%%%%%%%%%%%%%%%%%%%%%%%%%%%%%%%%%%%%%%%%%%%%%%%%%%%%%%%%%%%%%%%%%%%%%%%%%%%%%
%%%%%%%%%%%%%%%%%%%%%%%%%%%%%%%%%%%%%%%%%%%%%%%%%%%%%%%%%%%%%%%%%%%%%%%%%%%%%%%%%%%%%%%
%%%%%%%%%%%%%%%%%%%%%%%%%%%%%%%%%%%%%%%%%%%%%%%%%%%%%%%%%%%%%%%%%%%%%%%%%%%%%%%%%%%%%%%
\section{Mnemônico vetor nabla $\vec{\triangledown}$}

\begin{notation}[Uso do Mnemônico vetor nabla 
($\overrightarrow{\triangledown}$ ):]
Dado um vetor coluna $\VECTOR{x}\in \mathbb{R}^N$, usaremos o mnemônico\footnote{Falando de um modo rigoroso, 
 $\overrightarrow{\triangledown}$ não é um operador diferencial, 
e sim um mnemônico que nos ajuda a lembrar e representar uma série de operadores diferenciais.} 
vetor nabla ($\overrightarrow{\triangledown}$), como:
\begin{equation}
\overrightarrow{\triangledown}  \equiv \frac{\partial }{\partial \VECTOR{x}} =
\begin{bmatrix}
\frac{\partial  }{\partial x_{1}}\\
\frac{\partial  }{\partial x_{2}}\\
\vdots\\
\frac{\partial  }{\partial x_{n}}\\
\vdots\\
\frac{\partial  }{\partial x_{N}}
\end{bmatrix}
\end{equation}
\end{notation}
~

\begin{tcbattention}
Não deve ser confundido o mnemônico $\overrightarrow{\triangledown}$ 
com o operador $\triangledown$, cujo uso será explicado nas seguintes seções.
\end{tcbattention}
~

\begin{definition}[Derivada 
$\overrightarrow{\triangledown}^{\transpose}\VECTOR{f}(\VECTOR{x})$:]
\label{def:nabla:dot}
Dado 
um vetor coluna $\VECTOR{x}\in \mathbb{R}^N$ e 
uma função vetor coluna $\VECTOR{f}(\VECTOR{x}): \mathbb{R}^N \rightarrow \mathbb{R}^M$, 
definimos que:
\begin{equation}
\overrightarrow{\triangledown}.~\VECTOR{f}(\VECTOR{x}) \equiv
\overrightarrow{\triangledown}^{\transpose}\VECTOR{f}(\VECTOR{x})= 
\frac{\partial f_{1}(\VECTOR{x}) }{\partial x_{1}}+
\frac{\partial f_{2}(\VECTOR{x}) }{\partial x_{2}}+
\hdots+
\frac{\partial f_{n}(\VECTOR{x}) }{\partial x_{n}}+
\hdots+
\frac{\partial f_{N}(\VECTOR{x}) }{\partial x_{N}}=
\sum \limits_{n=1}^N \frac{\partial f_{n}(\VECTOR{x}) }{\partial x_{n}}
\end{equation}

\end{definition}
