
%%%%%%%%%%%%%%%%%%%%%%%%%%%%%%%%%%%%%%%%%%%%%%%%%%%%%%%%%%%%%%%%%%%%%%%%%%%%%%%%%%%%%%%
%%%%%%%%%%%%%%%%%%%%%%%%%%%%%%%%%%%%%%%%%%%%%%%%%%%%%%%%%%%%%%%%%%%%%%%%%%%%%%%%%%%%%%%
%%%%%%%%%%%%%%%%%%%%%%%%%%%%%%%%%%%%%%%%%%%%%%%%%%%%%%%%%%%%%%%%%%%%%%%%%%%%%%%%%%%%%%%
\section{Operador derivada para $e(\VECTOR{x})$, $\VECTOR{f}(\VECTOR{x})$ e  $\MATRIX{G}(\VECTOR{x})$}

%%%%%%%%%%%%%%%%%%%%%%%%%%%%%%%%%%%%%%%%%%%%%%%%%%%%%%%%%%%%%%%%%%%%%%%%%%%%%%%%
\subsection{Derivadas a respeito de $\VECTOR{x}^{\transpose}$}

\begin{definition}[Derivada de 
$e(\VECTOR{x})$ a respeito de $\VECTOR{x}^{\transpose}$:]\label{def:deltahor}
Dado 
um vetor coluna $\VECTOR{x}\in \mathbb{R}^N$ e 
uma função $e(\VECTOR{x}): \mathbb{R}^N \rightarrow \mathbb{R}$,
definimos que:
\begin{equation}
\left[\overrightarrow{\triangledown} e(\VECTOR{x})\right]^{\transpose} \equiv 
\frac{\partial e(\VECTOR{x}) }{\partial \VECTOR{x}^{\transpose}}= 
\left[
\begin{matrix}
\frac{\partial e(\VECTOR{x}) }{\partial x_{1}}&
\frac{\partial e(\VECTOR{x}) }{\partial x_{2}}&
\hdots&
\frac{\partial e(\VECTOR{x}) }{\partial x_{n}}&
\hdots&
\frac{\partial e(\VECTOR{x}) }{\partial x_{N}}
\end{matrix}
\right]= {\bigcup\limits_{n=1}^{\rightarrow}}^{N}{\frac{\partial e(\VECTOR{x}) }{\partial x_{n}}} 
\end{equation}
\end{definition}

\begin{definition}[Derivada de 
$\VECTOR{f}(\VECTOR{x})$ a respeito de $\VECTOR{x}^{\transpose}$:]\label{def:deltahor2}
Dado 
um vetor coluna $\VECTOR{x}\in \mathbb{R}^N$ e 
uma função vetor coluna $\VECTOR{f}(\VECTOR{x}): \mathbb{R}^N \rightarrow \mathbb{R}^M$, 
definimos que:
\begin{equation}
\left[\overrightarrow{\triangledown} \VECTOR{f}^{\transpose}(\VECTOR{x})\right]^{\transpose} \equiv 
\frac{\partial \VECTOR{f}(\VECTOR{x}) }{\partial \VECTOR{x}^{\transpose}}= 
\left[
\begin{matrix}
\frac{\partial \VECTOR{f}(\VECTOR{x}) }{\partial x_{1}}&
\frac{\partial \VECTOR{f}(\VECTOR{x}) }{\partial x_{2}}&
\hdots&
\frac{\partial \VECTOR{f}(\VECTOR{x}) }{\partial x_{n}}&
\hdots&
\frac{\partial \VECTOR{f}(\VECTOR{x}) }{\partial x_{N}}
\end{matrix}
\right]= {\bigcup\limits_{n=1}^{\rightarrow}}^{N}{\frac{\partial \VECTOR{f}(\VECTOR{x}) }{\partial x_{n}}} 
\end{equation}
\end{definition}

\begin{definition}[Derivada de 
$\MATRIX{G}(\VECTOR{x})$ a respeito de $\VECTOR{x}^{\transpose}$:]\label{def:deltahor3}
Dado 
um vetor coluna $\VECTOR{x}\in \mathbb{R}^N$ e 
uma função $\MATRIX{G}(\VECTOR{x}): \mathbb{R}^N \rightarrow \mathbb{R}^{M\times L}$, 
definimos que:
\begin{equation}
\frac{\partial \MATRIX{G}(\VECTOR{x}) }{\partial \VECTOR{x}^{\transpose}}= 
\left[
\begin{matrix}
\frac{\partial \MATRIX{G}(\VECTOR{x}) }{\partial x_{1}}&
\frac{\partial \MATRIX{G}(\VECTOR{x}) }{\partial x_{2}}&
\hdots&
\frac{\partial \MATRIX{G}(\VECTOR{x}) }{\partial x_{n}}&
\hdots&
\frac{\partial \MATRIX{G}(\VECTOR{x}) }{\partial x_{N}}
\end{matrix}
\right]= {\bigcup\limits_{n=1}^{\rightarrow}}^{N}{\frac{\partial \MATRIX{G}(\VECTOR{x}) }{\partial x_{n}}}.
\end{equation}
\end{definition}

\begin{comment}
Assim, obtemos o vetor linha
$\frac{\partial e(\VECTOR{x}) }{\partial \VECTOR{x}^{\transpose}} \in \mathbb{R}^{1\times N}$,
as matrizes
$\frac{\partial \VECTOR{f}(\VECTOR{x}) }{\partial \VECTOR{x}^{\transpose}} \in \mathbb{R}^{M \times N}$ e
$\frac{\partial \VECTOR{g}(\VECTOR{x}) }{\partial \VECTOR{x}^{\transpose}} \in \mathbb{R}^{M \times (LN)}$.
\end{comment}

\begin{example}[Uso da Definição \ref{def:deltahor}:]
Conhecida a função $e(\VECTOR{x}): \mathbb{R}^2 \rightarrow \mathbb{R}$, podemos calcular,
\begin{equation}
e(\VECTOR{x})=x_1^2+x_1 x_2+x_2^2
\qquad \rightarrow \qquad
\frac{\partial e(\VECTOR{x}) }{\partial \VECTOR{x}^{\transpose}}=
\left[ 2 x_1+x_2 \quad x_1 + 2 x_2\right]
\end{equation}
\end{example}


\begin{example}[Uso da Definição \ref{def:deltahor2}:]
Conhecida a função $\VECTOR{f}(\VECTOR{x}): \mathbb{R}^2 \rightarrow \mathbb{R}^4$, podemos calcular,
\begin{equation}
\VECTOR{f}(\VECTOR{x})=
\begin{bmatrix}
x_1^2+x_2\\
x_1+x_2^2\\
x_1 x_2 \\
x_1^2+x_2^2
\end{bmatrix}
\qquad \rightarrow \qquad
\frac{\partial \VECTOR{f}(\VECTOR{x}) }{\partial \VECTOR{x}^{\transpose}}= 
\begin{bmatrix}
2 x_1 & 1\\
1     & 2 x_2\\
x_2   & x_1\\
2 x_1 & 2 x_2
\end{bmatrix}
\end{equation}
\end{example}

\begin{example}[Uso da Definição \ref{def:deltahor3}:]
Conhecida a função $\MATRIX{G}(\VECTOR{x}): \mathbb{R}^2 \rightarrow \mathbb{R}^{3 \times 2}$, podemos calcular,
\begin{equation}
\MATRIX{G}(\VECTOR{x})=
\begin{bmatrix}
x_1^2       & x_1 x_2\\
x_1 x_2     & x_2^2\\
x_1^2 + x_2 & x_1 + x_2^2\\
x_1 + x_2   & x_1 + x_2
\end{bmatrix}
\qquad \rightarrow \qquad
\frac{\partial \MATRIX{G}(\VECTOR{x}) }{\partial \VECTOR{x}^{\transpose}}= 
\begin{bmatrix}
2 x_1 & x_2 & 0   &   x_1\\
  x_2 & 0   & x_1 & 2 x_2\\
2 x_1 & 1   & 1   & 2 x_2\\
    1 & 1   & 1   & 1
\end{bmatrix}
\end{equation}
\end{example}
%%%%%%%%%%%%%%%%%%%%%%%%%%%%%%%%%%%%%%%%%%%%%%%%%%%%%%%%%%%%%%%%%%%%%%%%%%%%%%%%
\subsection{Derivadas a respeito de $\VECTOR{x}$}

\begin{definition}[Derivada de 
$e(\VECTOR{x})$ a respeito de $\VECTOR{x}$:]\label{def:deltaver}
Dado 
um vetor coluna $\VECTOR{x}\in \mathbb{R}^N$ e 
uma função $e(\VECTOR{x}): \mathbb{R}^N \rightarrow \mathbb{R}$,
definimos que:
\begin{equation}
\overrightarrow{\triangledown} e(\VECTOR{x}) \equiv 
\frac{\partial e(\VECTOR{x}) }{\partial \VECTOR{x}}= 
\left[
\begin{matrix}
\frac{\partial e(\VECTOR{x}) }{\partial x_{1}} \\
\frac{\partial e(\VECTOR{x}) }{\partial x_{2}} \\
\vdots \\
\frac{\partial e(\VECTOR{x}) }{\partial x_{n}} \\
\vdots \\
\frac{\partial e(\VECTOR{x}) }{\partial x_{N-1}} \\
\frac{\partial e(\VECTOR{x}) }{\partial x_{N}} 
\end{matrix}
\right] = 
{\bigcup\limits_{n=1}^{\downarrow}}^{N}{\frac{\partial e(\VECTOR{x}) }{\partial x_{n}}} 
\end{equation}
\end{definition}

\begin{definition}[Derivada de 
$\VECTOR{f}(\VECTOR{x})$ a respeito de $\VECTOR{x}$:]\label{def:deltaver2}
Dado 
um vetor coluna $\VECTOR{x}\in \mathbb{R}^N$ e 
uma função vetor coluna $\VECTOR{f}(\VECTOR{x}): \mathbb{R}^N \rightarrow \mathbb{R}^M$, 
definimos que:
\begin{equation}
\frac{\partial \VECTOR{f}(\VECTOR{x}) }{\partial \VECTOR{x}}= 
\left[
\begin{matrix}
\frac{\partial \VECTOR{f}(\VECTOR{x}) }{\partial x_{1}} \\
\frac{\partial \VECTOR{f}(\VECTOR{x}) }{\partial x_{2}} \\
\vdots \\
\frac{\partial \VECTOR{f}(\VECTOR{x}) }{\partial x_{n}} \\
\vdots \\
%\frac{\partial \VECTOR{f}(\VECTOR{x}) }{\partial x_{N-1}} \\
\frac{\partial \VECTOR{f}(\VECTOR{x}) }{\partial x_{N}}
\end{matrix}
\right] =  
{\bigcup\limits_{n=1}^{\downarrow}}^{N}{\frac{\partial \VECTOR{f}(\VECTOR{x}) }{\partial x_{n}}}
\end{equation}
\end{definition}

\begin{definition}[Derivada de 
$\MATRIX{G}(\VECTOR{x})$ a respeito de $\VECTOR{x}$:]\label{def:deltaver3}
Dado 
um vetor coluna $\VECTOR{x}\in \mathbb{R}^N$ e 
uma função $\MATRIX{G}(\VECTOR{x}): \mathbb{R}^N \rightarrow \mathbb{R}^{M\times L}$, 
definimos que:
\begin{equation}
\frac{\partial \MATRIX{G}(\VECTOR{x}) }{\partial \VECTOR{x}}= 
\left[
\begin{matrix}
\frac{\partial \MATRIX{G}(\VECTOR{x}) }{\partial x_{1}} \\
\frac{\partial \MATRIX{G}(\VECTOR{x}) }{\partial x_{2}} \\
\vdots \\
\frac{\partial \MATRIX{G}(\VECTOR{x}) }{\partial x_{n}} \\
\vdots \\
%\frac{\partial \MATRIX{G}(\VECTOR{x}) }{\partial x_{N-1}} \\
\frac{\partial \MATRIX{G}(\VECTOR{x}) }{\partial x_{N}}
\end{matrix}
\right] = {\bigcup\limits_{n=1}^{\downarrow}}^{N}{\frac{\partial \MATRIX{G}(\VECTOR{x}) }{\partial x_{n}}}
\end{equation}
\end{definition}

\begin{comment}
Assim, 
$\frac{\partial e(\VECTOR{x}) }{\partial \VECTOR{x}} \in \mathbb{R}^{N \times 1}$,
$\frac{\partial \VECTOR{f}(\VECTOR{x}) }{\partial \VECTOR{x}} \in \mathbb{R}^{(MN) \times 1}$ e
$\frac{\partial \VECTOR{g}(\VECTOR{x}) }{\partial \VECTOR{x}} \in \mathbb{R}^{(MN) \times L}$.
\end{comment}


\begin{corollary}[Igualdade das derivadas cruzadas]\label{cor:derder}
Dado,
o vetor coluna $\VECTOR{x}\in \mathbb{R}^N$, 
a função $e(\VECTOR{x}): \mathbb{R}^N \rightarrow \mathbb{R}$; e
considerando as Definições \ref{def:deltahor}, \ref {def:deltaver} e o 
teorema da igualdade das derivadas cruzadas\footnote{Tambem conhecido como 
``teorema de Clairaut'' \cite[pp. 885]{stewart2008calculus},
 ``teorema de Clairaut-Schwarz'' \cite[pp. 311]{telles2015matematica}, ou ``teorema de Schwarz''.}; 
podemos deduzir que:
\begin{equation}
 \frac{\partial }{\partial \VECTOR{x}} \left( \frac{\partial e(\VECTOR{x} )}{\partial \VECTOR{x}^{\transpose}} \right) = 
\frac{\partial }{\partial \VECTOR{x}^{\transpose}} \left( \frac{\partial e(\VECTOR{x} )}{\partial \VECTOR{x}} \right)
\end{equation}
\end{corollary}

%%%%%%%%%%%%%%%%%%%%%%%%%%%%%%%%%%%%%%%%%%%%%%%%%%%%%%%%%%%%%%%%%%%%%%%%%%%%%%%%

\begin{example}[Uso da Definição \ref{def:deltaver}:]
Conhecida a função $e(\VECTOR{x}): \mathbb{R}^2 \rightarrow \mathbb{R}$, podemos calcular,
\begin{equation}
e(\VECTOR{x})=x_1^2+x_1 x_2+x_2^2
\qquad \rightarrow \qquad
\frac{\partial e(\VECTOR{x}) }{\partial \VECTOR{x}}=
\begin{bmatrix}
 2 x_1 + x_2 \\
 x_1 + 2 x_2
\end{bmatrix}
\end{equation}
\end{example}


\begin{example}[Uso da Definição \ref{def:deltaver2}:]
Conhecida a função $\VECTOR{f}(\VECTOR{x}): \mathbb{R}^2 \rightarrow \mathbb{R}^4$, podemos calcular,
\begin{equation}
\VECTOR{f}(\VECTOR{x})=
\begin{bmatrix}
x_1^2+x_2\\
x_1+x_2^2\\
x_1 x_2 \\
x_1^2+x_2^2
\end{bmatrix}
\qquad \rightarrow \qquad
\frac{\partial \VECTOR{f}(\VECTOR{x}) }{\partial \VECTOR{x}}= 
\begin{bmatrix}
2 x_1 \\
1     \\
x_2   \\
2 x_1 \\
1     \\
2 x_2 \\
x_1   \\
2 x_2
\end{bmatrix}
\end{equation}
\end{example}


\begin{example}[Uso da Definição \ref{def:deltaver3}:]
Conhecida a função $\MATRIX{G}(\VECTOR{x}): \mathbb{R}^2 \rightarrow \mathbb{R}^{3 \times 2}$, podemos calcular,
\begin{equation}
\MATRIX{G}(\VECTOR{x})=
\begin{bmatrix}
x_1^2       & x_1 x_2\\
x_1 x_2     & x_2^2\\
x_1^2 + x_2 & x_1 + x_2^2\\
x_1 + x_2   & x_1 + x_2
\end{bmatrix}
\qquad \rightarrow \qquad
\frac{\partial \MATRIX{G}(\VECTOR{x}) }{\partial \VECTOR{x}}= 
\begin{bmatrix}
2 x_1 & x_2\\ 
  x_2 & 0  \\ 
2 x_1 & 1  \\ 
    1 & 1  \\ 
0   &   x_1\\
x_1 & 2 x_2\\
1   & 2 x_2\\
1   & 1
\end{bmatrix}
\end{equation}
\end{example}

\begin{example}[Uso do Corolário \ref{cor:derder}:]
Conhecida a função $e(\VECTOR{x}): \mathbb{R}^2 \rightarrow \mathbb{R}$, podemos calcular,
\begin{equation}
e(\VECTOR{x})=x_1^2+x_1 x_2+x_2^2
\qquad \rightarrow \qquad
\frac{\partial^2 e(\VECTOR{x}) }{\partial \VECTOR{x} \partial \VECTOR{x}^{\transpose}}=
\frac{\partial^2 e(\VECTOR{x}) }{\partial \VECTOR{x}^{\transpose} \partial \VECTOR{x}}=
\begin{bmatrix}
 2 & 1 \\
 1 & 2
\end{bmatrix}
\end{equation}
\end{example}

\index{Schwarz, Karl Hermann Amandus}
\begin{elaboracion}[title=Karl Hermann Amandus Schwarz (1843-1921), width= 0.99\linewidth]
Matemático nascido em Jerzmanowa (Polônia), porém  educado na Alemanha.
Foi membro da Academia de Ciências de Berlim e professor da Universidade de Berlim,
Ele é lembrado pelo método aditivo de Schwarz,lema de Schwarz, o mapeamento de Schwarz-Christoffel, 
o teorema de Schwarz (teorema da igualdade das derivadas cruzadas), o teorema Schwarz-Ahlfors-Pick, entre outros \cite[pp. 297]{agarwal2014creators}.
\end{elaboracion}

