\section{Minimização de $||\MATRIX{A}\VECTOR{x}-\VECTOR{b}||_{\MATRIX{C}}^2$
}

\begin{theorem}\label{theo:minAxbCAxb}
Dados,
um vetor coluna $\VECTOR{x}\in \mathbb{R}^N$, 
um vetor coluna $\VECTOR{b}\in \mathbb{R}^M$,  
uma matriz $\MATRIX{A} \in \mathbb{R}^{M\times N}$, 
uma matriz diagonal $\MATRIX{C} \in \mathbb{R}^{M\times M}$, e 
definida a Eq. (\ref{eq:minAxbCAxb1}),
\begin{align}\label{eq:minAxbCAxb1}
e(\VECTOR{x}) & = ||\MATRIX{A}\VECTOR{x}-\VECTOR{b}||_{\MATRIX{C}}^2 \\
              & = (\MATRIX{A}\VECTOR{x}-\VECTOR{b})^{\transpose}\MATRIX{C}(\MATRIX{A}\VECTOR{x}-\VECTOR{b}).
\end{align}
Se desejamos ter o valor $\VECTOR{\hat{x}}$ que minimiza o escalar $e(\VECTOR{\hat{x}})$,
devemos usar\footnote{A demostração pode ser vista na Prova \ref{proof:theo:minAxbCAxb}.} a Eq. (\ref{eq:minAxbCAxb2}),
\begin{equation}\label{eq:minAxbCAxb2}
\VECTOR{\hat{x}} =
\left[ \MATRIX{A}^{\transpose}\MATRIX{C} \MATRIX{A} \right]^{-1}\MATRIX{A}^{\transpose}\MATRIX{C}\VECTOR{b}.
\end{equation}
Assim, o mínimo existe só sim $\MATRIX{A}^{\transpose}\MATRIX{C} \MATRIX{A}$ tem inversa.
\end{theorem}

\index{Pseudo-inversa de Moore-Penrose}
\index{Problema inverso!Linear}
\index{Minimização do erro quadrático!Linear}

