\section{Minimização de $||\VECTOR{f}(\VECTOR{x})-\VECTOR{b}||_{\MATRIX{C}}^2+\alpha||\VECTOR{x}-\VECTOR{q}||_{\MATRIX{D}}^2$}


\index{Minimização, métodos!Regularização de Tikhonov}
\index{Problema inverso!Não linear}
\index{Minimização do erro quadrático!Não linear}

\begin{theorem}[Solução iterativa]\label{theo:minfxbCfxbaxqaxq}
Dados,
os vetores coluna $\VECTOR{x}\in \mathbb{R}^N$, $\VECTOR{b}\in \mathbb{R}^M$ e $\VECTOR{q}\in \mathbb{R}^N$,  
uma função $\VECTOR{f}:\mathbb{R}^{N} \rightarrow \mathbb{R}^{M}$, 
as matrizes diagonais $\MATRIX{C} \in \mathbb{R}^{M\times M}$ e $\MATRIX{D} \in \mathbb{R}^{N\times N}$, e 
definida a Eq. (\ref{eq:minfxbCfxbaxqaxq1}),
\begin{align}\label{eq:minfxbCfxbaxqaxq1}
e(\VECTOR{x}) &=||\VECTOR{f}(\VECTOR{x})-\VECTOR{b}||_{\MATRIX{C}}^2+\alpha||\VECTOR{x}-\VECTOR{q}||_{\MATRIX{D}}^2\\
              & = (\VECTOR{f}(\VECTOR{x})-\VECTOR{b})^{\transpose}\MATRIX{C}(\VECTOR{f}(\VECTOR{x})-\VECTOR{b})+\alpha(\VECTOR{x}-\VECTOR{q})^{\transpose}\MATRIX{D}(\VECTOR{x}-\VECTOR{q}).
\end{align}
Se desejamos ter o ponto $\VECTOR{\hat{x}}$ que minimiza o escalar $e(\VECTOR{\hat{x}})$,
podemos achar este ponto usando iterativamente a Eq. (\ref{eq:minfxbCfxbaxqaxq2}),
onde  $\MATRIX{J}(\VECTOR{x})$ é a \hyperref[def:jacobian]{\textbf{matriz Jacobiana}}  de $\VECTOR{f}(\VECTOR{x})$.
\begin{equation}\label{eq:minfxbCfxbaxqaxq2}
\VECTOR{x}_{k}\leftarrow \VECTOR{x}_{k-1}+
\left[ \MATRIX{J}(\VECTOR{x}_{k-1})^{\transpose}\MATRIX{C} \MATRIX{J}(\VECTOR{x}_{k-1}) -\alpha\MATRIX{D} \right]^{-1}
 \left\{\MATRIX{J}(\VECTOR{x}_{k-1})^{\transpose}\MATRIX{C}\left[\VECTOR{f}(\VECTOR{x}_{k-1})-\VECTOR{b}\right]+\alpha\MATRIX{D}\left[\VECTOR{x}_{k-1}-\VECTOR{q}\right]\right\}
\end{equation}

Assim, $\VECTOR{\hat{x}}$ pode ser achado 
iniciando a Eq. (\ref{eq:minfxbCfxbaxqaxq2}) desde um $\VECTOR{x}_{0}$ qualquer, 
ate que $\VECTOR{x}_{k}$ seja muito próximo a $\VECTOR{x}_{k-1}$,
onde se declara que $\VECTOR{\hat{x}} \approx \VECTOR{x}_{k}$,
que corresponde a um mínimo\footnote{\label{ref:minfx}A
demostração pode ser vista na Prova \ref{proof:theo:minfxbCfxbaxqd}.} de $e(\VECTOR{x})$,
sem aclarar se é local ou global.


\textbf{Considerções:}

\begin{itemize}
\item É interessante verificar na Eq. (\ref{eq:minfxbCfxbaxqaxq2}) 
se  $\MATRIX{J}(\VECTOR{x}_{k-1}=\VECTOR{q}) = \MATRIX{0}$,
pois indica que existe\footref{ref:minfx} um ponto de inflexão 
(máximo, mínimo ou ponto de sela) em $e(\VECTOR{x}_{k-1}=\VECTOR{q})$;
consequentemente poderiamos ter achado um mínimo\footnote{\label{foot:labq}É 
interesante verificar o ponto $\VECTOR{q}$, uma vez só, 
antes de iniciar a busca iterativa.}.
\item A busca iterativa da Eq. (\ref{eq:minfxbCfxbaxqaxq2}) é considerada falida quando 
$\MATRIX{J}(\VECTOR{x}_{k-1})^{\transpose}\MATRIX{C} \MATRIX{J}(\VECTOR{x}_{k-1}) -\alpha\MATRIX{D}$
não tem inversa.
\end{itemize}

\end{theorem} 

