\section{Minimização de $||\VECTOR{f}(\VECTOR{x})-\VECTOR{b}||_{\MATRIX{C}}^2+\alpha||\VECTOR{x}-\VECTOR{x}_{last}||_{\MATRIX{D}}^2$}


\index{Minimização, métodos!Regularização de Tikhonov}
\index{Problema inverso!Não linear}
\index{Minimização do erro quadrático!Não linear}

\begin{theorem}[Solução iterativa]\label{theo:minfxbCfxbaxoaxo}
Dados,
os vetores coluna $\VECTOR{x}\in \mathbb{R}^N$, $\VECTOR{b}\in \mathbb{R}^M$ e $\VECTOR{x}_{last}\in \mathbb{R}^N$,  
uma função $\VECTOR{f}:\mathbb{R}^{N} \rightarrow \mathbb{R}^{M}$, 
as matrizes diagonais $\MATRIX{C} \in \mathbb{R}^{M\times M}$ e $\MATRIX{D} \in \mathbb{R}^{N\times N}$, e 
definida a Eq. (\ref{eq:minfxbCfxbaxoaxo1}),
\begin{equation}\label{eq:minfxbCfxbaxoaxo1}
e(\VECTOR{x})=||\VECTOR{f}(\VECTOR{x})-\VECTOR{b}||_{\MATRIX{C}}^2+\alpha||\VECTOR{x}-\VECTOR{x}_{last}||_{\MATRIX{D}}^2,
\end{equation}
tendo en consideração que $\VECTOR{x}_{last}$ é uma constante equivalente a $\VECTOR{x}_{k-1}$
numa busca iterativa; é dizer, o segundo somando na Eq. (\ref{eq:minfxbCfxbaxoaxo1}) 
procura minimizar $||\VECTOR{x}_{k}-\VECTOR{x}_{k-1}||_{\MATRIX{D}}^2$.


Se desejamos ter o ponto $\VECTOR{\hat{x}}$ que minimiza o escalar $e(\VECTOR{\hat{x}})$,
podemos achar este ponto usando iterativamente a Eq. (\ref{eq:minfxbCfxbaxoaxo2}),
onde  $\MATRIX{J}(\VECTOR{x})$ é a \hyperref[def:jacobian]{\textbf{matriz Jacobiana}}  de $\VECTOR{f}(\VECTOR{x})$.
\begin{equation}\label{eq:minfxbCfxbaxoaxo2}
\VECTOR{x}_{k}\leftarrow \VECTOR{x}_{k-1}+
\left[ \MATRIX{J}(\VECTOR{x}_{k-1})^{\transpose}\MATRIX{C} \MATRIX{J}(\VECTOR{x}_{k-1}) +\alpha\MATRIX{D} \right]^{-1}
 \left[\MATRIX{J}(\VECTOR{x}_{k-1})^{\transpose}\MATRIX{C}\left\{\VECTOR{b}-\VECTOR{f}(\VECTOR{x}_{k-1})\right\}\right] 
\end{equation}
Assim, $\VECTOR{\hat{x}}$ pode ser achado 
iniciando a Eq. (\ref{eq:minfxbCfxbaxoaxo2}) desde um $\VECTOR{x}_{0}$ qualquer, 
ate que $\VECTOR{x}_{k}$ seja muito próximo a $\VECTOR{x}_{k-1}$,
onde se declara que $\VECTOR{\hat{x}} \approx \VECTOR{x}_{k}$,
que corresponde a um mínimo\footnote{\label{ref:minfx}A
demostração pode ser vista na Prova \ref{proof:theo:minfxbCfxbaxod}.} de $e(\VECTOR{x})$,
sem aclarar se é local ou global.


\textbf{Considerções:}

\begin{itemize}
\item É interessante verificar sempre na Eq. (\ref{eq:minfxbCfxbaxoaxo2}) 
se  $\MATRIX{J}(\VECTOR{x}_{k-1}) = \MATRIX{0}$,
pois indica que existe\footref{ref:minfx} um ponto de inflexão 
(máximo, mínimo ou ponto de sela) em $e(\VECTOR{x}_{k-1})$;
consequentemente poderiamos ter achado um mínimo.
\item A busca iterativa da Eq. (\ref{eq:minfxbCfxbaxoaxo2}) é considerada falida quando 
$\MATRIX{J}(\VECTOR{x}_{k-1})^{\transpose}\MATRIX{C} \MATRIX{J}(\VECTOR{x}_{k-1}) -\alpha\MATRIX{D}$
não tem inversa.
\end{itemize}


\end{theorem} 

