\section{Provas dos teoremas}
 
%%%%%%%%%%%%%%%%%%%%%%%%%%%%%%%%%%%%%%%%%%%%%%%%%%%%%%%%%%%%%%%%%%%%%%%%%%%%%%%%%%%%%%%
%%%%%%%%%%%%%%%%%%%%%%%%%%%%%%%%%%%%%%%%%%%%%%%%%%%%%%%%%%%%%%%%%%%%%%%%%%%%%%%%%%%%%%%
\begin{myproofT}[Prova do Teorema \ref{theo:minhxhx}]\label{proof:theo:minhxhx}
Dados,
um escalar $x \in \mathbb{R}$, 
um escalar $a \in \mathbb{R}$,  
uma função $h:\mathbb{R} \rightarrow \mathbb{R}$, e 
definida a Eq. (\ref{eq:proof:minhxhx0}),
\begin{equation}\label{eq:proof:minhxhx0}
e(x)=||h(x)-a||^2.
\end{equation}
Para achar o valor  $\hat{x}$ que gere o menor valor de $e(x)$, é aplicado
o critério que um ponto crítico (máximo, mínimo ou ponto de inflexão) de $e(x)$ pode ser achado quando 
$\left. \frac{\partial e(x)}{\partial x }\right|_{x=\hat{x}} \equiv e'(\hat{x}) =0$.
Assim, 
usando o Teorema \ref{theo:derfxbCfxb0} podemos 
rescrever esta igualdade como a Eq. (\ref{eq:proof:minhxhxe}),
\begin{equation}\label{eq:proof:minhxhxe}
2  h'(x) \left[h(x) -a\right] = e'(x)=0,
\end{equation}
Da Eq. (\ref{eq:proof:minhxhxe}), observamos 
que existem duas formas de achar um ponto crítico em $e(x)$,
\begin{itemize}
 \item a primeira é achando um valor $x^{*}$ tal que $h'(x^{*})=0$, e
 \item a segunda forma é achando um valor $\hat{x}$ tal que $h(\hat{x}) -a=0$;
\end{itemize}
de isto se deduz que, um valor $x^{*}$ que provoque $h'(x^{*})=0$, representa um ponto crítico de $e(x)$ e $h(x)$;
porem, um ponto $\hat{x}$ que provoque $h(\hat{x})-a=0$, representa um ponto crítico de $e(x)$, mas não
necessariamente de $h(x)$. Assim, dada a natureza positiva de $e(x)$, 
os pontos críticos diferente dos existentes em $h(x)$,
são obrigatoriamente mínimos absolutos, pois provocam $e(\hat{x})$=0.

Por outro lado, usando o Teorema \ref{theo:derfxbCfxb} podemos 
rescrever a Eq. (\ref{eq:proof:minhxhxe}) de forma aproximada como a Eq. (\ref{eq:proof:minhxhx1}),
\begin{equation}\label{eq:proof:minhxhx1}
2  h'(p) \left[\left(h(p)-a\right) + h'(p) \left(\hat{x} - p\right)\right] \approx
e'(\hat{x})=0.
\end{equation}
de modo que, se consideramos $h'(p)\neq 0$, a equação pode ser rescrita como:
%um ponto crítico $\hat{x}$ de $e(x)$ pode ser achado aproximadamente quando:
\begin{equation}\label{eq:proof:minhxhx2}
\hat{x} \approx p - \frac{\left(h(p)-a\right)}{h'(p)}.
\end{equation}
Isto quer dizer que a Eq. \ref{eq:proof:minhxhx2}, \textbf{serve só para
achar pontos críticos de} $e(x)$ \textbf{que não pertençam a} $h(x)$.
Sendo este obrigatoriamente um ponto onde $h(\hat{x})-a\approx 0$, 
equivalente a um mínimo absoluto.

Assim, quanto mais próximo a $\hat{x}$ seja  o valor $p$, 
a Eq. (\ref{eq:proof:minhxhx2}) fica mais próximo a uma igualdade. Por outro lado,
a equação nos indica que dado um ponto  $p$ qualquer,
$p - \left[ h'(p) \right]^{-1}\left(h(p)-a\right)$
é um ponto próximo de $p$  na direção de um mínimo de $e(x)$.
De modo que, um bom critério para procurar um mínimo é seguir a seguinte 
equação iterativa,
\begin{equation}\label{eq:proof:minhxhx3}
p_{k} \leftarrow p_{k-1} - \frac{ \left(h(p_{k-1})-a\right)}{h'(p_{k-1})},
\end{equation}
iniciando desde um $p_{0}$ qualquer, ate que $p_{k}$ seja muito próximo a $p_{k-1}$,
onde se declara que $\hat{x} \approx p_{k}$. 


porem deve-se ter cuidado com valores de $p_{k}$ que provoquem 
$0<|h'(p_{k})|<\lim\limits_{\varepsilon \rightarrow +0}\varepsilon$,
pois estes valores tendem a divergir. Por outro lado sim se acha um valor 
de $p_{k}$ com $h'(p_{k})= 0$, deve ser
corroborado se este ponto tratasse de um máximo ou mínimo, relativo ou absoluto, ou ponto de inflexão;
usando algum método, por exemplo estudando o comportamento 
de $e(x)$ ou analisando a matriz hessiana de $e(x)$ avaliada em $p_{k}$.

\end{myproofT}





%%%%%%%%%%%%%%%%%%%%%%%%%%%%%%%%%%%%%%%%%%%%%%%%%%%%%%%%%%%%%%%%%%%%%%%%%%%%%%%%%%%%%%%
%%%%%%%%%%%%%%%%%%%%%%%%%%%%%%%%%%%%%%%%%%%%%%%%%%%%%%%%%%%%%%%%%%%%%%%%%%%%%%%%%%%%%%%
\begin{myproofT}[Prova do Teorema \ref{theo:minhxhxxbxb}]\label{proof:theo:minhxhxxbxb}
Dados,
um escalar $x \in \mathbb{R}$, 
um escalar $a \in \mathbb{R}$,
um escalar $\alpha \in \mathbb{R}^{+}$,
um escalar $b \in \mathbb{R}$,
uma função $h:\mathbb{R} \rightarrow \mathbb{R}$, e 
definida a Eq. (\ref{eq:proof:minhxhxxbxb0}),
\begin{equation}\label{eq:proof:minhxhxxbxb0}
e(x)=||h(x)-a||^2+\alpha ||x-b||^2.
\end{equation}
Para achar o valor  $\hat{x}$ que gere o menor valor de $e(x)$, é aplicado
o critério que um ponto crítico (máximo, mínimo ou ponto de inflexão) de $e(x)$ pode ser achado quando 
$\left. \frac{\partial e(x)}{\partial x }\right|_{x=\hat{x}} \equiv e'(\hat{x}) =0$.
Assim, 
usando o Teorema \ref{theo:derfxbCfxb0} e \ref{theo:derAxbAxb} podemos 
rescrever esta igualdade como a Eq. (\ref{eq:proof:minhxhxxbxbe}),
\begin{equation}\label{eq:proof:minhxhxxbxbe}
2  h'(x) \left[h(x) -a\right]+2\alpha (x-b)= e'(x)=0,
\end{equation}
Da Eq. (\ref{eq:proof:minhxhxxbxbe}), observamos 
que existem duas formas de achar um ponto crítico em $e(x)$,
\begin{itemize}
 \item a primeira é sim $h'(x)$ tem um fator $(x-b)$ tal que $h'(b)=0$, e
 \item a segunda forma é achando um valor $\hat{x}$ tal que $e'(\hat{x})=0$;
\end{itemize}
de isto se deduz que se $h'(x)$ tem um fator $(x-b)$ então, $x=b$ representa um ponto crítico de $e(x)$ e $h(x)$;
porem, existe um ponto $\hat{x}$ que provoca $e'(\hat{x})=0$, representa um ponto crítico de $e(x)$, exclusivamente. 
Assim, dada a natureza positiva de $e(x)$,este é obrigatoriamente um mínimo absoluto de $e(x)$.

Por outro lado, usando o Teorema \ref{theo:derfxbCfxbxqDxq} podemos 
rescrever a Eq. (\ref{eq:proof:minhxhxxbxbe}) de forma aproximada como a Eq. (\ref{eq:proof:minhxhxxbxb1}),
\begin{equation}\label{eq:proof:minhxhxxbxb1}
2  h'(p) \left[\left(h(p)-a\right) + h'(p) \left(\hat{x} - p\right)\right] \approx
e'(\hat{x})=0.
\end{equation}
de modo que, se consideramos $h'(p)\neq 0$, a equação pode ser rescrita como:
%um ponto crítico $\hat{x}$ de $e(x)$ pode ser achado aproximadamente quando:
\begin{equation}\label{eq:proof:minhxhxxbxb2}
\hat{x} \approx p - \frac{\left(h(p)-a\right)}{h'(p)}.
\end{equation}
Isto quer dizer que a Eq. \ref{eq:proof:minhxhxxbxb2}, \textbf{serve só para
achar pontos críticos de} $e(x)$ \textbf{que não pertençam a} $h(x)$.
Sendo este obrigatoriamente um ponto onde $h(\hat{x})-a\approx 0$, 
equivalente a um mínimo absoluto.

Assim, quanto mais próximo a $\hat{x}$ seja  o valor $p$, 
a Eq. (\ref{eq:proof:minhxhxxbxb2}) fica mais próximo a uma igualdade. Por outro lado,
a equação nos indica que dado um ponto  $p$ qualquer,
$p - \left[ h'(p) \right]^{-1}\left(h(p)-a\right)$
é um ponto próximo de $p$  na direção de um mínimo de $e(x)$.
De modo que, um bom critério para procurar um mínimo é seguir a seguinte 
equação iterativa,
\begin{equation}\label{eq:proof:minhxhxxbxb3}
p_{k} \leftarrow p_{k-1} - \frac{ \left(h(p_{k-1})-a\right)}{h'(p_{k-1})},
\end{equation}
iniciando desde um $p_{0}$ qualquer, ate que $p_{k}$ seja muito próximo a $p_{k-1}$,
onde se declara que $\hat{x} \approx p_{k}$. 


porem deve-se ter cuidado com valores de $p_{k}$ que provoquem 
$0<|h'(p_{k})|<\lim\limits_{\varepsilon \rightarrow +0}\varepsilon$,
pois estes valores tendem a divergir. Por outro lado sim se acha um valor 
de $p_{k}$ com $h'(p_{k})= 0$, deve ser
corroborado se este ponto tratasse de um máximo ou mínimo, relativo ou absoluto, ou ponto de inflexão;
usando algum método, por exemplo estudando o comportamento 
de $e(x)$ ou analisando a matriz hessiana de $e(x)$ avaliada em $p_{k}$.

\end{myproofT}