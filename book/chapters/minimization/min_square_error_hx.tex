\chapterimage{chapter_head_2.pdf} % Chapter heading image

\chapter{Minimização do erro em funções: $\mathbb{R}$ $\rightarrow$ $\mathbb{R}$}

%%%%%%%%%%%%%%%%%%%%%%%%%%%%%%%%%%%%%%%%%%%%%%%%%%%%%%%%%%%%%%%%%%%%%%%%%%%%%%%%%%%%%%%
\section{\textcolor{blue}{Minimização de $||h(x)-a||^2$}} 

\begin{theorem}[Solução iterativa]\label{theo:minhxhx}
Dados,
um escalar $x \in \mathbb{R}$, 
um escalar $a \in \mathbb{R}$,  
uma função $h:\mathbb{R} \rightarrow \mathbb{R}$, e 
definida a Eq. (\ref{eq:minhxhx1}),
\begin{equation}\label{eq:minhxhx1}
e(x)=||h(x)-a||^2.
\end{equation}
Se desejamos ter o valor $\hat{x}$ que minimiza o escalar $e(\hat{x})$,
este valor pode ser achado usando iterativamente a Eq. (\ref{eq:minhxhx2}),
\begin{equation}\label{eq:minhxhx2}
x_{k+1} \leftarrow x_k+
\frac{ a-h(x_k)}{h'(x_k)}
\end{equation}
Onde  $h'(x)=\frac{\partial h(x)}{\partial x}$. A busca iterativa é considerada 
falida quando se atinge $h'(x_k) = 0$.
Assim, $\hat{x}$ pode ser achado iniciando a Eq. (\ref{eq:minhxhx2}) desde um $x_{0}$ qualquer, ate que $x_{k}$ seja muito próximo a $x_{k+1}$,
onde se declara que $\hat{x} \approx x_{k+1}$; porem deve ser corroborado
que esse ponto tratasse de um máximo ou mínimo usando algum método, por exemplo estudando o comportamento 
de $e(x)$ ou analisando  $\frac{\partial^2 e(x)}{\partial x^2}$ avaliada em $\hat{x}$.

\textcolor{red}{Esta forma de iterar tambem e conhecida como metodo de Newton.}

\FALTAPROVA
A demostração pode ser vista na Prova \ref{proof:theo:minhxhx}.
\end{theorem}

%%%%%%%%%%%%%%%%%%%%%%%%%%%%%%%%%%%%%%%%%%%%%%%%%%%%%%%%%%%%%%%%%%%%%%%%%%%%%%%%%%%%%%%
\section{\textcolor{blue}{Minimização de $||h(x)-a||^2+\alpha ||x-b||^2$}}

\begin{theorem}[Solução iterativa]\label{theo:minhxhxxbxb}
Dados,
um escalar $\alpha \in \mathbb{R} | \alpha > 0$, 
um escalar $x \in \mathbb{R}$, 
um escalar $a \in \mathbb{R}$,  
uma função $h:\mathbb{R} \rightarrow \mathbb{R}$, e 
definida a Eq. (\ref{eq:minhxhxxbxb1}),
\begin{equation}\label{eq:minhxhxxbxb1}
e(x)=||h(x)-a||^2+\alpha ||x-b||^2.
\end{equation}
Se desejamos ter o valor $\hat{x}$ que minimiza o escalar $e(\hat{x})$,
este valor pode ser achado usando iterativamente a Eq. (\ref{eq:minhxhxxbxb2}),
\begin{equation}\label{eq:minhxhxxbxb2}
x_{k+1} \leftarrow x_k+
\frac{ h'(x_k) \left[a-h(x_k)\right]-\alpha\left[ x_k-b\right]}{\left[h'(x_k)\right]^2+\alpha}
\end{equation}
Onde  $h'(x)=\frac{\partial h(x)}{\partial x}$.
Assim, $\hat{x}$ pode ser achado iniciando a Eq. (\ref{eq:minhxhxxbxb2}) desde um $x_{0}$ qualquer, ate que $x_{k}$ seja muito próximo a $x_{k+1}$,
onde se declara que $\hat{x} \approx x_{k+1}$; porem deve ser corroborado
que esse ponto tratasse de um máximo ou mínimo usando algum método, por exemplo estudando o comportamento 
de $e(x)$ ou analisando  $\frac{\partial^2 e(x)}{\partial x^2}$ avaliada em $\hat{x}$.

\FALTAPROVA
A demostração pode ser vista na Prova \ref{proof:theo:minhxhxxbxb}.
\end{theorem}

%%%%%%%%%%%%%%%%%%%%%%%%%%%%%%%%%%%%%%%%%%%%%%%%%%%%%%%%%%%%%%%%%%%%%%%%%%%%%%%%%%%%%%%
\section{\textcolor{blue}{Minimização de $||h(x)-a||^2+\alpha ||x-x_{old}||^2$}}

\begin{theorem}[Solução iterativa]\label{theo:minhxhxxoxo}
Dados,
um escalar $\alpha \in \mathbb{R} | \alpha > 0$, 
um escalar $x \in \mathbb{R}$, 
um escalar $a \in \mathbb{R}$,  
uma função $h:\mathbb{R} \rightarrow \mathbb{R}$, e 
definida a Eq. (\ref{eq:minhxhxxoxo1}),
\begin{equation}\label{eq:minhxhxxoxo1}
e(x)=||h(x)-a||^2+\alpha ||x-x_{old}||^2.
\end{equation}
Se desejamos ter o valor $\hat{x}$ que minimiza o escalar $e(\hat{x})$,
este valor pode ser achado usando iterativamente a Eq. (\ref{eq:minhxhxxoxo2}),
\begin{equation}\label{eq:minhxhxxoxo2}
x_{k+1} \leftarrow x_k+
\frac{ h'(x_k) \left[a-h(x_k)\right] }{\left[h'(x_k)\right]^2+\alpha}
\end{equation}
Onde  $h'(x)=\frac{\partial h(x)}{\partial x}$.
Assim, $\hat{x}$ pode ser achado iniciando a Eq. (\ref{eq:minhxhxxoxo2}) desde um $x_{0}$ qualquer, ate que $x_{k}$ seja muito próximo a $x_{k+1}$,
onde se declara que $\hat{x} \approx x_{k+1}$; porem deve ser corroborado
que esse ponto tratasse de um máximo ou mínimo usando algum método, por exemplo estudando o comportamento 
de $e(x)$ ou analisando  $\frac{\partial^2 e(x)}{\partial x^2}$ avaliada em $\hat{x}$.

\FALTAPROVA
A demostração pode ser vista na Prova \ref{proof:theo:minhxhxxoxo}.
\end{theorem}

