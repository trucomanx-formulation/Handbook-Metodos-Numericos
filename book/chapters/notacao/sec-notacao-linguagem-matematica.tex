
%%%%%%%%%%%%%%%%%%%%%%%%%%%%%%%%%%%%%%%%%%%%%%%%%%%%%%%%%%%%%%%%%%%%%%%%%%%%%%%%
%%%%%%%%%%%%%%%%%%%%%%%%%%%%%%%%%%%%%%%%%%%%%%%%%%%%%%%%%%%%%%%%%%%%%%%%%%%%%%%%
%%%%%%%%%%%%%%%%%%%%%%%%%%%%%%%%%%%%%%%%%%%%%%%%%%%%%%%%%%%%%%%%%%%%%%%%%%%%%%%%
\section{A linguagem matemática}

\begin{description}

\item[Definição:] \index{Definição} Uma definição é uma declaração na qual as 
pessoas interessadas chegam a um acordo \cite[pp. 37]{solow1987como}.
Se a definição não é aceita, é impossível a comunicação de temas relacionados à definição.
\begin{example}[Uso de termos em triângulos:]~\\
\begin{itemize}
\item \textbf{Definição:} Um triângulo é chamado isósceles se dois dos seus lados são iguais.
\item \textbf{Definição:} Um triângulo é chamado retângulo se tem um ângulo com $90^{\circ}$.
\item \textbf{Definição:} Um ângulo é chamado reto se tem  $90^{\circ}$.
\end{itemize}
\end{example}

\item[Axioma ou Postulado:] \index{Axioma} \index{Postulado} 
Uma proposição que é aceita sem uma demonstração formal \cite[pp. 47]{fossa2009introducao} \cite[pp. 41]{solow1987como}.
\begin{example}[Sobre os ângulos num triângulo isósceles:]~\\
\begin{itemize}
\item \textbf{Quinto postulado de Euclides:} 
Se uma reta cortar duas outras retas, de modo que a soma dos dois ângulos interiores, 
de um mesmo lado, seja menor que dois ângulos retos; então, as duas outras retas se cruzam, 
quando suficientemente prolongadas, do lado da primeira reta em que se acham os dois ângulos.
\end{itemize}
\end{example}

Antigamente havia uma distinção mais acentuada entre os termos axioma e postulado \cite[pp. 115]{de1863ensaio};
porém, o uso popular foi diminuindo a distância entre esses dois termos \cite[pp. 243]{mora2000dicionario}.
Por exemplo, nesse antigo uso dos termos axioma e postulado, poderíamos entender esses termos como,
\begin{itemize}
\item \textbf{Axioma:} ``Isto é assim porque é evidente, não preciso demonstrar''.
\item \textbf{Postulado:} ``Para futuras interações, 
gostaria de propor que partamos da base que essa afirmação é verdadeira,
 para poder chegar a conclusões decorrentes dessa afirmação em nosso dialogo''.
\end{itemize}


\item[Lema:] \index{Lema} Um lema é uma proposição preliminar demonstrada, 
a qual será usada para demonstrar um teorema \cite[pp. 49]{fossa2009introducao}\cite[pp. 41]{solow1987como},
a afirmação indicada pelo lema não tem muita importância matemática em si mesma, 
mas cumpre um papel importante para a demonstração de um teorema.
Falando de forma estrita, um lema também é um teorema; quer dizer, uma proposição demonstrada.
\begin{example}[Sobre os ângulos num triângulo isósceles:]~\\
\begin{itemize}
\item \textbf{Lema:} Num triângulo, se um ângulo é reto, os outros dois somam $90^{\circ}$.
\end{itemize}
\end{example}

\item[Proposição:] \index{Proposição} Uma proposição é um enunciado ou afirmação que 
se busca demonstrar \cite[pp. 41]{solow1987como}.
\begin{example}[Sobre os ângulos num triângulo isósceles:]~\\
\begin{itemize}
\item \textbf{Proposição:} Num triângulo isósceles, 
uma linha que parte do vértice com lados iguais, e divide ao lado oposto na metade,
forma um ângulo reto com esse lado.
\end{itemize}
\end{example}

\item[Teorema:] \index{Teorema} Um teorema é uma proposição 
demonstrada \cite[pp. 49]{fossa2009introducao} \cite[pp. 41]{solow1987como}.
\begin{example}[Sobre os ângulos num triângulo isósceles:]~\\
\begin{itemize}
\item \textbf{Teorema:} Num triângulo isósceles, 
uma linha que parte do vértice com lados iguais e divide ao lado oposto na metade,
forma um ângulo reto com este lado.
\item \textbf{Prova:}  Se chamamos $2\beta$ ao ângulo entre os lados iguais, 
e $\alpha$ a cada um dos ângulos restantes do triângulo isósceles, sabendo
que a soma de ângulos num triângulo é $180^{\circ}=2\beta+2\alpha$,
concluímos que $\beta+\alpha=90^{\circ}$.
Dado que cada um dos dois triângulos formados pela divisão criada pela linha tem ângulos $\alpha$ e $\beta$,
podemos concluir que o ângulo restante tem $90^{\circ}$, quer dizer é um ângulo reto.
\end{itemize}
\end{example}

\item[Corolário:] \index{Corolário} Um corolário é uma proposição 
fácil de demonstrar que se deduz a partir de um teorema \cite[pp. 49]{fossa2009introducao} \cite[pp. 41]{solow1987como}.
Falando de forma estrita, um corolário também é um teorema; isto é, uma proposição demonstrada.
\begin{example}[Sobre os ângulos num triângulo isósceles:]~\\
\begin{itemize}
\item \textbf{Corolário:} Num triângulo isósceles, 
uma linha que parte do vértice com lados iguais, e divide ao lado oposto na metade,
cria dois triângulos retângulos.
\end{itemize}
\end{example}

\end{description}
