
%%%%%%%%%%%%%%%%%%%%%%%%%%%%%%%%%%%%%%%%%%%%%%%%%%%%%%%%%%%%%%%%%%%%%%%%%%%%%%%%
%%%%%%%%%%%%%%%%%%%%%%%%%%%%%%%%%%%%%%%%%%%%%%%%%%%%%%%%%%%%%%%%%%%%%%%%%%%%%%%%
%%%%%%%%%%%%%%%%%%%%%%%%%%%%%%%%%%%%%%%%%%%%%%%%%%%%%%%%%%%%%%%%%%%%%%%%%%%%%%%%
\section{A linguagem matemática}

\begin{description}

\item[Definição:] \index{Definição} Uma definição é uma declaração na qual as pessoas interessadas chegam a um acordo \cite[pp. 37]{solow1987como}.
Se a definição não é aceita é impossível a comunicação de temas relacionados a definição.
\begin{example}[Uso do termo isósceles:]~\\
\begin{itemize}
\item \textbf{Definição:} Um triângulo é chamado isósceles se dois de seus lados são iguais.
\item \textbf{Definição:} Um triângulo é chamado retângulo se tem um ângulo com $90^{\circ}$.
\item \textbf{Definição:} Um angulo é chamado reto se tem  $90^{\circ}$.
\end{itemize}
\end{example}

\item[Axioma ou Postulado:] \index{Axioma} \index{Postulado} 
Uma proposição que é aceita sem uma demostração formal \cite[pp. 47]{fossa2009introducao} \cite[pp. 41]{solow1987como}.
\begin{example}[Sobre os ângulos num triângulo isósceles:]~\\
\begin{itemize}
\item \textbf{Quinto Postulado de Euclides:} Se uma reta cortar duas outras retas de modo que a soma dos dois ângulos interiores, de um mesmo lado, seja menor que dois ângulos retos, então as duas outras retas se cruzam, quando suficientemente prolongadas, do lado da primeira reta em que se acham os dois ângulos.
\end{itemize}
\end{example}

Antigamente existia uma distinção entre os termos axioma e postulado \cite[pp. 115]{de1863ensaio},
porém o uso foi acurtando a distancia entre estes dois termos \cite[pp. 243]{mora2000dicionario}.
Por exemplo, nessa antiga concepção, poderíamos entender estes termos como,
\begin{itemize}
\item \textbf{Axioma:} ``Isto é assim porque é evidente, não preciso demostrar''.
\item \textbf{Postulado:} ``Para futuras interações gostaria que partamos da base que esta afirmação é verdadeira,
 para poder chegar a conclusões sobre este assunto em nosso dialogo''.
\end{itemize}


\item[Lema:] \index{Lema} Um lema é uma proposição preliminar demostrada, 
a qual será usada para demostrar um teorema \cite[pp. 49]{fossa2009introducao}\cite[pp. 41]{solow1987como},
a afirmação indicada pelo lema não tem muita importância matemática em sim mesma, mas cumpre um papel importante para a demostração de um teorema.
Falando de forma estrita um corolário também é um teorema; é dizer, uma proposição demostrada.
\begin{example}[Sobre os ângulos num triângulo isósceles:]~\\
\begin{itemize}
\item \textbf{Lema:} Num triangulo, se um angulo é reto os outros dois somam $90^{\circ}$.
\end{itemize}
\end{example}

\item[Proposição:] \index{Proposição} Uma proposição é um enunciado de inteires que se busca demostrar \cite[pp. 41]{solow1987como}.
\begin{example}[Sobre os ângulos num triângulo isósceles:]~\\
\begin{itemize}
\item \textbf{Proposição:} Num triangulo isósceles, uma linha que parte do vértice com lados iguais e divide ao lado oposto na metade,
forma um ângulo reto com este lado.
\end{itemize}
\end{example}

\item[Teorema:] \index{Teorema} Um teorema é uma proposição demostrada \cite[pp. 49]{fossa2009introducao} \cite[pp. 41]{solow1987como}.
\begin{example}[Sobre os ângulos num triângulo isósceles:]~\\
\begin{itemize}
\item \textbf{Teorema:} Num triangulo isósceles, uma linha que parte do vértice com lados iguais e divide ao lado oposto na metade,
forma um ângulo reto com este lado.
\item \textbf{Prova:}  Se chamamos $2\beta$ ao ângulo entre os lados iguais, 
e $\alpha$ a cada um dos ângulos restantes do triângulo isósceles. Sabendo
que a soma de ângulos num triângulo é $180^{\circ}=2\beta+2\alpha$,
concluímos que $\beta+\alpha=90^{\circ}$.
Dado que cada um dos dois triângulos formados pela divisão criada pela linha, tem ângulos $\alpha$ e $\beta$,
podemos concluir que o ângulo restante tem $90^{\circ}$, é dizer é um ângulo reto.
\end{itemize}
\end{example}

\item[Corolário:] \index{Corolário} Um corolário é uma proposição 
fácil de demostrar que se deduz a partir de um teorema \cite[pp. 49]{fossa2009introducao} \cite[pp. 41]{solow1987como}.
Falando de forma estrita um corolário também é um teorema; é dizer, uma proposição demostrada.
\begin{example}[Sobre os ângulos num triângulo isósceles:]~\\
\begin{itemize}
\item \textbf{Corolário:} Num triangulo isósceles, 
uma linha que parte do vértice com lados iguais e divide ao lado oposto na metade,
cria dois triângulos retângulos.
\end{itemize}
\end{example}

\end{description}
