%%%%%%%%%%%%%%%%%%%%%%%%%%%%%%%%%%%%%%%%%%%%%%%%%%%%%%%%%%%%%%%%%%%%%%%%%%%%%%%%
%%%%%%%%%%%%%%%%%%%%%%%%%%%%%%%%%%%%%%%%%%%%%%%%%%%%%%%%%%%%%%%%%%%%%%%%%%%%%%%%
%%%%%%%%%%%%%%%%%%%%%%%%%%%%%%%%%%%%%%%%%%%%%%%%%%%%%%%%%%%%%%%%%%%%%%%%%%%%%%%%
\section{Notação usada para matrizes, vetores e funções}
\begin{notation}[Método em que os valores escalares, vetores ou matrizes são definidos:]~\\
\begin{tabular}{p{.2\textwidth} | p{.4\textwidth} | p{.3\textwidth} }
\hline	
\textbf{Tipo} & \textbf{Descrição} & \textbf{Formatação} \\ \hline
$\MATRIX{A}$, $\MATRIX{B}$, ..., $\MATRIX{X}$, $\MATRIX{Y}$, $\MATRIX{Z}$& Matriz. & Maiúsculo e negrito \\
$\VECTOR{a}$, $\VECTOR{b}$, ..., $\VECTOR{x}$, $\VECTOR{y}$, $\VECTOR{z}$ & Vetor ou conjunto. & Minúsculo e negrito \\
%$A$, $B$, ..., $X$, $Y$, $Z$ & ------- & Maiúsculo \\
$a$, $b$, ..., $x$, $y$, $z$ & Escalar variável. & Minúsculo \\
$A$, $B$, ..., $X$, $Y$, $Z$ & Escalar constante. & Maiúsculo \\
$\alpha$, $\beta$, ..., $\chi$, $\psi$, $\omega$ & Escalar variável ou constante. & Letras gregas  \\ \hline
\end{tabular}
\end{notation}


\begin{notation}[Funções notáveis usadas neste livro:]~\\
\begin{tabular}{p{.11\textwidth} |  p{.8\textwidth} }
\hline	
\textbf{Função} & \textbf{Descrição} \\ \hline
%$card(\VECTOR{a})$ & Número de elementos, cardinalidade, do vetor ou conjunto $\VECTOR{a}$. \\
$\funcnumel(\MATRIX{A})$ & Número de elementos da matriz $\MATRIX{A}$. \\
\hline
%$dim(\VECTOR{a},1)$ & Primeira dimensão do vetor $\VECTOR{a}$, número de linhas. \\
%$dim(\VECTOR{a},2)$ & Segunda dimensão do vetor $\VECTOR{a}$, número de colunas. \\
$dim(\MATRIX{A},1)$ & Primeira dimensão da matriz $\MATRIX{A}$, número de linhas. \\
$dim(\MATRIX{A},2)$ & Segunda dimensão da matriz $\MATRIX{A}$, número de colunas. \\
$dim(\MATRIX{A},N)$ & Dimensão $N$ da matriz $\MATRIX{A}$, se não tiver retorna $1$. \\
\hline
$\funcinv(\MATRIX{A})$ & Inversa da matriz $\MATRIX{A}$, isto é equivalente a escrever $\MATRIX{A}^{-1}$. \\
$\MATRIX{A}^{-1}$ & Inversa da matriz $\MATRIX{A}$, isto é equivalente a escrever $\funcinv(\MATRIX{A})$. \\
\hline
%$\functrans(\VECTOR{a})$ & Transposta do vetor $\VECTOR{a}$, é equivalente a escrever $\VECTOR{a}^{\transpose}$. \\
%$\VECTOR{a}^{\transpose}$ & Transposta do vetor $\VECTOR{a}$, é equivalente a escrever $trans(\VECTOR{a})$. \\
$\functrans(\MATRIX{A})$ & Transposta da matriz $\MATRIX{A}$, isto é equivalente a escrever $\MATRIX{A}^{\transpose}$. \\
$\MATRIX{A}^{\transpose}$ & Transposta da matriz $\MATRIX{A}$, isto é equivalente a escrever $trans(\MATRIX{A})$. \\
\hline
$\funcdiag(\VECTOR{a})$ & Matriz diagonal construída a partir de pôr na diagonal de uma matriz nula os elementos do vetor $\VECTOR{a}$. \\
\hline
$\lambda_{\MATRIX{A}}$ & Matriz diagonal em que os elementos da diagonal se correspondem com os autovalores da matriz $\MATRIX{A}$. \\
\hline
$det(\MATRIX{A})$ & Determinante da matriz $\MATRIX{A}$. \\
\hline
\end{tabular}
\end{notation}

\newpage
\begin{notation}[Método em que os elementos dos vetores são definidos:]~\\
\begin{tabular}{p{.05\textwidth} | p{.6\textwidth} | p{.25\textwidth}}
\hline	
\textbf{Tipo} & \textbf{Descrição} & \textbf{Formatação} \\ \hline
$a_{i}$ & Elemento $i$-ésimo do vetor ou conjunto  $\VECTOR{a}$.& Minúsculo \\
%$\VECTOR{a}_{(i)}$ & Elemento $i$-ésimo do vetor $\VECTOR{a}$ & Minúsculo e negrito \\ \hline
\hline
\end{tabular}
\end{notation}


\begin{notation}[Método em que os elementos das matrizes bidimensionais são definidos:]~\\
\begin{tabular}{p{.05\textwidth} | p{.6\textwidth} | p{.25\textwidth}}
\hline	
\textbf{Tipo} & \textbf{Descrição} & \textbf{Formatação} \\ \hline
$a_{ij}$ & Escalar formado pelo elemento na linha $i$, coluna $j$ da matriz $\MATRIX{A}$. & Minúsculo \\ \hline
%$\MATRIX{A}_{(i,j)}$& Elemento na linha $i$, coluna $j$ da matriz $\MATRIX{A}$ & Maiúsculo e negrito \\ \hline
%$\VECTOR{a}_{i}$ & Linha ou coluna $i$-ésima da matriz $\MATRIX{A}$ (ambíguo) & Minúsculo e negrito \\
$a_{i:}$ & Vetor linha formado pela linha $i$-ésima da matriz $\MATRIX{A}$.  & Minúsculo \\
%$\MATRIX{A}_{(i,:)}$& Vetor formado pela linha $i$-ésima da matriz $\MATRIX{A}$ & Maiúsculo e negrito \\
$a_{:i}$ & Vetor coluna formado pela coluna $i$-ésima da matriz $\MATRIX{A}$.  & Minúsculo \\
%$\MATRIX{A}_{(:,i)}$& Vetor formado pela coluna $i$-ésima da matriz $\MATRIX{A}$ & Maiúsculo e negrito \\ \hline
\hline
\end{tabular}
\end{notation}


\begin{notation}[Funções em tempo continuo e discreto:]~\\
\begin{tabular}{p{.05\textwidth} | p{.85\textwidth} }
\hline	
\textbf{Tipo}            & \textbf{Descrição} \\ \hline
$x(t)$          & Função escalar de domínio continuo, representado pela variável $t$. \\ \hline
$x[n]$          & Função escalar de domínio discreto, em que a variável $n$ representa a $n$-ésima amostra. \\ \hline
$\VECTOR{x}(t)$ & Função matricial de domínio continuo, representado pela variável $t$.  \\ \hline
$\VECTOR{x}[n]$ & Função matricial de domínio discreto, em que a variável $n$ representa a $n$-ésima amostra. \\
\hline
\end{tabular}
\end{notation}

