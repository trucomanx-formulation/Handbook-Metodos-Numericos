%%%%%%%%%%%%%%%%%%%%%%%%%%%%%%%%%%%%%%%%%%%%%%%%%%%%%%%%%%%%%%%%%%%%%%%%%%%%%%%%
%%%%%%%%%%%%%%%%%%%%%%%%%%%%%%%%%%%%%%%%%%%%%%%%%%%%%%%%%%%%%%%%%%%%%%%%%%%%%%%%
%%%%%%%%%%%%%%%%%%%%%%%%%%%%%%%%%%%%%%%%%%%%%%%%%%%%%%%%%%%%%%%%%%%%%%%%%%%%%%%%
\section{Problema \wellposed~ vs. \illposed }

%%%%%%%%%%%%%%%%%%%%%%%%%%%%%%%%%%%%%%%%%%%%%%%%%%%%%%%%%%%%%%%%%%%%%%%%%%%%%%%%
\subsection{Problema \wellposed}
\index{Problema!\Wellposed}
\index{Well-posed}
\begin{definition}[Problema \wellposed:]
\label{def:bem-posto:1}
Conhecido um modelo matemático ou sistema que desejamos resolver para obter uma solução;
indicamos que este é um problema \wellposed~ (do inglês ``well-posed'') se cumpre 3 condições \cite[pp. 16]{gockenbach2016linear},
\begin{itemize}
\item \textbf{Existência:} Existe uma solução que verifica o sistema.
\item \textbf{Unicidade:} A solução é única.
\item \textbf{Estabilidade:} A solução depende continuamente da saída;
é dizer, pequenas variações na resposta provem de pequenas variações nos dados usados para calcular a resposta.
\end{itemize}
\end{definition}

\begin{example}[Problema \wellposed:]
Conhecido o problema de obter o vetor $\VECTOR{x} \in \mathbb{R}^{N}$,
que cumpre o sistema,
\begin{equation}
\MATRIX{A}\VECTOR{x}=\VECTOR{y},
\end{equation}
 tendo como dados o vetor $\VECTOR{y} \in \mathbb{R}^{N}$ e 
a matriz $\MATRIX{A}: \mathbb{R}^{N} \rightarrow \mathbb{R}^{N}$ que carateriza ao sistema.
Falamos que o problema está \wellposed~ se a matriz $\MATRIX{A}$ tem inversa e esta é limitada \cite[pp. 18]{gockenbach2016linear}. 
\end{example}

%%%%%%%%%%%%%%%%%%%%%%%%%%%%%%%%%%%%%%%%%%%%%%%%%%%%%%%%%%%%%%%%%%%%%%%%%%%%%%%%
\subsection{Problema \illposed}
\index{Ill-posed}
\index{Problema!\Illposed}
\begin{definition}[Problema \illposed:]
\label{def:mal-posto:1}
Um problema é chamado \illposed~ (do inglês ``ill-posed'') se não cumpre um o mais das condições que definem a um problema 
\wellposed~ \cite[pp. 18]{gockenbach2016linear}.

Os problemas inversos geralmente são relacionados a um problema \illposed,
pois pelo geral não cumprem uma ou mais das condições para ser catalogados como \wellposed. 
\end{definition}

\begin{example}[Problema \illposed~ sem solução:]
Conhecido o problema de obter o vetor $\VECTOR{x}\in \mathbb{R}^N$,
que cumpre o sistema,
\begin{equation}
\MATRIX{A}\VECTOR{x}=\VECTOR{y},
\qquad
\MATRIX{A}=
\begin{bmatrix}
1 & 1\\
2 & 1\\
0 & 1
\end{bmatrix}
\qquad
\VECTOR{y}=
\begin{bmatrix}
2\\
3\\
2
\end{bmatrix}.
\end{equation}
Catalogamos este problema como \illposed~ pois não existe um vetor $\VECTOR{x}$
que verifique o sistema $\MATRIX{A}\VECTOR{x}=\VECTOR{y}$.
\end{example}


\begin{example}[Problema \illposed~ sem solução única:]
Conhecido o problema de obter o vetor $\VECTOR{x}\in \mathbb{R}^N$,
que cumpre o sistema,
\begin{equation}
\MATRIX{A}\VECTOR{x}=\VECTOR{y},
\qquad
\MATRIX{A}=
\begin{bmatrix}
1 & 1\\
1 & 1
\end{bmatrix}
\qquad
\VECTOR{y}=
\begin{bmatrix}
2\\
2
\end{bmatrix}.
\end{equation}
Catalogamos este problema como \illposed~  pois existem múltiplas soluções $\VECTOR{x}$
que cumprem o sistema,
\begin{equation}
\VECTOR{x}=
\begin{bmatrix}
2\\
0
\end{bmatrix}
+\alpha
\begin{bmatrix}
-1\\
1
\end{bmatrix},
\qquad
\forall \alpha \in \mathbb{R}.
\end{equation}

\end{example}
