\chapterimage{chapter_head_2.pdf} % Chapter heading image

\chapter{Derivada de funções com variável vetorial}

%%%%%%%%%%%%%%%%%%%%%%%%%%%%%%%%%%%%%%%%%%%%%%%%%%%%%%%%%%%%%%%%%%%%%%%%%%%%%%%%%%%%%%%
%%%%%%%%%%%%%%%%%%%%%%%%%%%%%%%%%%%%%%%%%%%%%%%%%%%%%%%%%%%%%%%%%%%%%%%%%%%%%%%%%%%%%%%
\section{Derivada de $e(\mathbf{x})$ e $\mathbf{f}(\mathbf{x})$}

\begin{definition}\label{def:deltahor}
Se 
$\mathbf{x}\in \mathbb{R}^N$ é um vetor linha com elementos $x_n\in \mathbb{R}$ de modo que
$n\in \mathbb{N}$, $1 \leq n \leq N$, 
a função $e(\mathbf{x}): \mathbb{R}^N \rightarrow \mathbb{R}$ é um escalar, e
a função $\mathbf{f}(\mathbf{x}): \mathbb{R}^N \rightarrow \mathbb{R}^M$ é um vetor coluna, 
então definimos que:

\begin{equation}
\frac{\partial e(\mathbf{x}) }{\partial \mathbf{x}^{\transpose}}= 
\left[
\begin{matrix}
\frac{\partial e(\mathbf{x}) }{\partial x_{1}}&
\frac{\partial e(\mathbf{x}) }{\partial x_{2}}&
\hdots&
\frac{\partial e(\mathbf{x}) }{\partial x_{n}}&
\hdots&
\frac{\partial e(\mathbf{x}) }{\partial x_{N}}
\end{matrix}
\right]= {\bigcup\limits_{n=1}^{\rightarrow}}^{N}{\frac{\partial e(\mathbf{x}) }{\partial x_{n}}} 
\end{equation}

\begin{equation}
\frac{\partial \mathbf{f}(\mathbf{x}) }{\partial \mathbf{x}^{\transpose}}= 
\left[
\begin{matrix}
\frac{\partial \mathbf{f}(\mathbf{x}) }{\partial x_{1}}&
\frac{\partial \mathbf{f}(\mathbf{x}) }{\partial x_{2}}&
\hdots&
\frac{\partial \mathbf{f}(\mathbf{x}) }{\partial x_{n}}&
\hdots&
\frac{\partial \mathbf{f}(\mathbf{x}) }{\partial x_{N}}
\end{matrix}
\right]= {\bigcup\limits_{n=1}^{\rightarrow}}^{N}{\frac{\partial \mathbf{f}(\mathbf{x}) }{\partial x_{n}}} 
\end{equation}

\begin{equation}
\frac{\partial \mathbf{G}(\mathbf{x}) }{\partial \mathbf{x}^{\transpose}}= 
\left[
\begin{matrix}
\frac{\partial \mathbf{G}(\mathbf{x}) }{\partial x_{1}}&
\frac{\partial \mathbf{G}(\mathbf{x}) }{\partial x_{2}}&
\hdots&
\frac{\partial \mathbf{G}(\mathbf{x}) }{\partial x_{n}}&
\hdots&
\frac{\partial \mathbf{G}(\mathbf{x}) }{\partial x_{N}}
\end{matrix}
\right]= {\bigcup\limits_{n=1}^{\rightarrow}}^{N}{\frac{\partial \mathbf{G}(\mathbf{x}) }{\partial x_{n}}} 
\end{equation}
\end{definition}

\begin{definition}\label{def:deltaver}
Se 
$\mathbf{x}\in \mathbb{R}^N$ é um vetor coluna com elementos $x_n\in \mathbb{R}$ de modo que
$n\in \mathbb{N}$, $1 \leq n \leq N$, 
a função $e(\mathbf{x}): \mathbb{R}^N \rightarrow \mathbb{R}$ é um escalar, e
a função $\mathbf{f}(\mathbf{x}): \mathbb{R}^N \rightarrow \mathbb{R}^M$ é um vetor coluna, 
então definimos que:
\begin{equation}
\frac{\partial e(\mathbf{x}) }{\partial \mathbf{x}}= 
\left[
\begin{matrix}
\frac{\partial e(\mathbf{x}) }{\partial x_{1}} \\
\frac{\partial e(\mathbf{x}) }{\partial x_{2}} \\
\vdots \\
\frac{\partial e(\mathbf{x}) }{\partial x_{n}} \\
\vdots \\
\frac{\partial e(\mathbf{x}) }{\partial x_{N}} 
\end{matrix}
\right] = \functrans \left( \frac{\partial e(\mathbf{x}) }{\partial \mathbf{x}^{\transpose}} \right) =
\funcvec \left( \frac{\partial e(\mathbf{x}) }{\partial \mathbf{x}^{\transpose}} \right) =
{\bigcup\limits_{n=1}^{\downarrow}}^{N}{\frac{\partial e(\mathbf{x}) }{\partial x_{n}}} 
\end{equation}

\begin{equation}
\frac{\partial \mathbf{f}(\mathbf{x}) }{\partial \mathbf{x}}= 
\left[
\begin{matrix}
\frac{\partial \mathbf{f}(\mathbf{x}) }{\partial x_{1}} \\
\frac{\partial \mathbf{f}(\mathbf{x}) }{\partial x_{2}} \\
\vdots \\
\frac{\partial \mathbf{f}(\mathbf{x}) }{\partial x_{n}} \\
\vdots \\
\frac{\partial \mathbf{f}(\mathbf{x}) }{\partial x_{N}}
\end{matrix}
\right] =  \funcvec \left( \frac{\partial \mathbf{f}(\mathbf{x}) }{\partial \mathbf{x}^{\transpose}} \right) =
{\bigcup\limits_{n=1}^{\downarrow}}^{N}{\frac{\partial \mathbf{f}(\mathbf{x}) }{\partial x_{n}}}
\end{equation}

\begin{equation}
\frac{\partial \mathbf{G}(\mathbf{x}) }{\partial \mathbf{x}}= 
\left[
\begin{matrix}
\frac{\partial \mathbf{G}(\mathbf{x}) }{\partial x_{1}} \\
\frac{\partial \mathbf{G}(\mathbf{x}) }{\partial x_{2}} \\
\vdots \\
\frac{\partial \mathbf{G}(\mathbf{x}) }{\partial x_{n}} \\
\vdots \\
\frac{\partial \mathbf{G}(\mathbf{x}) }{\partial x_{N}}
\end{matrix}
\right] = {\bigcup\limits_{n=1}^{\downarrow}}^{N}{\frac{\partial \mathbf{G}(\mathbf{x}) }{\partial x_{n}}}
\end{equation}
\end{definition}
%%%%%%%%%%%%%%%%%%%%%%%%%%%%%%%%%%%%%%%%%%%%%%%%%%%%%%%%%%%%%%%%%%%%%%%%%%%%%%%%%%%%%%%
%%%%%%%%%%%%%%%%%%%%%%%%%%%%%%%%%%%%%%%%%%%%%%%%%%%%%%%%%%%%%%%%%%%%%%%%%%%%%%%%%%%%%%%
\section{Derivada de $\mathbf{A}\mathbf{x}$}

\begin{theorem}\label{theo:derAx}
Se 
$\mathbf{x}\in \mathbb{R}^N$ é um vetor coluna com elementos $x_n$ de modo que
$n\in \mathbb{N}$, $1 \leq n \leq N$, e 
$\mathbf{A} \in \mathbb{R}^{M\times N}$ é uma matriz com elementos $a_{mn}$ de modo que
$m\in \mathbb{N}$, $1 \leq m \leq M$, então se cumpre que:
\begin{equation}
\frac{\partial \mathbf{A}\mathbf{x}}{\partial x_n}=a_{:n}
\end{equation}
A demostração pode ser vista na Prova \ref{proof:theo:derAx}.
\end{theorem}

\begin{corollaryT}[Derivada de $\mathbf{A}\mathbf{x}$ em relação ao vector $\mathbf{x}^{\transpose}$]\label{coro:derAx1}
Aplicando a Definição \ref{def:deltahor} junto ao Teorema \ref{theo:derAx}, é
fácil deduzir que:
\begin{equation}
\frac{\partial \mathbf{A}\mathbf{x}}{\partial \mathbf{x}^{\transpose}}=
\left[
\begin{matrix}
 a_{:1} &  a_{:2} &  \cdots &  a_{:N}
\end{matrix}
\right]=
\mathbf{A}
\end{equation}
\end{corollaryT}

\begin{corollaryT}[Derivada de $\mathbf{A}\mathbf{x}$ em relação ao vector $\mathbf{x}$]\label{coro:derAx2}
Aplicando a Definição \ref{def:deltaver} junto ao Teorema \ref{theo:derAx}, é
fácil deduzir que:
\begin{equation}
\frac{\partial \mathbf{A}\mathbf{x}}{\partial \mathbf{x}}=
\left[
\begin{matrix}
 a_{:1} \\  a_{:2} \\  \vdots \\  a_{:N}
\end{matrix}
\right]=\funcvec(\mathbf{A})
\end{equation}
\end{corollaryT}

\begin{corollaryT}[Derivada de $\mathbf{a}^{\transpose}\mathbf{x}$ em relação ao vector $\mathbf{x}^{\transpose}$]\label{coro:derAx3}
Aplicando a Definição \ref{def:deltahor} junto ao Teorema \ref{theo:derAx} e sabendo que $\mathbf{a}^{\transpose}$
é um vetor linha, é
fácil deduzir que:
\begin{equation}
\frac{\partial \mathbf{a}^{\transpose}\mathbf{x}}{\partial \mathbf{x}^{\transpose}}=\mathbf{a}^{\transpose}
\end{equation}
\end{corollaryT}

\begin{corollaryT}[Derivada de $\mathbf{a}^{\transpose}\mathbf{x}$ em relação ao vector $\mathbf{x}$]\label{coro:derAx4}
Aplicando a Definição \ref{def:deltaver} junto ao Teorema \ref{theo:derAx} e sabendo que $\mathbf{a}^{\transpose}$
é um vetor linha, é
fácil deduzir que:
\begin{equation}
\frac{\partial \mathbf{a}^{\transpose}\mathbf{x}}{\partial \mathbf{x}}=\mathbf{a}
\end{equation}
\end{corollaryT}

%%%%%%%%%%%%%%%%%%%%%%%%%%%%%%%%%%%%%%%%%%%%%%%%%%%%%%%%%%%%%%%%%%%%%%%%%%%%%%%%%%%%%%%
%%%%%%%%%%%%%%%%%%%%%%%%%%%%%%%%%%%%%%%%%%%%%%%%%%%%%%%%%%%%%%%%%%%%%%%%%%%%%%%%%%%%%%%
\section{Derivada de $||\mathbf{A}\mathbf{x}||^2$ 
}

\begin{theorem}\label{theo:derxAtAx}
Se 
$\mathbf{x}\in \mathbb{R}^N$ é um vetor coluna com elementos $x_n$ de modo que
$n\in \mathbb{N}$, $1 \leq n \leq N$, e 
$\mathbf{A} \in \mathbb{R}^{M\times N}$ é uma matriz com elementos $a_{mn}$ de modo que
$m\in \mathbb{N}$, $1 \leq m \leq M$, então se cumpre que:
\begin{equation}
\begin{matrix}
\frac{\partial ||\mathbf{A}\mathbf{x}||^2 }{\partial x_n}&=&
\frac{\partial \left(\mathbf{A}\mathbf{x}\right)^{\transpose}\left(\mathbf{A}\mathbf{x}\right)}{\partial x_n}&=&
2\left(\mathbf{A}\mathbf{x}\right)^{\transpose}a_{:n}\\
~&~&~&=& 2\left(a_{:n}\right)^{\transpose}\mathbf{A}\mathbf{x}
\end{matrix}
\end{equation}
A demostração pode ser vista na Prova \ref{proof:theo:derxAtAx}.
\end{theorem}

\begin{corollaryT}[Derivada de $||\mathbf{A}\mathbf{x}||^2$ em relação ao vector $\mathbf{x}^{\transpose}$]\label{coro:derxAtAx1}
Aplicando a Definição \ref{def:deltahor} junto ao Teorema \ref{theo:derxAtAx}, é
fácil deduzir que:
\begin{equation}
\frac{\partial ||\mathbf{A}\mathbf{x}||^2 }{\partial \mathbf{x}^{\transpose}}=
\frac{\partial \left(\mathbf{A}\mathbf{x}\right)^{\transpose}\left(\mathbf{A}\mathbf{x}\right)}{\partial \mathbf{x}^{\transpose}}=
2\left(\mathbf{A}^{\transpose}\mathbf{A}\mathbf{x}\right)^{\transpose}
\end{equation}
\end{corollaryT}

\begin{corollaryT}[Derivada de $||\mathbf{A}\mathbf{x}||^2$ em relação ao vector $\mathbf{x}$]\label{coro:derxAtAx2}
Aplicando a Definição \ref{def:deltaver} junto ao Teorema \ref{theo:derxAtAx}, é
fácil deduzir que:
\begin{equation}
\frac{\partial ||\mathbf{A}\mathbf{x}||^2 }{\partial \mathbf{x}}=
\frac{\partial \left(\mathbf{A}\mathbf{x}\right)^{\transpose}\left(\mathbf{A}\mathbf{x}\right)}{\partial \mathbf{x}}=
2 \mathbf{A}^{\transpose}\mathbf{A}\mathbf{x}
\end{equation}
\end{corollaryT}

%%%%%%%%%%%%%%%%%%%%%%%%%%%%%%%%%%%%%%%%%%%%%%%%%%%%%%%%%%%%%%%%%%%%%%%%%%%%%%%%%%%%%%%
%%%%%%%%%%%%%%%%%%%%%%%%%%%%%%%%%%%%%%%%%%%%%%%%%%%%%%%%%%%%%%%%%%%%%%%%%%%%%%%%%%%%%%%
\section{Derivada de $||\mathbf{A}\mathbf{x}-\mathbf{b}||_{\mathbf{C}}^2$ 
}

\begin{theorem}\label{theo:derAxbAxb}
Se 
$\mathbf{x}\in \mathbb{R}^N$ é um vetor coluna com elementos $x_n$ de modo que
$n\in \mathbb{N}$, $1 \leq n \leq N$, 
$\mathbf{b}\in \mathbb{R}^M$ é um vetor coluna com elementos $b_m$ de modo que
$m\in \mathbb{N}$, $1 \leq m \leq M$,  
$\mathbf{A} \in \mathbb{R}^{M\times N}$ é uma matriz com elementos $a_{mn}$, e
$\mathbf{C} \in \mathbb{R}^{M\times M}$ é uma matriz diagonal, 
então se cumpre que:
\begin{equation}
\begin{matrix}
\frac{\partial ||\mathbf{A}\mathbf{x}-\mathbf{b}||_{\mathbf{C}}^2}{\partial x_n}&=&
\frac{\partial \left(\mathbf{A}\mathbf{x}-\mathbf{b}\right)^{\transpose}\mathbf{C}\left(\mathbf{A}\mathbf{x}-\mathbf{b}\right)}{\partial x_n}&=&
2\left(\mathbf{A}\mathbf{x}-\mathbf{b}\right)^{\transpose}\mathbf{C}a_{:n}\\
~&~&~&=& 2\left(a_{:n}\right)^{\transpose}\mathbf{C}\left(\mathbf{A}\mathbf{x}-  \mathbf{b}\right)
\end{matrix}
\end{equation}
A demostração pode ser vista na Prova \ref{proof:theo:derAxbAxb}.
\end{theorem}

\begin{corollaryT}[Derivada de $||\mathbf{A}\mathbf{x}-\mathbf{b}||_{\mathbf{C}}^2$ em relação ao vector $\mathbf{x}^{\transpose}$]\label{coro:derAxbAxb1}
Aplicando a Definição \ref{def:deltahor} junto ao Teorema \ref{theo:derAxbAxb}, é
fácil deduzir que:
\begin{equation}
\frac{\partial ||\mathbf{A}\mathbf{x}-\mathbf{b}||^2}{\partial \mathbf{x}^{\transpose}}=
2\left(\mathbf{A}\mathbf{x}- \mathbf{b} \right)^{\transpose}\mathbf{C}\mathbf{A}
\end{equation}
\end{corollaryT}

\begin{corollaryT}[Derivada de $||\mathbf{A}\mathbf{x}-\mathbf{b}||_{\mathbf{C}}^2$ em relação ao vector $\mathbf{x}$]\label{coro:derAxbAxb2}
Aplicando a Definição \ref{def:deltaver} junto ao Teorema \ref{theo:derAxbAxb}, é
fácil deduzir que:
\begin{equation}
\frac{\partial ||\mathbf{A}\mathbf{x}-\mathbf{b}||^2}{\partial \mathbf{x} }=
2 \mathbf{A}^{\transpose}\mathbf{C}\left(\mathbf{A}\mathbf{x}-\mathbf{b}\right)
\end{equation}
\end{corollaryT}

%%%%%%%%%%%%%%%%%%%%%%%%%%%%%%%%%%%%%%%%%%%%%%%%%%%%%%%%%%%%%%%%%%%%%%%%%%%%%%%%%%%%%%%
%%%%%%%%%%%%%%%%%%%%%%%%%%%%%%%%%%%%%%%%%%%%%%%%%%%%%%%%%%%%%%%%%%%%%%%%%%%%%%%%%%%%%%%
\section{Derivada de $||\mathbf{f}(\mathbf{x})-\mathbf{b}||_{\mathbf{C}}^2$  
(Aproximação e exato)
}

\begin{theorem}[Valor exato]\label{theo:derfxbCfxb0}
Se 
$\mathbf{x}\in \mathbb{R}^N$ é um vetor coluna, 
$\mathbf{b}\in \mathbb{R}^M$ é um vetor coluna,  
$\mathbf{f}: \mathbb{R}^{N}\rightarrow \mathbb{R}^{M}$ é uma função de valor vectorial, e
$\mathbf{C} \in \mathbb{R}^{M\times M}$ é uma matriz diagonal, 
então se cumpre que:
\begin{equation}
\frac{\partial ||\mathbf{f}(\mathbf{x})-\mathbf{b}||_{\mathbf{C}}^2}{\partial \mathbf{x}} =
2 \mathbf{J}(\mathbf{x})^{\transpose}\mathbf{C}\left[\mathbf{f}(\mathbf{x})-\mathbf{b}\right],
\end{equation}
onde $\mathbf{J}(\mathbf{x})$ é a matriz Jacobiana de $\mathbf{f}(\mathbf{x})$.

A demostração pode ser vista na Prova \ref{proof:theo:derfxbCfxb0}.
\end{theorem}
\index{Jacobian}
\index{Serie de Taylor}

\begin{theorem}[Valor aproximado]\label{theo:derfxbCfxb}
Se 
$\mathbf{x}\in \mathbb{R}^N$ é um vetor coluna, 
$\mathbf{b}\in \mathbb{R}^M$ é um vetor coluna,  
$\mathbf{f}: \mathbb{R}^{N}\rightarrow \mathbb{R}^{M}$ é uma função de valor vectorial, e
$\mathbf{C} \in \mathbb{R}^{M\times M}$ é uma matriz diagonal, 
então se cumpre que:
\begin{equation}
\frac{\partial ||\mathbf{f}(\mathbf{x})-\mathbf{b}||_{\mathbf{C}}^2}{\partial \mathbf{x}} \approx
2 \mathbf{J}(\mathbf{p})^{\transpose}\mathbf{C}\left[\mathbf{J}(\mathbf{p})\left(\mathbf{x} - \mathbf{p}\right)-\left(\mathbf{b}-\mathbf{f}(\mathbf{p})\right)\right]
\end{equation}

Onde é considerada a aproximação
$\mathbf{f}(\mathbf{x})\approx \mathbf{f}(\mathbf{p})+\mathbf{J}(\mathbf{p})\left(\mathbf{x}-\mathbf{p}\right)$,
usando a serie de Taylor para funções multivariáveis. Sendo $\mathbf{p}$ um ponto fixo no domínio de $\mathbf{f}(\mathbf{x})$,  ao redor do qual é feita  aproximação
da função $\mathbf{f}(\mathbf{x})$,
e $\mathbf{J}(\mathbf{p})$ é a matriz Jacobiana de $\mathbf{f}(\mathbf{x})$ avaliado no ponto $\mathbf{p}$.

A demostração pode ser vista na Prova \ref{proof:theo:derfxbCfxb}.
\end{theorem}
\index{Jacobian}
\index{Serie de Taylor}

%%%%%%%%%%%%%%%%%%%%%%%%%%%%%%%%%%%%%%%%%%%%%%%%%%%%%%%%%%%%%%%%%%%%%%%%%%%%%%%%%%%%%%%
%%%%%%%%%%%%%%%%%%%%%%%%%%%%%%%%%%%%%%%%%%%%%%%%%%%%%%%%%%%%%%%%%%%%%%%%%%%%%%%%%%%%%%%
\section{Derivada de segundo ordem de $||\mathbf{f}(\mathbf{x})-\mathbf{b}||_{\mathbf{C}}^2$  
(Aproximação)
}

\begin{theorem}[Valor exato]\label{theo:der2fxbCfxb0}
Se 
$\mathbf{x}\in \mathbb{R}^N$ é um vetor coluna, 
$\mathbf{b}\in \mathbb{R}^M$ é um vetor coluna,  
$\mathbf{f}: \mathbb{R}^{N}\rightarrow \mathbb{R}^{M}$ é uma função de valor vectorial,
$\mathbf{C} \in \mathbb{R}^{M\times M}$ é uma matriz diagonal, e
definida a função $e(\mathbf{x})$,
\begin{equation}
e(\mathbf{x})= ||\mathbf{f}(\mathbf{x})-\mathbf{b}||_{\mathbf{C}}^2.
\end{equation}
Então a matriz Hessiana $\mathbf{H}(\mathbf{x})$ de $e(\mathbf{x})$ é igual a:
\begin{equation}
\mathbf{H}(\mathbf{x}) = \frac{\partial}{\partial \mathbf{x}^{\transpose}}\left(  
\frac{\partial e(\mathbf{x}) }{\partial \mathbf{x}} \right) = 2 
\mathbf{B}_{H}(\mathbf{x}) \mathbf{B}_{D}(\mathbf{x})
+2 \mathbf{J}(\mathbf{x})^{\transpose}\mathbf{C} \mathbf{J}(\mathbf{x}),
\end{equation}
onde 
\begin{equation}
 \mathbf{B}_{H}(\mathbf{x})={\bigcup\limits_{n=1}^{\rightarrow}}^{N}\left\{\frac{\partial \mathbf{J}(\mathbf{x})^{\transpose} }{\partial x_{n}}\right\},
\end{equation}
\begin{equation}
 \mathbf{B}_{D}(\mathbf{x})={\bigcup\limits_{n=1}^{\searrow}}^{N}\left\{ \mathbf{C} \left( \mathbf{f}(\mathbf{x})-\mathbf{b} \right) \right\}. 
\end{equation}

A demostração pode ser vista na Prova \ref{proof:theo:der2fxbCfxb0}.
\end{theorem}

\begin{theorem}[Valor aproximado]\label{theo:der2fxbCfxb}
Se 
$\mathbf{x}\in \mathbb{R}^N$ é um vetor coluna, 
$\mathbf{b}\in \mathbb{R}^M$ é um vetor coluna,  
$\mathbf{f}: \mathbb{R}^{N}\rightarrow \mathbb{R}^{M}$ é uma função de valor vectorial,
$\mathbf{C} \in \mathbb{R}^{M\times M}$ é uma matriz diagonal, e
definida a função $e(\mathbf{x})$,
\begin{equation}
e(\mathbf{x})= ||\mathbf{f}(\mathbf{x})-\mathbf{b}||_{\mathbf{C}}^2.
\end{equation}
Então a matriz Hessiana de $e(\mathbf{x})$ é igual a:
\begin{equation}
\frac{\partial}{\partial \mathbf{x}^{\transpose}}\left(  
\frac{\partial e(\mathbf{x}) }{\partial \mathbf{x}} \right) \approx
2 \mathbf{J}(\mathbf{p})^{\transpose}\mathbf{C} \mathbf{J}(\mathbf{p})
\end{equation}

Onde é considerada a aproximação
$\mathbf{f}(\mathbf{x})\approx \mathbf{f}(\mathbf{p})+\mathbf{J}(\mathbf{p})\left(\mathbf{x}-\mathbf{p}\right)$,
usando a serie de Taylor para funções multivariáveis. Sendo $\mathbf{p}$ um ponto fixo no domínio de $\mathbf{f}(\mathbf{x})$,  ao redor do qual é feita  aproximação
da função $\mathbf{f}(\mathbf{x})$,
e $\mathbf{J}(\mathbf{p})$ é a matriz Jacobiana de $\mathbf{f}(\mathbf{x})$ avaliado no ponto $\mathbf{p}$.

A demostração pode ser vista na Prova \ref{proof:theo:der2fxbCfxb}.
\end{theorem}
\index{Hessian}
\index{Serie de Taylor}


%%%%%%%%%%%%%%%%%%%%%%%%%%%%%%%%%%%%%%%%%%%%%%%%%%%%%%%%%%%%%%%%%%%%%%%%%%%%%%%%%%%%%%%
%%%%%%%%%%%%%%%%%%%%%%%%%%%%%%%%%%%%%%%%%%%%%%%%%%%%%%%%%%%%%%%%%%%%%%%%%%%%%%%%%%%%%%%
\section{Derivada de $||\mathbf{f}(\mathbf{x})-\mathbf{b}||_{\mathbf{C}}^2+\alpha||\mathbf{x}-\mathbf{q}||_{\mathbf{D}}^2$  
(Aproximação)
}

\begin{theorem}\label{theo:derfxbCfxbxqDxq}
Se 
$\mathbf{x}\in \mathbb{R}^N$ é um vetor coluna, 
$\mathbf{b}\in \mathbb{R}^M$ é um vetor coluna,
$\mathbf{q}\in \mathbb{R}^N$ é um vetor coluna, 
$\mathbf{f}: \mathbb{R}^{N}\rightarrow \mathbb{R}^{M}$ é uma função de valor vectorial, 
$\mathbf{C} \in \mathbb{R}^{M\times M}$ é uma matriz diagonal, e
$\mathbf{D} \in \mathbb{R}^{N\times N}$ é uma matriz diagonal, 
então se cumpre que:
\begin{equation}
\frac{\partial \left(||\mathbf{f}(\mathbf{x})-\mathbf{b}||_{\mathbf{C}}^2+\alpha||\mathbf{x}-\mathbf{q}||_{\mathbf{D}}^2\right)}{\partial \mathbf{x}} \approx
2 \mathbf{J}(\mathbf{p})^{\transpose}\mathbf{C}\left[\mathbf{J}(\mathbf{p})\mathbf{x} - \mathbf{J}(\mathbf{p})\mathbf{p}-\mathbf{b}+\mathbf{f}(\mathbf{p})\right]
+2\alpha\mathbf{D}\left(\mathbf{x}-\mathbf{q}\right)
\end{equation}

Onde é considerada a aproximação
$\mathbf{f}(\mathbf{x})\approx \mathbf{f}(\mathbf{p})+\mathbf{J}(\mathbf{p})\left(\mathbf{x}-\mathbf{p}\right)$,
usando a serie de Taylor para funções multivariáveis. Sendo $\mathbf{p}$ um ponto fixo no domínio de $\mathbf{f}(\mathbf{x})$,  ao redor do qual é feita  aproximação
da função $\mathbf{f}(\mathbf{x})$,
e $\mathbf{J}(\mathbf{p})$ é a matriz Jacobiana de $\mathbf{f}(\mathbf{x})$ avaliado no ponto $\mathbf{p}$.

A demostração pode ser vista na Prova \ref{proof:theo:derfxbCfxbxqDxq}.
\end{theorem}

\index{Serie de Taylor}