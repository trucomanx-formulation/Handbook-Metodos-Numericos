\chapterimage{chapter_head_2.pdf} % Chapter heading image

\chapter{Derivadas de funções com variável vetorial}

%%%%%%%%%%%%%%%%%%%%%%%%%%%%%%%%%%%%%%%%%%%%%%%%%%%%%%%%%%%%%%%%%%%%%%%%%%%%%%%%%%%%%%%
\section{Derivadas do escalar $f(\mathbf{x})$}
\begin{definition}
Se $\mathbf{x}$ é um vetor coluna de $N$ elementos e $f(\mathbf{x})$ é 
uma função de $\mathbb{R}^{N}$ a $\mathbb{R}$, então definimos que:
\begin{equation}
\frac{\partial f(\mathbf{x}) }{\partial \mathbf{x}}= 
\left[
\begin{matrix}
\frac{\partial f(\mathbf{x}) }{\partial x_{1}}&
\frac{\partial f(\mathbf{x}) }{\partial x_{2}}&
\frac{\partial f(\mathbf{x}) }{\partial x_{3}}&
\hdots&
\frac{\partial f(\mathbf{x}) }{\partial x_{N}}
\end{matrix}
\right]^{T}
\end{equation}
\end{definition}

\begin{definition}
Se $\mathbf{x}^{T}$ é um vetor linha de $N$ elementos e $f(\mathbf{x})$ é 
uma função de $\mathbb{R}^{N}$ a $\mathbb{R}$, então definimos que:
\begin{equation}
\frac{\partial f(\mathbf{x}) }{\partial \mathbf{x}^{T}}= 
\left[
\begin{matrix}
\frac{\partial f(\mathbf{x}) }{\partial x_{1}}&
\frac{\partial f(\mathbf{x}) }{\partial x_{2}}&
\frac{\partial f(\mathbf{x}) }{\partial x_{3}}&
\hdots&
\frac{\partial f(\mathbf{x}) }{\partial x_{N}}
\end{matrix}
\right]
\end{equation}
\end{definition}

%%%%%%%%%%%%%%%%%%%%%%%%%%%%%%%%%%%%%%%%%%%%%%%%%%%%%%%%%%%%%%%%%%%%%%%%%%%%%%%%%%%%%%%
\section{Derivadas de $\mathbf{A}\mathbf{x}$}

\begin{theorem}
Se $\mathbf{x}$ é um vetor coluna de $N$ elementos e $\mathbf{A}$ uma matriz de $M\times N$, então se cumpre que:
\begin{equation}
\frac{\partial \mathbf{A}\mathbf{x}}{\partial x_n}=a_{:n}
\end{equation}
\end{theorem}

\begin{corollaryT}[Corollary name]
\begin{equation}
\frac{\partial \mathbf{A}\mathbf{x}}{\partial \mathbf{x}^{T}}=
\left[
\begin{matrix}
 a_{:1} &  a_{:2} &  \cdots &  a_{:N}
\end{matrix}
\right]=
\mathbf{A}
\end{equation}
\end{corollaryT}

\begin{corollaryT}[Corollary name]
\begin{equation}
\frac{\partial \mathbf{A}\mathbf{x}}{\partial \mathbf{x}}=
\left[
\begin{matrix}
 a_{:1}\\
 a_{:2}\\
 \vdots\\
 a_{:N}\\
\end{matrix}
\right]
\end{equation}

\end{corollaryT}
\begin{corollaryT}[Corollary name]
\begin{equation}
\frac{\partial \mathbf{a}^{T}\mathbf{x}}{\partial \mathbf{x}^{T}}=\mathbf{a}^{T}
\end{equation}
\end{corollaryT}

\begin{corollaryT}[Corollary name]
\begin{equation}
\frac{\partial \mathbf{a}^{T}\mathbf{x}}{\partial \mathbf{x}}=\mathbf{a}
\end{equation}
\end{corollaryT}

%%%%%%%%%%%%%%%%%%%%%%%%%%%%%%%%%%%%%%%%%%%%%%%%%%%%%%%%%%%%%%%%%%%%%%%%%%%%%%%%%%%%%%%
\section{Derivadas de $||\mathbf{A}\mathbf{x}||^2$ 
}

\begin{theorem}
Se $\mathbf{x}$ é um vetor coluna de $N$ elementos e $\mathbf{A}$ uma matriz de $M\times N$, então se cumpre que:
\begin{equation}
\begin{matrix}
\frac{\partial ||\mathbf{A}\mathbf{x}||^2 }{\partial x_n}&=&
\frac{\partial \left(\mathbf{A}\mathbf{x}\right)^{T}\left(\mathbf{A}\mathbf{x}\right)}{\partial x_n}&=&
2\left(\mathbf{A}\mathbf{x}\right)^{T}a_{:n}\\
~&~&~&=& 2\left(a_{:n}\right)^{T}\mathbf{A}\mathbf{x}
\end{matrix}
\end{equation}
\end{theorem}

\begin{corollaryT}[Corollary name]
\begin{equation}
\frac{\partial ||\mathbf{A}\mathbf{x}||^2 }{\partial \mathbf{x}^{T}}=
\frac{\partial \left(\mathbf{A}\mathbf{x}\right)^{T}\left(\mathbf{A}\mathbf{x}\right)}{\partial \mathbf{x}^{T}}=
2\left(\mathbf{A}^{T}\mathbf{A}\mathbf{x}\right)^{T}
\end{equation}
\end{corollaryT}

\begin{corollaryT}[Corollary name]
\begin{equation}
\frac{\partial ||\mathbf{A}\mathbf{x}||^2 }{\partial \mathbf{x}}=
\frac{\partial \left(\mathbf{A}\mathbf{x}\right)^{T}\left(\mathbf{A}\mathbf{x}\right)}{\partial \mathbf{x}}=
2 \mathbf{A}^{T}\mathbf{A}\mathbf{x}
\end{equation}
\end{corollaryT}


%%%%%%%%%%%%%%%%%%%%%%%%%%%%%%%%%%%%%%%%%%%%%%%%%%%%%%%%%%%%%%%%%%%%%%%%%%%%%%%%%%%%%%%
\section{Derivadas de $||\mathbf{A}\mathbf{x}-\mathbf{b}||^2$ 
}

\begin{theorem}
Se $\mathbf{x}$ é um vetor coluna de $N$ elementos,
$\mathbf{b}$ é um vetor coluna de $M$ elementos e 
$\mathbf{A}$ uma matriz de $M\times N$, então se cumpre que:
\begin{equation}
\begin{matrix}
\frac{\partial ||\mathbf{A}\mathbf{x}-\mathbf{b}||^2}{\partial x_n}&=&
\frac{\partial \left(\mathbf{A}\mathbf{x}-\mathbf{b}\right)^{T}\left(\mathbf{A}\mathbf{x}-\mathbf{b}\right)}{\partial x_n}&=&
2\left(\mathbf{A}\mathbf{x}\right)^{T}a_{:n}-2  \mathbf{b}^{T} a_{:n}\\
~&~&~&=& 2\left(a_{:n}\right)^{T}\mathbf{A}\mathbf{x}-2 a_{:n}^{T}  \mathbf{b}
\end{matrix}
\end{equation}
\end{theorem}


\begin{corollaryT}[Corollary name]
\begin{equation}
\frac{\partial ||\mathbf{A}\mathbf{x}-\mathbf{b}||^2}{\partial \mathbf{x}^{T}}=
2\left(\mathbf{x}^{T}\mathbf{A}^{T}- \mathbf{b}^{T} \right)\mathbf{A}
\end{equation}
\end{corollaryT}

\begin{corollaryT}[Corollary name]
\begin{equation}
\frac{\partial ||\mathbf{A}\mathbf{x}-\mathbf{b}||^2}{\partial \mathbf{x} }=
2 \mathbf{A}^{T}\left(\mathbf{A}\mathbf{x}-\mathbf{b}\right)
\end{equation}
\end{corollaryT}
%%%%%%%%%%%%%%%%%%%%%%%%%%%%%%%%%%%%%%%%%%%%%%%%%%%%%%%%%%%%%%%%%%%%%%%%%%%%%%%%%%%%%%%
\section{Derivadas de $||\mathbf{A}\mathbf{x}-\mathbf{b}||_{\mathbf{C}}^2$ 
}


\begin{theorem}
Se $\mathbf{x}$ é um vetor coluna de $N$ elementos,
$\mathbf{b}$ é um vetor coluna de $M$ elementos, 
$\mathbf{A}$ uma matriz de $M\times N$ e
$\mathbf{C}$ uma matriz diagonal de $M \times M$, então se cumpre que:
\begin{equation}
\frac{\partial ||\mathbf{A}\mathbf{x}-\mathbf{b}||_{\mathbf{C}}^2}{\partial \mathbf{x}^{T}}=
2\left(\mathbf{x}^{T}\mathbf{A}^{T}- \mathbf{b}^{T} \right)\mathbf{C}\mathbf{A}
\end{equation}
\begin{equation}
\frac{\partial ||\mathbf{A}\mathbf{x}-\mathbf{b}||_{\mathbf{C}}^2}{\partial \mathbf{x} }=
2 \mathbf{A}^{T}\mathbf{C}\left(\mathbf{A}\mathbf{x}-\mathbf{b}\right)
\end{equation}
\end{theorem}

%%%%%%%%%%%%%%%%%%%%%%%%%%%%%%%%%%%%%%%%%%%%%%%%%%%%%%%%%%%%%%%%%%%%%%%%%%%%%%%%%%%%%%%
\section{Derivadas de $||\mathbf{f}(\mathbf{x})-\mathbf{b}||_{\mathbf{C}}^2$  
(Aproximação)
}

\begin{theorem}
Se $\mathbf{x}$ é um vetor com $N$ elementos, $\mathbf{f}(\mathbf{x})$ e 
$\mathbf{b}$ são vetores coluna de $M$ elementos sendo $\mathbf{b}$ uma constante e
$\mathbf{C}$ uma matriz diagonal de $M \times M$, 
então se cumpre que:
\begin{equation}
\frac{\partial ||\mathbf{f}(\mathbf{x})-\mathbf{b}||_{\mathbf{C}}^2}{\partial \mathbf{x}} \approx
\frac{\partial ||\mathbf{f}(\mathbf{p})+\mathbf{J}(\mathbf{p})\left(\mathbf{x}-\mathbf{p}\right)-\mathbf{b}||_{\mathbf{C}}^2}{\partial \mathbf{x}}
\end{equation}
Onde é considerada a aproximação
$\mathbf{f}(\mathbf{x})\approx \mathbf{f}(\mathbf{p})+\mathbf{J}(\mathbf{p})\left(\mathbf{x}-\mathbf{p}\right)$,
usando a serie de Taylor. Sendo $\mathbf{p}$ um ponto fixo em $\mathbf{x}$,  ao redor do qual é feita  aproximação
da função $\mathbf{f}(\mathbf{x})$,
e $\mathbf{J}(\mathbf{p})$ é o jacobiano de $\mathbf{f}(\mathbf{x})$ avaliado no ponto $\mathbf{p}$.
Assim nos temos que:
\begin{equation}
\frac{\partial ||\mathbf{f}(\mathbf{p})+\mathbf{J}(\mathbf{p})\left(\mathbf{x}-\mathbf{p}\right)-\mathbf{b}||_{\mathbf{C}}^2}{\partial \mathbf{x}}=
2 \mathbf{J}(\mathbf{p})^{T}\mathbf{C}\left(\mathbf{J}(\mathbf{p})\mathbf{x} - \mathbf{J}(\mathbf{p})\mathbf{p}-\mathbf{b}+\mathbf{f}(\mathbf{p})\right)
\end{equation}
\end{theorem}


%%%%%%%%%%%%%%%%%%%%%%%%%%%%%%%%%%%%%%%%%%%%%%%%%%%%%%%%%%%%%%%%%%%%%%%%%%%%%%%%%%%%%%%
\section{Derivadas de $||\mathbf{f}(\mathbf{x})-\mathbf{b}||_{\mathbf{C}}^2+\alpha||\mathbf{x}-\mathbf{q}||_{\mathbf{D}}^2$  
(Aproximação)
}

\begin{theorem}
Se $\mathbf{x}$ e $\mathbf{q}$  são vetores com $N$ elementos sendo $\mathbf{q}$ uma constante,  e
$\mathbf{D}$ uma matriz diagonal de $N \times N$, ademais $\mathbf{f}(\mathbf{x})$ e 
$\mathbf{b}$ são vetores coluna de $M$ elementos sendo $\mathbf{b}$ uma constante, e
$\mathbf{C}$ uma matriz diagonal de $M \times M$, 
então se cumpre que:
\begin{equation}
\frac{\partial \left(||\mathbf{f}(\mathbf{x})-\mathbf{b}||_{\mathbf{C}}^2+\alpha||\mathbf{x}-\mathbf{q}||_{\mathbf{D}}^2\right)}{\partial \mathbf{x}} \approx
2 \mathbf{J}(\mathbf{p})^{T}\mathbf{C}\left(\mathbf{J}(\mathbf{p})\mathbf{x} - \mathbf{J}(\mathbf{p})\mathbf{p}-\mathbf{b}+\mathbf{f}(\mathbf{p})\right)
+2\alpha\mathbf{D}\left(\mathbf{x}-\mathbf{q}\right)
\end{equation}
Onde é considerada a aproximação
$\mathbf{f}(\mathbf{x})\approx \mathbf{f}(\mathbf{p})+\mathbf{J}(\mathbf{p})\left(\mathbf{x}-\mathbf{p}\right)$,
usando a serie de Taylor. Sendo $\mathbf{p}$ um ponto fixo em $\mathbf{x}$,  ao redor do qual é feita  aproximação
da função $\mathbf{f}(\mathbf{x})$,
e $\mathbf{J}(\mathbf{p})$ é o jacobiano de $\mathbf{f}(\mathbf{x})$ avaliado no ponto $\mathbf{p}$.
\end{theorem}