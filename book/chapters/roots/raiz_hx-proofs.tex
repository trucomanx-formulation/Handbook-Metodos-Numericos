\section{Provas dos teoremas}
 
%%%%%%%%%%%%%%%%%%%%%%%%%%%%%%%%%%%%%%%%%%%%%%%%%%%%%%%%%%%%%%%%%%%%%%%%%%%%%%%%%%%%%%%
%%%%%%%%%%%%%%%%%%%%%%%%%%%%%%%%%%%%%%%%%%%%%%%%%%%%%%%%%%%%%%%%%%%%%%%%%%%%%%%%%%%%%%%
\begin{myproofT}[Relativa ao Teorema \ref{theo:rootshx}:]\label{proof:theo:rootshx}
Dados
os escalares $\delta \in \mathbb{R}_+$, 
$x \in \mathbb{R}$, 
uma função $h:\mathbb{R} \rightarrow \mathbb{R}$, e 
conhecida a validade da Eq. (\ref{eq:prrof:rootshx1}),
\begin{equation}\label{eq:prrof:rootshx1}
0=h(x);
\end{equation}
se desejarmos ter o valor $x=\hat{x}$ que cumpra que $||h(\hat{x})||<\delta$, 
é aplicado o critério mostrado no Teorema \ref{theo:minhxhx}, no qual se indica que uma forma de achar o 
$x=\hat{x}$ que minimize $e(x)=||h(x)||^2$ é mediante a seguinte equação iterativa.  
\begin{equation}\label{eq:prrof:rootshx1:2}
x_{k}=x_{k-1}-\frac{h(x_{k-1})}{h'(x_{k-1})}.
\end{equation}

\end{myproofT}

\begin{myproofT}[Prova da continuidade de 
$h(x)/h'(x)$:]\label{proof:theo:cont:rootshx}
Conhecida uma função $h(x)$  diferenciável em $x=a$, 
ao menos até a $n$-ésima derivada na qual $h^{(n)}(a)\neq 0$.
Se $h(a)=0$ e $h'(a)=0$ o fator
\begin{equation}\label{eq:prrof:cont1}
F(x)=\frac{h(x)}{h'(x)}
\end{equation}
causa uma indeterminação de zero dividido por zero para $x=a$.
Para resolver esse problema, é aplicada de forma consecutiva a regra de l'Hôpital
sobre a Eq. (\ref{eq:prrof:cont1}); assim podemos afirmar
\begin{equation}\label{eq:prrof:cont2}
F(a)=\lim_{x\rightarrow a}\frac{h(x)}{h'(x)}=\frac{h^{(n-1)}(a)}{h^{(n)}(a)},
\end{equation}
sendo que $n$ é escolhido como o primeiro valor avaliado em $a$, 
no qual se cumpra que $h^{(n)}(a)\neq 0$. 
Assim, podemos definir que os lugares onde $h(a)=0$ e $h'(a)=0$,
 também tem um valor $F(a)\neq 0$,
se $h(x)$ tem ao menos uma derivada $n$-ésima em $x=a$ que provoque $h^{(n)}(a)\neq 0$.

Uma forma simples de expressar isso é indicar que $F(a)\neq 0$, se é possível
expressar $h(x)$ em \hyperref[def:taylor]{\textbf{série de Taylor}} ao redor de $x=a$. 
\end{myproofT}

%%%%%%%%%%%%%%%%%%%%%%%%%%%%%%%%%%%%%%%%%%%%%%%%%%%%%%%%%%%%%%%%%%%%%%%%%%%%%%%%%%%%%%%
%%%%%%%%%%%%%%%%%%%%%%%%%%%%%%%%%%%%%%%%%%%%%%%%%%%%%%%%%%%%%%%%%%%%%%%%%%%%%%%%%%%%%%%
\begin{myproofT}[Relativa ao Teorema \ref{theo:rootshxreg}:]\label{proof:theo:rootshxreg}
Dados
os escalares $\delta \in \mathbb{R}_+$,
$\alpha \in \mathbb{R}_+$, 
$x \in \mathbb{R}$, 
uma função $h:\mathbb{R} \rightarrow \mathbb{R}$, e 
conhecida a validade da Eq. (\ref{eq:prrof:rootshxreg1}),
\begin{equation}\label{eq:prrof:rootshxreg1}
0=h(x);
\end{equation}
se desejarmos ter o valor $x=\hat{x}$ que cumpra que $||h(\hat{x})||<\delta$, 
é aplicado o critério mostrado no Teorema \ref{theo:minhxhxxoxo},
em que se indica que uma forma de achar o valor 
$x=\hat{x}$ que minimiza $e(x)=||h(x)||^2+\alpha||x-x_{last}||^2$, 
é mediante a seguinte equação iterativa,  
\begin{equation}\label{eq:prrof:rootshxreg1:2}
x_{k}=x_{k-1}-\frac{h'(x_{k-1}) h(x_{k-1})}{h'(x_{k-1})^2+\alpha};
\end{equation}
na qual $\alpha$ é um multiplicador de Lagrange escolhido por nós.

A Eq. (\ref{eq:prrof:rootshxreg1:2}) tenta minimizar simultaneamente 
as funções de custo $||h(x)||^2$ 
e $||x-x_{last}||^2$, para nos aproximar a Eq. (\ref{eq:prrof:rootshxreg1});
a busca iterativa acha a convergência quando obtém valores $x_{k}\approx x_{k-1}$;
porém esta pode acontecer quando achamos um mínimo local ou global;
por isso, é importante verificar se $||h(x)||<\delta$. 
\end{myproofT}
