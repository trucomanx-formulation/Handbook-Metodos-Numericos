%%%%%%%%%%%%%%%%%%%%%%%%%%%%%%%%%%%%%%%%%%%%%%%%%%%%%%%%%%%%%%%%%%%%%%%%%%%%%%%%%%%%%%%
%%%%%%%%%%%%%%%%%%%%%%%%%%%%%%%%%%%%%%%%%%%%%%%%%%%%%%%%%%%%%%%%%%%%%%%%%%%%%%%%%%%%%%%
%%%%%%%%%%%%%%%%%%%%%%%%%%%%%%%%%%%%%%%%%%%%%%%%%%%%%%%%%%%%%%%%%%%%%%%%%%%%%%%%%%%%%%%
\section{Matrizes semelhantes}

\index{Matriz!Semelhante}

\begin{definition}[Matrizes semelhantes:]\label{def:similhante0}
Conhecidas as matrizes  $\MATRIX{A} \in \mathbb{C}^{N \times N}$ e $\MATRIX{B} \in \mathbb{C}^{N \times N}$,
elas são chamadas semelhantes \cite[pp. 67]{golub2013matrix} 
se existe uma matriz $\MATRIX{P} \in \mathbb{C}^{N \times N}$ que provoque
\begin{equation}
\MATRIX{A} = \MATRIX{P}^{-1} \MATRIX{B} \MATRIX{P}.
\end{equation}
\end{definition}

\begin{theorem}[Autovalores em matrizes semelhantes:]\label{theo:similhante1}
Conhecidas as matrizes $\MATRIX{A} \in \mathbb{C}^{N \times N}$ e $\MATRIX{B} \in \mathbb{C}^{N \times N}$,
\begin{itemize}
\item Se as matrizes $\MATRIX{A}$ e $\MATRIX{B}$ são semelhantes, então elas têm os mesmos autovalores
\footnote{A demonstração pode ser vista na Prova \ref{proof:theo:similhante1}.} \cite[pp. 67]{golub2013matrix}.
\end{itemize}
\end{theorem}

%%%%%%%%%%%%%%%%%%%%%%%%%%%%%%%%%%%%%%%%%%%%%%%%%%%%%%%%%%%%%%%%%%%%%%%%%%%%%%%%%%%%%%%
%%%%%%%%%%%%%%%%%%%%%%%%%%%%%%%%%%%%%%%%%%%%%%%%%%%%%%%%%%%%%%%%%%%%%%%%%%%%%%%%%%%%%%%
%%%%%%%%%%%%%%%%%%%%%%%%%%%%%%%%%%%%%%%%%%%%%%%%%%%%%%%%%%%%%%%%%%%%%%%%%%%%%%%%%%%%%%%
\section{Diagonalização de matrizes}
\index{Matriz!Diagonalizável}

\begin{definition}[Matriz diagonalizável:]\label{def:diagonalization0}
Conhecida uma matriz quadrada $\MATRIX{A} \in \mathbb{C}^{N \times N}$ com
autovalores $\lambda_n$, $\forall n \in \{1, 2, ..., N\}$.
\begin{itemize}
\item Esta é definida como diagonalizável se é semelhante a uma matriz diagonal \cite[pp. 67]{golub2013matrix};
quer dizer, existe uma matriz $\MATRIX{P}$ que provoca
\begin{equation}
\lambda_\MATRIX{A}\equiv
\begin{bmatrix}
\lambda_1 & 0         & ...    & 0 \\
0         & \lambda_2 & ...    & 0 \\
\vdots    & \vdots    & \ddots & \vdots \\
0         & 0         & ...    & \lambda_N
\end{bmatrix}
=\MATRIX{P}^{-1}\MATRIX{A}\MATRIX{P}.
\end{equation}
\end{itemize}
\end{definition}



\begin{theorem}[Diagonalização de matrizes:]\label{theo:diagonalization1}
Conhecida uma matriz quadrada $\MATRIX{A} \in \mathbb{C}^{N \times N}$ com
autovalores $\lambda_n$, $\forall n \in \{1, 2, ..., N\}$.
\begin{itemize}
\item Se $\MATRIX{A}$ tem $N$  autovetores $\VECTOR{v}_n \in \mathbb{C}^{N}$  linearmente independentes
agrupados na matriz $\MATRIX{V}=\left[\VECTOR{v}_1\quad \VECTOR{v}_2\quad  ...\quad \VECTOR{v}_N\right]$.
Então\footnote{A demonstração pode ser vista na Prova \ref{proof:theo:diagonalization1:a}.} 
$\MATRIX{A}$ é diagonalizável \cite[pp. 67]{golub2013matrix} \cite[pp. 61]{axelsson1996iterative};
com
\begin{equation}
\begin{bmatrix}
\lambda_1 & 0         & ...    & 0 \\
%0         & \lambda_2 & ...    & 0 \\
\vdots    & \vdots    & \ddots & \vdots \\
0         & 0         & ...    & \lambda_N
\end{bmatrix}
=\MATRIX{V}^{-1}\MATRIX{A}\MATRIX{V}.
\end{equation}
\item Se $\MATRIX{A}\in \mathbb{R}^{N \times N}$ é uma matriz simétrica, então\footnote{A
demonstração é uma consequência imediata do Teorema \ref{theo:ShurDecomposition0} (Decomposição de Schur).}
existe uma matriz $\MATRIX{Q}$ \hyperref[def:ortogonalmatrix0]{\textbf{ortogonal}} que cumpre \cite[pp. 67, 440]{golub2013matrix};
com
\begin{equation}
\begin{bmatrix}
\lambda_1 & 0         & ...    & 0 \\
%0         & \lambda_2 & ...    & 0 \\
\vdots    & \vdots    & \ddots & \vdots \\
0         & 0         & ...    & \lambda_N
\end{bmatrix}
=\MATRIX{Q}^{-1}\MATRIX{A}\MATRIX{Q}
\equiv \MATRIX{Q}^{\transpose}\MATRIX{A}\MATRIX{Q},
\end{equation}
sendo que $\MATRIX{Q}=\left[\VECTOR{q}_1\quad \VECTOR{q}_2\quad ...\quad \VECTOR{q}_N\right]$,
e $\VECTOR{q}_i^{\transpose}\VECTOR{q}_j=1$ quando $i=j$ e zero em outros casos.
\end{itemize}
\end{theorem}
