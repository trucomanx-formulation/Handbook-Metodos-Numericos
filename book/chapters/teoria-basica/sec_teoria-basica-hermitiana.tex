%%%%%%%%%%%%%%%%%%%%%%%%%%%%%%%%%%%%%%%%%%%%%%%%%%%%%%%%%%%%%%%%%%%%%%%%%%%%%%%%%%%%%%%
%%%%%%%%%%%%%%%%%%%%%%%%%%%%%%%%%%%%%%%%%%%%%%%%%%%%%%%%%%%%%%%%%%%%%%%%%%%%%%%%%%%%%%%
%%%%%%%%%%%%%%%%%%%%%%%%%%%%%%%%%%%%%%%%%%%%%%%%%%%%%%%%%%%%%%%%%%%%%%%%%%%%%%%%%%%%%%%
\section{ Matrizes hermitianas}

\index{Matriz!Hermitiana}

\begin{definition}[Matriz hermitiana (matriz autoadjunta):]\label{def:hermitianamatrix0}
Dada, $\MATRIX{A} \in \mathbb{C}^{N \times N}$ uma matriz quadrada. 
\begin{itemize}
\item $\MATRIX{A}$ é definida como uma matriz hermitiana quando se cumpre que 
\cite[pp. 12]{pillai2008space} \cite[pp. 411]{korn2013mathematical}, 
\begin{equation}
\MATRIX{A} = \MATRIX{A}^{\ast} \equiv \overline{\MATRIX{A}}^{\transpose}.
\end{equation}
\item Outras notações para descrever as matrizes hermitianas são 
\begin{equation}
\MATRIX{A}^{\ast} \equiv \MATRIX{A}^{\mathsf{H}}\equiv \MATRIX{A}^{\dagger} 
\end{equation}
\end{itemize}
\end{definition}

\begin{tcbattention}
\begin{itemize}
%\item As matrizes hermitianas podem ser entendidas como uma extensão complexa das 
%\hyperref[def:symmetricmatrix0]{\textbf{matrizes simétricas}} reais.
\item $\MATRIX{A}^{\ast}$ é chamado ``conjugado transposto'' ou ``transposto hermitiano'' de $\MATRIX{A}$.
\end{itemize}
\end{tcbattention}

\index{Forma hermitiana}
\begin{definition}[Forma hermitiana:]\label{def:formhermitiana0}
Conhecida uma matriz hermitiana $\MATRIX{A}\in \mathbb{C}^{N \times N}$ e 
o vetor $\VECTOR{x}=[x_1,~x_2,~...,~x_N]^{\transpose}\in \mathbb{C}^{N}$,
a \textbf{forma hermitiana} $\phi\left(\VECTOR{x}\right)$ é definida como
\cite[pp. 386]{mirsky2012introduction} \cite[pp. 339]{datta2016matrix}, 
\begin{equation}
\phi\left(\VECTOR{x}\right) = 
\VECTOR{x}^{\ast}\MATRIX{A}\VECTOR{x} \equiv 
\sum_{i=1}^{N} \sum_{j=1}^{N} a_{ij}\overline{x}_i x_j.
\end{equation}
\end{definition}

\index{Forma hermitiana real}
\begin{definition}[Forma hermitiana real:]\label{def:formhermitianareal0}
Se $\MATRIX{W} \in \mathbb{R}^{N \times N}$ é uma \hyperref[def:symmetricmatrix0]{\textbf{matriz real e simétrica}}
e $\VECTOR{x}=[x_1,~x_2,~...,~x_N]^{\transpose}\in \mathbb{C}^{N}$,
então a \textbf{forma hermitiana real} 
$\phi\left(\VECTOR{x}\right) = \VECTOR{x}^{\ast}\MATRIX{W}\VECTOR{x}$ \cite[pp. 386]{mirsky2012introduction}.
\end{definition}

\index{Forma quadrática}
\begin{definition}[Forma hermitiana e forma quadrática:]\label{def:formhermitianareal1}
Se $\MATRIX{W} \in \mathbb{R}^{N \times N}$ é uma \hyperref[def:symmetricmatrix0]{\textbf{matriz real e simétrica}},
e  $\VECTOR{x}=[x_1,~x_2,~...,~x_N]^{\transpose}\in \mathbb{R}^{N}$,
então a forma hermitiana se transforma na \textbf{forma quadrática} 
$\phi\left(\VECTOR{x}\right) = \VECTOR{x}^{\transpose}\MATRIX{W}\VECTOR{x}$ \cite[pp. 386]{mirsky2012introduction}.
\end{definition}

%%%%%%%%%%%%%%%%%%%%%%%%%%%%%%%%%%%%%%%%%%%%%%%%%%%%%%%%%%%%%%%%%%%%%%%%%%%%%%%%
\begin{theorem}[Autovalores nas matrizes hermitianas:]\label{theo:autovalorhermitianmatrix0}
Dada uma matriz $\MATRIX{A} \in \mathbb{C}^{N \times N}$.
\begin{itemize}
\item Se $\MATRIX{A}$ é uma matriz hermitiana, então\footnote{A
demonstração pode ser vista na Prova \ref{proof:theo:autovalorhermitianmatrix0}.} 
os seus autovalores são reais \cite[pp. 309]{robbin2018matrix}.
\end{itemize}
\end{theorem}

%%%%%%%%%%%%%%%%%%%%%%%%%%%%%%%%%%%%%%%%%%%%%%%%%%%%%%%%%%%%%%%%%%%%%%%%%%%%%%%%
\begin{theorem}[Autovetores ortogonais nas matrizes hermitianas:]\label{theo:autovetorortogonalhermitian0}
Dada uma matriz hermitiana $\MATRIX{A} \in \mathbb{C}^{N \times N}$.
\begin{itemize}
\item Se dois eigenvetores de $\MATRIX{A}$ correspondem a dois autovalores distintos, então\footnote{A
demonstração pode ser vista na Prova \ref{proof:theo:autovetorortogonalhermitian0}.} 
os autovetores são ortogonais \cite[pp. 309]{robbin2018matrix}.
\end{itemize}
\end{theorem}

%%%%%%%%%%%%%%%%%%%%%%%%%%%%%%%%%%%%%%%%%%%%%%%%%%%%%%%%%%%%%%%%%%%%%%%%%%%%%%%%%%%%%%%
%%%%%%%%%%%%%%%%%%%%%%%%%%%%%%%%%%%%%%%%%%%%%%%%%%%%%%%%%%%%%%%%%%%%%%%%%%%%%%%%%%%%%%%
%%%%%%%%%%%%%%%%%%%%%%%%%%%%%%%%%%%%%%%%%%%%%%%%%%%%%%%%%%%%%%%%%%%%%%%%%%%%%%%%%%%%%%%
\section{ Diagonalização de matrizes hermitianas}

%%%%%%%%%%%%%%%%%%%%%%%%%%%%%%%%%%%%%%%%%%%%%%%%%%%%%%%%%%%%%%%%%%%%%%%%%%%%%%%%
\index{Matriz!Unitária }
\begin{definition}[Matriz unitária:]\label{def:unitarymatrix0}
Conhecida uma matriz $\MATRIX{A} \in \mathbb{C}^{N \times N}$, 
esta é definida unitária quando \cite[pp. 80]{golub2013matrix} 
\begin{equation}
\MATRIX{A}^{\ast} = \MATRIX{A}^{-1}.
\end{equation}
\end{definition}
%%%%%%%%%%%%%%%%%%%%%%%%%%%%%%%%%%%%%%%%%%%%%%%%%%%%%%%%%%%%%%%%%%%%%%%%%%%%%%%%
\index{Matriz!Normal}
\begin{definition}[Matriz normal:]\label{def:normalmatrix0}
Conhecida uma matriz $\MATRIX{A} \in \mathbb{C}^{N \times N}$
esta é definida normal quando \cite[pp. 226]{hartfiel2000matrix} 
\begin{equation}
\MATRIX{A}^{\ast} \MATRIX{A}= \MATRIX{A} \MATRIX{A}^{\ast}.
\end{equation}
\end{definition}

\begin{tcbattention}
\begin{itemize}
\item As matrizes simétricas e as matrizes hermitianas são matrizes normais.
\end{itemize}
\end{tcbattention}

%%%%%%%%%%%%%%%%%%%%%%%%%%%%%%%%%%%%%%%%%%%%%%%%%%%%%%%%%%%%%%%%%%%%%%%%%%%%%%%%
\begin{theorem}[Matriz unitária e matriz hermitiana:]\label{theo:unitariahermitian0}
Dada uma matriz hermitiana $\MATRIX{A} \in \mathbb{C}^{N \times N}$.
\begin{itemize}
\item Se $\MATRIX{U}$ é uma \hyperref[def:unitarymatrix0]{\textbf{matriz unitária}}, 
então $\MATRIX{U}^{\ast} \MATRIX{A} \MATRIX{U}$ 
é também uma matriz hermitiana \cite[pp. 59]{axelsson1996iterative}.
\end{itemize}
\end{theorem}

%%%%%%%%%%%%%%%%%%%%%%%%%%%%%%%%%%%%%%%%%%%%%%%%%%%%%%%%%%%%%%%%%%%%%%%%%%%%%%%%
\begin{theorem}[Decomposição de Schur:]\label{theo:ShurDecomposition0}
Dada uma matriz $\MATRIX{A} \in \mathbb{C}^{N \times N}$.
\begin{itemize}
\item Esta é sempre  unitariamente similar a uma matriz triangular superior 
\cite[pp. 59]{axelsson1996iterative} \cite[pp. 351]{golub2013matrix},
\begin{equation}
\MATRIX{U}^{-1} \MATRIX{A} \MATRIX{U} \equiv
\MATRIX{U}^{\ast} \MATRIX{A} \MATRIX{U} =
\begin{bmatrix}
\lambda_1 & b_{12}    & ...    & b_{1N} \\
0         & \lambda_2 & ...    & b_{2N} \\
\vdots    & \vdots    & \ddots & \vdots \\
0         & 0         & ...    & \lambda_N
\end{bmatrix},
\end{equation}
sendo que $\MATRIX{U}$ é uma \hyperref[def:unitarymatrix0]{\textbf{matriz unitária}},
$\lambda_n$ são os autovalores de $\MATRIX{A}$, e $b_{ij}\in \mathbb{C}$.
\end{itemize}
\end{theorem}

%%%%%%%%%%%%%%%%%%%%%%%%%%%%%%%%%%%%%%%%%%%%%%%%%%%%%%%%%%%%%%%%%%%%%%%%%%%%%%%%
\begin{theorem}[Matriz hermitiana similar a uma matriz diagonal:]\label{theo:HermitianDDecomposition0}
Dada uma matriz hermitiana $\MATRIX{A} \in \mathbb{C}^{N \times N}$.
\begin{itemize}
\item Esta é sempre\footnote{A
demonstração pode ser vista na Prova \ref{proof:theo:HermitianDDecomposition0}.}  
unitariamente similar a uma matriz diagonal
\cite[pp. 60]{axelsson1996iterative},
\begin{equation}
\MATRIX{U}^{-1} \MATRIX{A} \MATRIX{U} \equiv
\MATRIX{U}^{\ast} \MATRIX{A} \MATRIX{U} =
\funcdiag\left(
\begin{bmatrix}
\lambda_1 & \lambda_2 & ...    & \lambda_N \\
\end{bmatrix}^{\transpose}\right),
\end{equation}
sendo que $\MATRIX{U}$ é uma \hyperref[def:unitarymatrix0]{\textbf{matriz unitária}}, e
$\lambda_n$ são os autovalores de $\MATRIX{A}$.
\end{itemize}
\end{theorem}

%%%%%%%%%%%%%%%%%%%%%%%%%%%%%%%%%%%%%%%%%%%%%%%%%%%%%%%%%%%%%%%%%%%%%%%%%%%%%%%%%%%%%%%
%%%%%%%%%%%%%%%%%%%%%%%%%%%%%%%%%%%%%%%%%%%%%%%%%%%%%%%%%%%%%%%%%%%%%%%%%%%%%%%%%%%%%%%
%%%%%%%%%%%%%%%%%%%%%%%%%%%%%%%%%%%%%%%%%%%%%%%%%%%%%%%%%%%%%%%%%%%%%%%%%%%%%%%%%%%%%%%
\section{ Matrizes hermitianas definidas positivas}


%%%%%%%%%%%%%%%%%%%%%%%%%%%%%%%%%%%%%%%%%%%%%%%%%%%%%%%%%%%%%%%%%%%%%%%%%%%%%%%%
\index{Matriz!Hermitiana definida positiva}
\begin{definition}[Matriz hermitiana definida positiva:]\label{def:hermitianapositivematrix0}
Conhecida uma matriz hermitiana $\MATRIX{A} \in \mathbb{C}^{N \times N}$. 
\begin{itemize}
\item Esta é definida positiva, se para todo vetor não nulo $\VECTOR{x} \in \mathbb{C}^N$
se cumpre que \cite[pp. 63]{ipsen2009numerical} 
\begin{equation}
\VECTOR{x}^{\ast} \MATRIX{A} \VECTOR{x} > 0.
\end{equation}
\end{itemize}
\end{definition}


%%%%%%%%%%%%%%%%%%%%%%%%%%%%%%%%%%%%%%%%%%%%%%%%%%%%%%%%%%%%%%%%%%%%%%%%%%%%%%%%
\begin{theorem}[Matriz hermitiana definida positiva é não singular:]\label{theo:prophermitianpositivematrix2}
Dada uma matriz hermitiana $\MATRIX{A} \in \mathbb{C}^{N \times N}$.
\begin{itemize}
%%%%%
\item Se $\MATRIX{A}$ é definida positiva, então\footnote{A
demonstração pode ser vista na Prova \ref{proof:theo:prophermitianpositivematrix1:c}.}  
esta é não singular \cite[pp. 63]{ipsen2009numerical}.
\end{itemize}
\end{theorem}
%%%%%%%%%%%%%%%%%%%%%%%%%%%%%%%%%%%%%%%%%%%%%%%%%%%%%%%%%%%%%%%%%%%%%%%%%%%%%%%%
\begin{theorem}[Propriedades das matrizes hermitianas definidas positivas:]\label{theo:prophermitianpositivematrix0}
Dada uma matriz hermitiana $\MATRIX{A} \in \mathbb{C}^{N \times N}$.
\begin{itemize}
%%%%%
\item $\MATRIX{A}$ é definida positiva se e só se $\MATRIX{A}=\MATRIX{R}\MATRIX{R}^{\ast}$,
de modo que\footnote{A
demonstração pode ser vista na Prova \ref{proof:theo:prophermitianpositivematrix1:a}.}  
\begin{equation}
\VECTOR{x}^{\ast} \MATRIX{A} \VECTOR{x} > 0 \quad \iff \quad \MATRIX{A}=\MATRIX{R}\MATRIX{R}^{\ast},
\end{equation}
sendo que $\VECTOR{x} \in \mathbb{C}^{N}$ é não nulo e
$\MATRIX{R}$ é uma matriz não singular \cite[pp. 309]{hartfiel2000matrix}. %\cite[pp. 420]{korn2013mathematical}.
\end{itemize}
\end{theorem}



%%%%%%%%%%%%%%%%%%%%%%%%%%%%%%%%%%%%%%%%%%%%%%%%%%%%%%%%%%%%%%%%%%%%%%%%%%%%%%%%
\begin{theorem}[Matriz hermitiana 
$\MATRIX{B}^{\ast}\MATRIX{A}\MATRIX{B}$ definida positiva:]\label{theo:prophermitianpositivematrix1}
Dada uma matriz hermitiana $\MATRIX{A} \in \mathbb{C}^{N \times N}$.
\begin{itemize}
%%%%%
\item Se $\MATRIX{A}$ é definida positiva e 
$\MATRIX{B} \in \mathbb{C}^{N \times N}$ é não singular, então\footnote{A
demonstração pode ser vista na Prova \ref{proof:theo:prophermitianpositivematrix1:b}.} 
$\MATRIX{B}^{\ast}\MATRIX{A}\MATRIX{B}$ é também uma matriz definida positiva \cite[pp. 64]{ipsen2009numerical}.
\end{itemize}
\end{theorem}

%%%%%%%%%%%%%%%%%%%%%%%%%%%%%%%%%%%%%%%%%%%%%%%%%%%%%%%%%%%%%%%%%%%%%%%%%%%%%%%%
\begin{theorem}[Autovalores positivos e matrizes hermitianas definidas positivas:]\label{theo:hermitianpositivematrix1}
Conhecida uma matriz hermitiana $\MATRIX{A} \in \mathbb{C}^{N \times N}$, com  autovalores $\lambda_n$,
%e autovetores $\VECTOR{e}_n$, 
$\forall n \in \{1, 2, ..., N\}$.
\begin{itemize}
\item $\MATRIX{A}$ é uma matriz definida positiva se e só se\footnote{A
demonstração pode ser vista na Prova \ref{proof:theo:hermitianpositivematrix1}.}  
todos os autovalores são positivos \cite[pp. 309]{hartfiel2000matrix},
\begin{equation}
\VECTOR{x}^{\ast} \MATRIX{A} \VECTOR{x} > 0 \quad \iff \quad \lambda_n > 0,
\end{equation}
em que $\VECTOR{x} \in \mathbb{C}^{N}$ é não nulo.
\end{itemize}
\end{theorem}
