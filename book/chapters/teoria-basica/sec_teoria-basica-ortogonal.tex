%%%%%%%%%%%%%%%%%%%%%%%%%%%%%%%%%%%%%%%%%%%%%%%%%%%%%%%%%%%%%%%%%%%%%%%%%%%%%%%%%%%%%%%
%%%%%%%%%%%%%%%%%%%%%%%%%%%%%%%%%%%%%%%%%%%%%%%%%%%%%%%%%%%%%%%%%%%%%%%%%%%%%%%%%%%%%%%
%%%%%%%%%%%%%%%%%%%%%%%%%%%%%%%%%%%%%%%%%%%%%%%%%%%%%%%%%%%%%%%%%%%%%%%%%%%%%%%%%%%%%%%
\section{ Matrizes ortogonais}

\index{Matriz!Ortogonal}

\begin{definition}[Matriz ortogonal:]\label{def:ortogonalmatrix0}
Conhecida uma matriz quadrada $\MATRIX{A} \in \mathbb{R}^{N \times N}$. 
\begin{itemize}
\item Esta é uma matriz ortogonal quando se cumpre que \cite[pp. 66]{golub2013matrix}
\begin{equation}
\MATRIX{A}^{\transpose} = \MATRIX{A}^{-1}.
\end{equation}
\end{itemize}
\end{definition}

\begin{theorem}[Colunas da matriz ortogonal:]\label{theo:ortogonalmatrix0}
Conhecida uma matriz ortogonal $\MATRIX{Q} \in \mathbb{R}^{N \times N}$,
\begin{itemize}
\item A matriz $\MATRIX{Q}=\left[\VECTOR{q}_1\quad \VECTOR{q}_2\quad ... \quad \VECTOR{q}_N \right]$
está formada pela agrupação em colunas de um conjunto de $N$ vetores $\VECTOR{q}_n \in \mathbb{R}^{N}$
ortogonais \cite[pp. 66]{golub2013matrix}, em que 
\begin{equation}
\VECTOR{q}_i^{\transpose}\VECTOR{q}_j=
\left\{ 
\begin{matrix}
1 & se & i = j\\
0 & se & i \neq j
\end{matrix}
\right.
\end{equation} 
\end{itemize}
\end{theorem}

