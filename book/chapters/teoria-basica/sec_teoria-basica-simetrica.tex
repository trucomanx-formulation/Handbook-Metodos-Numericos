%%%%%%%%%%%%%%%%%%%%%%%%%%%%%%%%%%%%%%%%%%%%%%%%%%%%%%%%%%%%%%%%%%%%%%%%%%%%%%%%%%%%%%%
%%%%%%%%%%%%%%%%%%%%%%%%%%%%%%%%%%%%%%%%%%%%%%%%%%%%%%%%%%%%%%%%%%%%%%%%%%%%%%%%%%%%%%%
%%%%%%%%%%%%%%%%%%%%%%%%%%%%%%%%%%%%%%%%%%%%%%%%%%%%%%%%%%%%%%%%%%%%%%%%%%%%%%%%%%%%%%%
\section{ Matrizes simétricas}

\index{Matriz!Simétrica}

\begin{definition}[Matriz simétrica:]\label{def:symmetricmatrix0}
Conhecida uma matriz quadrada $\MATRIX{A} \in \mathbb{R}^{N \times N}$. 
\begin{itemize}
\item Esta é uma matriz simétrica, se se cumpre que \cite[pp. 18]{golub2013matrix} 
\begin{equation}
\MATRIX{A}^{\transpose} = \MATRIX{A}
\end{equation}
\end{itemize}
\end{definition}

%%%%%%%%%%%%%%%%%%%%%%%%%%%%%%%%%%%%%%%%%%%%%%%%%%%%%%%%%%%%%%%%%%%%%%%%%%%%%%%%
\begin{theorem}[Forma quadratica e matrizes simétricas:]\label{theo:simetricmatrix0}
Conhecida uma matriz quadrada $\MATRIX{A} \in \mathbb{R}^{N \times N}$.
\begin{itemize}
\item Se definimos a matriz simétrica $\MATRIX{S}=\frac{\MATRIX{A}+\MATRIX{A}^{\transpose}}{2}$ então\footnote{A
demostração pode ser vista na Prova \ref{proof:theo:simetricmatrix0}.} 
podemos afirmar que para qualquer vetor $\VECTOR{x} \in \mathbb{R}^{N}$ se cumpre que 
\begin{equation}
\VECTOR{x}^{\transpose}\MATRIX{S}\VECTOR{x} = 
\VECTOR{x}^{\transpose}\MATRIX{A}\VECTOR{x} =
\VECTOR{x}^{\transpose}\MATRIX{A}^{\transpose}\VECTOR{x}
\end{equation}
\end{itemize}
\end{theorem}


%%%%%%%%%%%%%%%%%%%%%%%%%%%%%%%%%%%%%%%%%%%%%%%%%%%%%%%%%%%%%%%%%%%%%%%%%%%%%%%%
\begin{theorem}[As matrizes simétricas são diagonalizaveis:]\label{theo:simetricmatrix2}
Conhecida uma matriz quadrada $\MATRIX{A} \in \mathbb{R}^{N \times N}$, 
com  autovalores $\lambda_n$, $\forall n \in \{1, 2, ..., N\}$.
%e autovetores $\VECTOR{e}_n$, $\forall n \in \{1, 2, ..., N\}$.
\begin{itemize}
\item Se $\MATRIX{A}$ é uma matriz simétrica então
%\footnote{A demostração pode ser vista na Prova \ref{proof:theo:simetricmatrix2}.} 
existe uma matriz $\MATRIX{Q}$ ortogonal que cumpre \cite[pp. 67,440]{golub2013matrix}
\begin{equation}
\begin{bmatrix}
\lambda_1 & 0         & ...    & 0 \\
0         & \lambda_2 & ...    & 0 \\
\vdots    & \vdots    & \ddots & \vdots \\
0         & 0         & ...    & \lambda_N
\end{bmatrix}
=\MATRIX{Q}^{-1}\MATRIX{A}\MATRIX{Q}.
\end{equation}
\end{itemize}
\end{theorem}

%%%%%%%%%%%%%%%%%%%%%%%%%%%%%%%%%%%%%%%%%%%%%%%%%%%%%%%%%%%%%%%%%%%%%%%%%%%%%%%%
\begin{theorem}[Autovalor de maior valor em matrizes simétricas:]\label{theo:simetricmatrix3}
Conhecida uma matriz quadrada $\MATRIX{A} \in \mathbb{R}^{N \times N}$, 
com  autovalores $\lambda_n$, $\forall n \in \{1, 2, ..., N\}$.
%e autovetores $\VECTOR{e}_n$, $\forall n \in \{1, 2, ..., N\}$.
\begin{itemize}
\item Se $\MATRIX{A}$ é uma matriz simétrica então, $\lambda_{min}(\MATRIX{A})$ e $\lambda_{max}(\MATRIX{A})$,
o autovalor de menor e maior valor da  matriz $\MATRIX{A}$, respetivamente, cumpre \cite[pp. 67]{golub2013matrix}
\begin{equation}
\lambda_{min}(\MATRIX{A}) = \min_{\VECTOR{x}\neq 0} \frac{\VECTOR{x}^{\transpose}\MATRIX{A}\VECTOR{x}}{\VECTOR{x}^{\transpose}\VECTOR{x}},
\qquad
\lambda_{max}(\MATRIX{A}) = \max_{\VECTOR{x}\neq 0} \frac{\VECTOR{x}^{\transpose}\MATRIX{A}\VECTOR{x}}{\VECTOR{x}^{\transpose}\VECTOR{x}},
\end{equation}
\end{itemize}
onde $\VECTOR{x} \in \mathbb{R}^N$.
\end{theorem}

%%%%%%%%%%%%%%%%%%%%%%%%%%%%%%%%%%%%%%%%%%%%%%%%%%%%%%%%%%%%%%%%%%%%%%%%%%%%%%%%
\begin{theorem}[Autovalor de maior valor em matrizes simétricas:]\label{theo:simetricmatrix4}
Conhecida uma matriz quadrada $\MATRIX{A} \in \mathbb{R}^{N \times N}$, 
%com  autovalores $\lambda_n$, $\forall n \in \{1, 2, ..., N\}$.
%e autovetores $\VECTOR{e}_n$, $\forall n \in \{1, 2, ..., N\}$.
\begin{itemize}
\item Se $\MATRIX{A}$ é uma matriz simétrica, então
\footnote{A demostração pode ser vista na Prova \ref{proof:theo:simetricmatrix4}.} 
 a matriz $\MATRIX{A}^{-1}$  é simétrica também,
\begin{equation}
\MATRIX{A}=\MATRIX{A}^{\transpose}
\qquad \rightarrow \qquad
\MATRIX{A}^{-1}=\left\{\MATRIX{A}^{-1}\right\}^{\transpose}
\end{equation}
\end{itemize}
\end{theorem}

%%%%%%%%%%%%%%%%%%%%%%%%%%%%%%%%%%%%%%%%%%%%%%%%%%%%%%%%%%%%%%%%%%%%%%%%%%%%%%%%
\begin{lema}[Matriz transposta e matrizes simétricas:]\label{lema:simetricmatrix5}
Conhecida uma matriz $\MATRIX{B} \in \mathbb{R}^{N \times M}$.
\begin{itemize}
\item A matriz $\MATRIX{B}^{\transpose}\MATRIX{B} \in \mathbb{R}^{M \times M}$ é uma matriz simétrica.
\end{itemize}
\end{lema}


