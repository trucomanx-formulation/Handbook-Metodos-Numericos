%%%%%%%%%%%%%%%%%%%%%%%%%%%%%%%%%%%%%%%%%%%%%%%%%%%%%%%%%%%%%%%%%%%%%%%%%%%%%%%%%%%%%%%
%%%%%%%%%%%%%%%%%%%%%%%%%%%%%%%%%%%%%%%%%%%%%%%%%%%%%%%%%%%%%%%%%%%%%%%%%%%%%%%%%%%%%%%
%%%%%%%%%%%%%%%%%%%%%%%%%%%%%%%%%%%%%%%%%%%%%%%%%%%%%%%%%%%%%%%%%%%%%%%%%%%%%%%%%%%%%%%
\section{Propriedades varias com matrizes}


\begin{theorem}[Propiedades das matrices inversas:]\label{theo:matrixgeneric2}
Conhecidas as matrizes quadradas $\MATRIX{A} \in \mathbb{R}^{N \times N}$ e $\MATRIX{B} \in \mathbb{R}^{N \times N}$,
podemos afirmar \cite[pp. 65]{golub2013matrix},
\begin{itemize}
\item  A inversa da multiplicação de duas matrizes,
\begin{equation}
\MATRIX{A}^{-1} \MATRIX{B}^{-1} = \left\{\MATRIX{B}\MATRIX{A}\right\}^{-1}.
\end{equation}
\item A inversa da matriz transposta,
\begin{equation}
\left\{\MATRIX{A}^{-1}\right\}^{\transpose}  = \left\{\MATRIX{A}^{\transpose}\right\}^{-1}.
\end{equation}
\end{itemize}
\end{theorem}

\begin{theorem}[Propiedades com determinantes:]\label{theo:matrixgeneric1}
Conhecidas as matrizes quadradas $\MATRIX{A} \in \mathbb{R}^{N \times N}$ e $\MATRIX{B} \in \mathbb{R}^{N \times N}$,
e o escalar $\alpha \in \mathbb{R}$, podemos afirmar \cite[pp. 66]{golub2013matrix},
\begin{itemize}
\item Determinante da matriz transposta,
\begin{equation}
det(\MATRIX{A})=det(\MATRIX{A}^{\transpose}).
\end{equation}
\item Determinante da multiplicação de uma matriz e um escalar,
\begin{equation}
det(\alpha \MATRIX{A})=\alpha^N~det(\MATRIX{A}).
\end{equation}
\item Determinante da multiplicação de duas matrizes,
\begin{equation}
det(\MATRIX{A}\MATRIX{B})=det(\MATRIX{A})det(\MATRIX{B}).
\end{equation}
\end{itemize}
\end{theorem}


\begin{definition}[Autovalor e autovetores de uma matriz:]\label{def:matrixgeneric0}
Seja a matriz $\MATRIX{A}$ um operador linear $\MATRIX{A}: \mathbb{C}^{N} \rightarrow \mathbb{C}^{N}$,  
e $\VECTOR{v} \in \mathbb{C}^{N} \neq\VECTOR{0}$. Se existe um escalar $\lambda \in \mathbb{C}$ onde, 
%\begin{equation}
$\MATRIX{A}\VECTOR{v}=\lambda\VECTOR{v}$,
%\end{equation} 
então $\VECTOR{v}$ é um autovetor de $\MATRIX{A}$, e $\lambda$ é seu autovalor asociado.
\end{definition}


\begin{theorem}[Autovalor  na matriz transposta:]\label{theo:matrixgeneric3}
Conhecida uma matriz quadrada $\MATRIX{A} \in \mathbb{R}^{N \times N}$, 
com  autovalores $\lambda_n$, $\forall n \in \{1, 2, ..., N\}$.
%e autovetores $\VECTOR{e}_n$, $\forall n \in \{1, 2, ..., N\}$.
\begin{itemize}
\item A matriz $\MATRIX{A}^{\transpose}$ tem
\footnote{A demostração pode ser vista na Prova \ref{proof:theo:matrixgeneric3}.} 
os mesmos autovalores que $\MATRIX{A}$, é dizer tem o mesmo polinômio caraterístico $p(\lambda)$,
\begin{equation}
p(\lambda)=det(\MATRIX{A}-\lambda \MATRIX{I})=det(\MATRIX{A}^{\transpose}-\lambda \MATRIX{I}).
\end{equation}
\end{itemize}
\end{theorem}
