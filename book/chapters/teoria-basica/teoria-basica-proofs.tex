\section{Provas dos teoremas}

%%%%%%%%%%%%%%%%%%%%%%%%%%%%%%%%%%%%%%%%%%%%%%%%%%%%%%%%%%%%%%%%%%%%%%%%%%%%%%%%%%%%%%%
%%%%%%%%%%%%%%%%%%%%%%%%%%%%%%%%%%%%%%%%%%%%%%%%%%%%%%%%%%%%%%%%%%%%%%%%%%%%%%%%%%%%%%%
\begin{myproofT}[Relativa ao Teorema \ref{theo:simetricmatrix0}:]\label{proof:theo:simetricmatrix0}
Conhecida uma matriz quadrada $\MATRIX{A} \in \mathbb{R}^{N \times N}$,
se definimos a matriz simétrica $\MATRIX{S}=\frac{\MATRIX{A}+\MATRIX{A}^{\transpose}}{2}$ e
o vetor $\VECTOR{x} \in \mathbb{R}^{N}$, então
\begin{equation}
\VECTOR{x}^{\transpose}\MATRIX{S}\VECTOR{x} = 
\frac{\VECTOR{x}^{\transpose}\MATRIX{A}\VECTOR{x}}{2} +
\frac{\VECTOR{x}^{\transpose}\MATRIX{A}^{\transpose}\VECTOR{x}}{2},
\end{equation}
dado que cada somando da equação anterior é um escalar podemos afirmar que 
\begin{equation}
\VECTOR{x}^{\transpose}\MATRIX{S}\VECTOR{x} = 
\frac{\VECTOR{x}^{\transpose}\MATRIX{A}\VECTOR{x}}{2} +
\left[\frac{\VECTOR{x}^{\transpose}\MATRIX{A}^{\transpose}\VECTOR{x}}{2}\right]^{\transpose}
\end{equation}
e consequentemente que
\begin{equation}
\VECTOR{x}^{\transpose}\MATRIX{S}\VECTOR{x} = 
\VECTOR{x}^{\transpose}\MATRIX{A}\VECTOR{x} =
\VECTOR{x}^{\transpose}\MATRIX{A}^{\transpose}\VECTOR{x}.
\end{equation} 
\end{myproofT}

%%%%%%%%%%%%%%%%%%%%%%%%%%%%%%%%%%%%%%%%%%%%%%%%%%%%%%%%%%%%%%%%%%%%%%%%%%%%%%%%%%%%%%%
%%%%%%%%%%%%%%%%%%%%%%%%%%%%%%%%%%%%%%%%%%%%%%%%%%%%%%%%%%%%%%%%%%%%%%%%%%%%%%%%%%%%%%%
\begin{myproofT}[Relativa ao Teorema \ref{theo:matrixgeneric3}:]\label{proof:theo:matrixgeneric3}
Conhecida uma matriz quadrada $\MATRIX{A} \in \mathbb{R}^{N \times N}$, 
então os  autovalores $\lambda_n$, $\forall n \in \{1, 2, ..., N\}$ são calculados mediante 
seu polinômio caraterístico,
\begin{equation}
p(\lambda)=det(\MATRIX{A}-\lambda \MATRIX{I}),
\end{equation}
onde $\MATRIX{I}$ é uma matriz identidade,
\begin{equation}
p(\lambda)^{\transpose}=det(\MATRIX{A}-\lambda \MATRIX{I})^{\transpose}
\quad \rightarrow \quad
p(\lambda)=det(\MATRIX{A}^{\transpose}-\lambda \MATRIX{I}^{\transpose}),
\end{equation}
\begin{equation}
p(\lambda)=det(\MATRIX{A}^{\transpose}-\lambda \MATRIX{I}).
\end{equation}
\end{myproofT}

%%%%%%%%%%%%%%%%%%%%%%%%%%%%%%%%%%%%%%%%%%%%%%%%%%%%%%%%%%%%%%%%%%%%%%%%%%%%%%%%%%%%%%%
%%%%%%%%%%%%%%%%%%%%%%%%%%%%%%%%%%%%%%%%%%%%%%%%%%%%%%%%%%%%%%%%%%%%%%%%%%%%%%%%%%%%%%%
\begin{myproofT}[Relativa ao Teorema \ref{theo:simetricmatrix4}:]\label{proof:theo:simetricmatrix4}
Conhecida uma matriz quadrada $\MATRIX{A} \in \mathbb{R}^{N \times N}$ simétrica e não singular, 
\begin{equation}
\MATRIX{I}=\MATRIX{A}\MATRIX{A}^{-1}
\quad \rightarrow \quad
\MATRIX{I}^{\transpose}=\left\{\MATRIX{A}\MATRIX{A}^{-1}\right\}^{\transpose}
\quad \rightarrow \quad
\MATRIX{I}=\left\{\MATRIX{A}^{-1}\right\}^{\transpose} \MATRIX{A}^{\transpose}
\end{equation}
\begin{equation}
\MATRIX{I}=\left\{\MATRIX{A}^{-1}\right\}^{\transpose} \MATRIX{A}
\quad \rightarrow \quad
\MATRIX{A}^{-1}=\left\{\MATRIX{A}^{-1}\right\}^{\transpose}.
\end{equation}
\end{myproofT}


%%%%%%%%%%%%%%%%%%%%%%%%%%%%%%%%%%%%%%%%%%%%%%%%%%%%%%%%%%%%%%%%%%%%%%%%%%%%%%%%%%%%%%%
%%%%%%%%%%%%%%%%%%%%%%%%%%%%%%%%%%%%%%%%%%%%%%%%%%%%%%%%%%%%%%%%%%%%%%%%%%%%%%%%%%%%%%%
\begin{myproofT}[Relativa ao Teorema \ref{theo:semipositivematrix1}:]\label{proof:theo:semipositivematrix1}
Conhecida uma matriz $\MATRIX{A} \in \mathbb{R}^{N \times M}$,
e um vetor $\VECTOR{x} \in \mathbb{R}^{N}$ não nulo,
\begin{equation}
\VECTOR{x}^{\transpose}\MATRIX{A}\MATRIX{A}^{\transpose}\VECTOR{x}
\quad \rightarrow \quad
\left\{\MATRIX{A}^{\transpose}\VECTOR{x}\right\}\left\{\MATRIX{A}^{\transpose}\VECTOR{x}\right\},
\end{equation}
se $\VECTOR{y}=\MATRIX{A}^{\transpose}\VECTOR{x}$, 
podemos perceber que existe a possibilidade de que $\VECTOR{y}=\VECTOR{0}$,
pelo que podemos afirmar que 
\begin{equation}
\VECTOR{y}^{\transpose}\VECTOR{y}\geq 0,
\quad \rightarrow \quad
\VECTOR{x}^{\transpose}\MATRIX{A}\MATRIX{A}^{\transpose}\VECTOR{x}\geq 0.
\end{equation}
\end{myproofT}

%%%%%%%%%%%%%%%%%%%%%%%%%%%%%%%%%%%%%%%%%%%%%%%%%%%%%%%%%%%%%%%%%%%%%%%%%%%%%%%%%%%%%%%
%%%%%%%%%%%%%%%%%%%%%%%%%%%%%%%%%%%%%%%%%%%%%%%%%%%%%%%%%%%%%%%%%%%%%%%%%%%%%%%%%%%%%%%
\begin{myproofT}[Relativa ao Teorema \ref{theo:semipositivematrix1}:]\label{proof:theo:semipositivematrix2}
Conhecida uma matriz $\MATRIX{A} \in \mathbb{R}^{N \times N}$, com  autovalores $\lambda_n$,
e autovetores $\VECTOR{v}_n$, $\forall n \in \{1, 2, ..., N\}$,
\begin{equation}
\MATRIX{A}\VECTOR{v}_n = \lambda_n \VECTOR{v}_n, 
\end{equation}
\begin{equation}
\VECTOR{v}_n^{\transpose}\MATRIX{A}\VECTOR{v}_n = \lambda_n \VECTOR{v}_n^{\transpose}\VECTOR{v}_n;
\end{equation}
sabendo que a matriz é semidefinida positiva,
\begin{equation}
0 \leq \VECTOR{v}^{\transpose}\MATRIX{A}\VECTOR{v} = \lambda_n \VECTOR{v}_n^{\transpose}\VECTOR{v}_n,
\end{equation}
\begin{equation}
\lambda_n \VECTOR{v}_n^{\transpose}\VECTOR{v}_n \geq 0,
\end{equation}
dado que sempre $\VECTOR{v}_n^{\transpose}\VECTOR{v}_n> 0$, concluímos
\begin{equation}
\lambda_n  \geq 0.
\end{equation} 
\end{myproofT}

%%%%%%%%%%%%%%%%%%%%%%%%%%%%%%%%%%%%%%%%%%%%%%%%%%%%%%%%%%%%%%%%%%%%%%%%%%%%%%%%%%%%%%%
%%%%%%%%%%%%%%%%%%%%%%%%%%%%%%%%%%%%%%%%%%%%%%%%%%%%%%%%%%%%%%%%%%%%%%%%%%%%%%%%%%%%%%%
\begin{myproofT}[Relativa ao Teorema \ref{theo:positivematrix1}:]\label{proof:theo:positivematrix1}
Conhecida uma matriz quadrada $\MATRIX{A} \in \mathbb{R}^{N \times N}$ definida positiva, com  autovalores $\lambda_n$,
e autovetores $\VECTOR{e}_n$, $\forall n \in \{1, 2, ..., N\}$, de modo que,
\begin{equation}\label{eq:proof:theo:positivematrix1:0}
\MATRIX{A}\VECTOR{e}_n=\lambda_n \VECTOR{e}_n.
\end{equation}
\begin{equation}\label{eq:proof:theo:positivematrix1:1}
\VECTOR{e}_n^{\transpose}\MATRIX{A}\VECTOR{e}_n=\lambda_n \VECTOR{e}_n^{\transpose}\VECTOR{e}_n.
\end{equation}

\begin{itemize}
\item Usando a Eq. (\ref{eq:proof:theo:positivematrix1:1}) e a Definição \ref{def:positivematrix0} podemos afirmar que
\begin{equation}
\VECTOR{e}_n^{\transpose}\MATRIX{A}\VECTOR{e}_n >0
\end{equation} 
o que implica que
\begin{equation}
\lambda_n \VECTOR{e}_n^{\transpose}\VECTOR{e}_n > 0
\quad \rightarrow \quad
\lambda_n  \geq 0
\end{equation} 
\end{itemize}
\end{myproofT}


%%%%%%%%%%%%%%%%%%%%%%%%%%%%%%%%%%%%%%%%%%%%%%%%%%%%%%%%%%%%%%%%%%%%%%%%%%%%%%%%%%%%%%%
%%%%%%%%%%%%%%%%%%%%%%%%%%%%%%%%%%%%%%%%%%%%%%%%%%%%%%%%%%%%%%%%%%%%%%%%%%%%%%%%%%%%%%%
\begin{myproofT}[Relativa ao Teorema \ref{theo:positivematrix1}:]\label{proof:theo:positivematrix2}
Conhecida uma matriz quadrada $\MATRIX{A} \in \mathbb{R}^{N \times N}$ simétrica,
com  autovalores $\lambda_n>0$, $\forall n \in \{1, 2, ..., N\}$;
sabemos pelo Teorema \ref{theo:simetricmatrix2}, 
que existe uma \hyperref[def:ortogonalmatrix0]{\textbf{matriz ortogonal}} $\MATRIX{Q}$,
\begin{equation}
\MATRIX{Q}^{\transpose}\MATRIX{A}\MATRIX{Q} = \MATRIX{\lambda}_{\MATRIX{A}}\equiv
\begin{bmatrix}
\lambda_1 & 0         & ...    & 0 \\
0         & \lambda_2 & ...    & 0 \\
\vdots    & \vdots    & \ddots & \vdots \\
0         & 0         & ...    & \lambda_N
\end{bmatrix}.
\end{equation}
Se multiplicamos esta igualdade por um vector $\VECTOR{x} \in \mathbb{R}^N$ (não nulo), obtemos,
\begin{equation}
\VECTOR{x}^{\transpose} \MATRIX{Q}^{\transpose}\MATRIX{A}\MATRIX{Q} \VECTOR{x} = 
\VECTOR{x}^{\transpose} \MATRIX{\lambda}_{\MATRIX{A}} \VECTOR{x},
\end{equation}
\begin{equation}
\VECTOR{x}^{\transpose} \MATRIX{Q}^{\transpose}\MATRIX{A}\MATRIX{Q} \VECTOR{x} 
= \sum \limits_{n=1}^{N}\lambda_n x_n^2.
\end{equation}
Aplicando uma troca de variáveis de modo que $\VECTOR{y}=\MATRIX{Q} \VECTOR{x}$,
\begin{equation}
\VECTOR{y}^{\transpose} \MATRIX{A} \VECTOR{y} 
= \sum \limits_{n=1}^{N}\lambda_n x_n^2.
\end{equation}
Dado que a soma de todos os $\lambda_n x_n^2$ é positiva, podemos afirmar que
\begin{equation}
\VECTOR{y}^{\transpose} \MATRIX{A} \VECTOR{y} > 0;
\end{equation}
é dizer $\MATRIX{A}$ é definida positiva.
\end{myproofT}

%%%%%%%%%%%%%%%%%%%%%%%%%%%%%%%%%%%%%%%%%%%%%%%%%%%%%%%%%%%%%%%%%%%%%%%%%%%%%%%%%%%%%%%
%%%%%%%%%%%%%%%%%%%%%%%%%%%%%%%%%%%%%%%%%%%%%%%%%%%%%%%%%%%%%%%%%%%%%%%%%%%%%%%%%%%%%%%
\begin{myproofT}[Relativa ao Teorema \ref{theo:positivematrix:2}:]\label{proof:theo:positivematrix:2}
Conhecida uma matriz definida positiva e simétrica $\MATRIX{A} \in \mathbb{R}^{N \times N}$,
com  autovalores $\lambda_n>0$, $\forall n \in \{1, 2, ..., N\}$;
se $\MATRIX{A}$  é invertível então os autovalores de $\MATRIX{A}^{-1}$ são  $\frac{1}{\lambda_n}>0$.
Assim, se $\MATRIX{A}$ é simétrica, $\MATRIX{A}^{-1}$ é simétrica também;
então, se $\MATRIX{A}^{-1}$ é simétrica com autovalores positivos, esta é definida positiva.
\end{myproofT}

%%%%%%%%%%%%%%%%%%%%%%%%%%%%%%%%%%%%%%%%%%%%%%%%%%%%%%%%%%%%%%%%%%%%%%%%%%%%%%%%%%%%%%%
%%%%%%%%%%%%%%%%%%%%%%%%%%%%%%%%%%%%%%%%%%%%%%%%%%%%%%%%%%%%%%%%%%%%%%%%%%%%%%%%%%%%%%%
\begin{myproofT}[Relativa ao Teorema \ref{theo:similhante1}:]\label{proof:theo:similhante1}
Conhecidas as matrizes semelhantes $\MATRIX{A} \in \mathbb{C}^{N \times N}$ e $\MATRIX{B} \in \mathbb{R}^{N \times N}$,
de modo que estas cumprem 
\begin{equation}
\MATRIX{A} = \MATRIX{P}^{-1} \MATRIX{B} \MATRIX{P},
\end{equation}
para um $\MATRIX{P}$ dado.

Se calculamos o polinômio caraterístico $p_{\MATRIX{A}}(\lambda)$ de $\MATRIX{A}$,
\begin{equation}
p_{\MATRIX{A}}(\lambda)=det\left(A-\lambda \MATRIX{I}\right),
\end{equation}
podemos rescrever esta equação
\begin{equation}
p_{\MATRIX{A}}(\lambda)=det\left(\MATRIX{P}^{-1} \MATRIX{B} \MATRIX{P}-\lambda \MATRIX{P}^{-1} \MATRIX{I} \MATRIX{P} \right),
\end{equation}
\begin{equation}
p_{\MATRIX{A}}(\lambda)=det\left( \MATRIX{P}^{-1}\right)det\left( \MATRIX{B} -\lambda \MATRIX{I} \right)det\left( \MATRIX{P}\right),
\end{equation}
\begin{equation}
p_{\MATRIX{A}}(\lambda)=det\left( \MATRIX{B} -\lambda \MATRIX{I} \right) = p_{\MATRIX{B}}(\lambda).
\end{equation}
\end{myproofT}

%%%%%%%%%%%%%%%%%%%%%%%%%%%%%%%%%%%%%%%%%%%%%%%%%%%%%%%%%%%%%%%%%%%%%%%%%%%%%%%%
%%%%%%%%%%%%%%%%%%%%%%%%%%%%%%%%%%%%%%%%%%%%%%%%%%%%%%%%%%%%%%%%%%%%%%%%%%%%%%%%
\begin{myproofT}[Relativa ao Teorema \ref{theo:diagonalization1}:]\label{proof:theo:diagonalization1:a}
Conhecida uma matriz quadrada $\MATRIX{A} \in \mathbb{C}^{N \times N}$ com
autovalores $\lambda_n$, $\forall n \in \{1, 2, ..., N\}$.
Se $\MATRIX{A}$ tem $N$  autovetores (coluna) $\VECTOR{v}_n \in \mathbb{C}^{N}$  linearmente independentes,
podemos afirmar que
\begin{equation}
\begin{bmatrix}
\MATRIX{A} \VECTOR{v}_1 & \MATRIX{A} \VECTOR{v}_2 & ... & \MATRIX{A} \VECTOR{v}_N
\end{bmatrix}=
\begin{bmatrix}
\lambda_1 \VECTOR{v}_1 & \lambda_2 \VECTOR{v}_2 & ... & \lambda_N \VECTOR{v}_N
\end{bmatrix}
\end{equation}
\begin{equation}
\MATRIX{A}
\begin{bmatrix}
 \VECTOR{v}_1 &  \VECTOR{v}_2 & ... &  \VECTOR{v}_N
\end{bmatrix}=
\begin{bmatrix}
 \VECTOR{v}_1 & \VECTOR{v}_2 & ... &  \VECTOR{v}_N
\end{bmatrix}
\funcdiag\left(
\begin{bmatrix}
\lambda_1 & \lambda_2 & \dots & \lambda_N
\end{bmatrix}^{\transpose}\right).
\end{equation}
Se definimos $\MATRIX{V}=\left[\VECTOR{v}_1\quad \VECTOR{v}_2\quad \VECTOR{v}_3\quad ...\quad \VECTOR{v}_N\right]$,
e $\lambda_{\MATRIX{A}}=\funcdiag\left(\left[\lambda_1\quad \lambda_2\quad \lambda_3\quad ...\quad \lambda_N\right]^{\transpose}\right)$
então
\begin{equation}
\MATRIX{A} \MATRIX{V} = \MATRIX{V} \lambda_{\MATRIX{A}},
\quad \rightarrow \quad
\MATRIX{V}^{-1} \MATRIX{A} \MATRIX{V} =  \lambda_{\MATRIX{A}}.
\end{equation}
\end{myproofT}
%%%%%%%%%%%%%%%%%%%%%%%%%%%%%%%%%%%%%%%%%%%%%%%%%%%%%%%%%%%%%%%%%%%%%%%%%%%%%%%%
%%%%%%%%%%%%%%%%%%%%%%%%%%%%%%%%%%%%%%%%%%%%%%%%%%%%%%%%%%%%%%%%%%%%%%%%%%%%%%%%
\begin{myproofT}[Relativa ao Teorema \ref{theo:autovalorhermitianmatrix0}:]\label{proof:theo:autovalorhermitianmatrix0}
Dada uma matriz hermitiana $\MATRIX{A} \in \mathbb{C}^{N \times N}$, com  autovalores $\lambda_n$,
e autovetores $\VECTOR{v}_n$, $\forall n \in \{1, 2, ..., N\}$, podemos afirmar que,
\begin{equation}
\MATRIX{A}\VECTOR{v}_n=\lambda_n \VECTOR{v}_n
\end{equation}
\begin{equation}
\VECTOR{v}_n^{\ast}\MATRIX{A}\VECTOR{v}_n=\lambda_n \VECTOR{v}_n^{\ast} \VECTOR{v}_n=\lambda_n \NORM{\VECTOR{v}_n}^2,
\end{equation}
aplicando o operador transposto conjugado obtemos
\begin{equation}
\VECTOR{v}_n^{\ast}\MATRIX{A}^{\ast}\VECTOR{v}_n=\overline{\lambda}_n \NORM{\VECTOR{v}_n}^2,
\end{equation}
\begin{equation}
\VECTOR{v}_n^{\ast}\MATRIX{A}\VECTOR{v}_n=\overline{\lambda}_n \NORM{\VECTOR{v}_n}^2,
\end{equation}
o que indica que $\overline{\lambda}_n=\lambda_n$ e consequentemente $\lambda_n$ é real.
\end{myproofT}

%%%%%%%%%%%%%%%%%%%%%%%%%%%%%%%%%%%%%%%%%%%%%%%%%%%%%%%%%%%%%%%%%%%%%%%%%%%%%%%%
%%%%%%%%%%%%%%%%%%%%%%%%%%%%%%%%%%%%%%%%%%%%%%%%%%%%%%%%%%%%%%%%%%%%%%%%%%%%%%%%
\begin{myproofT}[Relativa ao Teorema \ref{theo:autovetorortogonalhermitian0}:]\label{proof:theo:autovetorortogonalhermitian0}
Dada uma matriz hermitiana $\MATRIX{A} \in \mathbb{C}^{N \times N}$, com  autovalores $\lambda_n$,
e autovetores $\VECTOR{v}_n$, $\forall n \in \{1, 2, ..., N\}$, podemos afirmar que,
\begin{equation}
\MATRIX{A} \VECTOR{v}_a= \lambda_a \VECTOR{v}_a 
\quad \wedge \quad 
\MATRIX{A} \VECTOR{v}_b= \lambda_b \VECTOR{v}_b
\quad \wedge \quad 
\lambda_a \neq \lambda_b, 
\end{equation}
\begin{equation}
\begin{matrix}
\lambda_a \innerprod{\VECTOR{v}_a}{\VECTOR{v}_b} & = & \lambda_a \VECTOR{v}_a^{\transpose} \overline{\VECTOR{v}}_b\\
~ & = & \left( \MATRIX{A}\VECTOR{v}_a\right)^{\transpose} \overline{\VECTOR{v}}_b \\
~ & = & \VECTOR{v}_a^{\transpose} \MATRIX{A}^{\transpose} \overline{\VECTOR{v}}_b \\
~ & = & \VECTOR{v}_a^{\transpose} \overline{\MATRIX{A}} \overline{\VECTOR{v}}_b \\
~ & = & \VECTOR{v}_a^{\transpose} \overline{\lambda_b} \overline{\VECTOR{v}}_b,
\end{matrix}
\end{equation}
lembrando que pelo Teorema \ref{theo:autovalorhermitianmatrix0} todos os autovalores são reais 
numa matriz hermitiana,
\begin{equation}
\lambda_a \innerprod{\VECTOR{v}_a}{\VECTOR{v}_b} = \lambda_b \innerprod{\VECTOR{v}_a}{\VECTOR{v}_b}.
\end{equation}
Dado que se partiu de assumir que $\lambda_a \neq \lambda_b$, 
então se conclui que $\innerprod{\VECTOR{v}_a}{\VECTOR{v}_b}=0$.
\end{myproofT}

%%%%%%%%%%%%%%%%%%%%%%%%%%%%%%%%%%%%%%%%%%%%%%%%%%%%%%%%%%%%%%%%%%%%%%%%%%%%%%%%
%%%%%%%%%%%%%%%%%%%%%%%%%%%%%%%%%%%%%%%%%%%%%%%%%%%%%%%%%%%%%%%%%%%%%%%%%%%%%%%%
\begin{myproofT}[Relativa ao Teorema \ref{theo:HermitianDDecomposition0}:]\label{proof:theo:HermitianDDecomposition0}
Dada uma matriz $\MATRIX{A} \in \mathbb{C}^{N \times N}$,
sabemos pelo Teorema \ref{theo:ShurDecomposition0} (decomposição de Schur)
\begin{equation}
\MATRIX{U}^{-1} \MATRIX{A} \MATRIX{U} \equiv
\MATRIX{U}^{\ast} \MATRIX{A} \MATRIX{U} =
\begin{bmatrix}
\lambda_1 & b_{12}    & ...    & b_{1N} \\
0         & \lambda_2 & ...    & b_{2N} \\
\vdots    & \vdots    & \ddots & \vdots \\
0         & 0         & ...    & \lambda_N
\end{bmatrix}.
\end{equation}
Se ademais conhecemos que a matriz $\MATRIX{A}$ é hermitiana  
e calculamos $\left(\MATRIX{U}^{\ast} \MATRIX{A} \MATRIX{U}\right)^{\ast}$,
\begin{equation}
\left(\MATRIX{U}^{\ast} \MATRIX{A} \MATRIX{U}\right)^{\ast} =
\MATRIX{U}^{\ast} \MATRIX{A}^{\ast} \MATRIX{U}  =
\MATRIX{U}^{\ast} \MATRIX{A} \MATRIX{U} =
\begin{bmatrix}
\overline{\lambda}_1 & 0                    & ...    & 0 \\
\overline{b}_{12}    & \overline{\lambda}_2 & ...    & 0 \\
\vdots               & \vdots               & \ddots & \vdots \\
\overline{b}_{1N}    & \overline{b}_{2N}    & ...    & \overline{\lambda}_N
\end{bmatrix}.
\end{equation}
Disto se deduz que $\overline{\lambda}_n=\lambda_n$; é dizer $\lambda_n$ é real,
e que  $b_{ij}=0, \forall j>i$; é dizer $\MATRIX{U}^{\ast} \MATRIX{A} \MATRIX{U}$
é uma matriz diagonal.
\end{myproofT}


%%%%%%%%%%%%%%%%%%%%%%%%%%%%%%%%%%%%%%%%%%%%%%%%%%%%%%%%%%%%%%%%%%%%%%%%%%%%%%%%
%%%%%%%%%%%%%%%%%%%%%%%%%%%%%%%%%%%%%%%%%%%%%%%%%%%%%%%%%%%%%%%%%%%%%%%%%%%%%%%%
\begin{myproofT}[Relativa ao Teorema \ref{theo:prophermitianpositivematrix2}:]\label{proof:theo:prophermitianpositivematrix1:c}
Dada uma matriz hermitiana $\MATRIX{A} \in \mathbb{C}^{N \times N}$ definida positiva;
se imaginamos a hipótese que $\MATRIX{A}$ é singular; é dizer não invertível,
podemos afirmar que para algum vetor $\VECTOR{x}$ não nulo
é possível obter
\begin{equation}\label{eq:proof:theo:prophermitianpositivematrix1:c:1}
\MATRIX{A}\VECTOR{x}= 0 
\quad \rightarrow \quad
\VECTOR{x}^{\ast} \MATRIX{A} \VECTOR{x}= 0.
\end{equation}
Dado que $\MATRIX{A}$ é hermitiana definida positiva, 
o que implica obrigatoriamente que $\VECTOR{x}^{\ast} \MATRIX{A} \VECTOR{x}>0$,
podemos afirmar que a hipótese da Eq. (\ref{eq:proof:theo:prophermitianpositivematrix1:c:1}) é falsa, 
de modo que a matriz $\MATRIX{A}$ é não singular (invertível).
\end{myproofT}

%%%%%%%%%%%%%%%%%%%%%%%%%%%%%%%%%%%%%%%%%%%%%%%%%%%%%%%%%%%%%%%%%%%%%%%%%%%%%%%%
%%%%%%%%%%%%%%%%%%%%%%%%%%%%%%%%%%%%%%%%%%%%%%%%%%%%%%%%%%%%%%%%%%%%%%%%%%%%%%%%
\begin{myproofT}[Relativa ao Teorema \ref{theo:prophermitianpositivematrix0}:]\label{proof:theo:prophermitianpositivematrix1:a}
Dado um vetor $\VECTOR{x} \in \mathbb{C}^{N}$ não nulo
e uma matriz $\MATRIX{A} \in \mathbb{C}^{N \times N}$. 
\begin{itemize}
\item Se
 $\MATRIX{A}=\MATRIX{R}\MATRIX{R}^{\ast}$, 
onde $\MATRIX{R}$ é uma matriz invertível (não singular), então
\begin{equation}
\VECTOR{x}^{\ast} \MATRIX{A} \VECTOR{x} =
\VECTOR{x}^{\ast} \MATRIX{R}\MATRIX{R}^{\ast} \VECTOR{x} =
\left(\MATRIX{R}^{\ast} \VECTOR{x}\right)^{\ast} \MATRIX{R}^{\ast} \VECTOR{x}=
\NORM{\MATRIX{R}^{\ast} \VECTOR{x}}^2.
\end{equation}
Dado que $\MATRIX{R}^{\ast} \VECTOR{x}\neq 0$ e $\VECTOR{x}^{\ast}\MATRIX{R} \neq 0$ 
para todo vetor $\VECTOR{x}$ não nulo e $\MATRIX{R}$ não singular,
podemos afirmar que 
\begin{equation}
\left(\MATRIX{R}^{\ast} \VECTOR{x}\right)^{\ast} \MATRIX{R}^{\ast} \VECTOR{x}>0
\quad \rightarrow \quad
\VECTOR{x}^{\ast} \MATRIX{A} \VECTOR{x}>0.
\end{equation}
\item Se usamos o Teorema \ref{theo:HermitianDDecomposition0} e \ref{theo:hermitianpositivematrix1}, 
podemos afirmar que 
$\MATRIX{A}=\MATRIX{U}\lambda_{\MATRIX{A}} \MATRIX{U}^{\ast}$ e que seus autovalores $\lambda_n>0$, consequentemente
\begin{equation}
\MATRIX{A}=
\MATRIX{U}\lambda_{\MATRIX{A}} \MATRIX{U}^{\ast}=
\MATRIX{U}\sqrt{\lambda_{\MATRIX{A}}} \sqrt{\lambda_{\MATRIX{A}}} \MATRIX{U}^{\ast}=
\left(\sqrt{\lambda_{\MATRIX{A}}} \MATRIX{U}^{\ast}\right)^{\ast} \left(\sqrt{\lambda_{\MATRIX{A}}} \MATRIX{U}^{\ast}\right).
\end{equation}
Se adicionalmente definimos $\MATRIX{R}^{\ast}\equiv \sqrt{\lambda_{\MATRIX{A}}} \MATRIX{U}^{\ast}$,
podemos afirmar que $\MATRIX{A}=\MATRIX{R}\MATRIX{R}^{\ast}$.
\end{itemize}
\end{myproofT}

%%%%%%%%%%%%%%%%%%%%%%%%%%%%%%%%%%%%%%%%%%%%%%%%%%%%%%%%%%%%%%%%%%%%%%%%%%%%%%%%%%%%%%%
%%%%%%%%%%%%%%%%%%%%%%%%%%%%%%%%%%%%%%%%%%%%%%%%%%%%%%%%%%%%%%%%%%%%%%%%%%%%%%%%%%%%%%%

\begin{myproofT}[Relativa ao Teorema \ref{theo:prophermitianpositivematrix1}:]\label{proof:theo:prophermitianpositivematrix1:b}
Se $\MATRIX{A} \in \mathbb{C}^{N \times N}$ é uma matriz hermitiana definida positiva, 
$\MATRIX{B} \in \mathbb{C}^{N \times N}$ é não singular e
$\VECTOR{x} \in \mathbb{C}^{N}$ é não nulo, então
\begin{equation}
\VECTOR{y}=\MATRIX{B}\VECTOR{x}\neq 0 \iff \VECTOR{x}\neq 0
\end{equation}
\begin{equation}
\VECTOR{x}^{\ast} \MATRIX{B}^{\ast} \MATRIX{A} \MATRIX{B} \VECTOR{x} =
\left(\MATRIX{B}\VECTOR{x}\right)^{\ast} \MATRIX{A} \MATRIX{B} \VECTOR{x} =
\VECTOR{y}^{\ast} \MATRIX{A} \VECTOR{y}>0,
\end{equation}
se $\VECTOR{y}\neq 0$ então $\MATRIX{B}^{\ast} \MATRIX{A} \MATRIX{B}$
é uma matriz hermitiana definida positiva.
\end{myproofT}


%%%%%%%%%%%%%%%%%%%%%%%%%%%%%%%%%%%%%%%%%%%%%%%%%%%%%%%%%%%%%%%%%%%%%%%%%%%%%%%%%%%%%%%
%%%%%%%%%%%%%%%%%%%%%%%%%%%%%%%%%%%%%%%%%%%%%%%%%%%%%%%%%%%%%%%%%%%%%%%%%%%%%%%%%%%%%%%
\begin{myproofT}[Relativa ao Teorema \ref{theo:hermitianpositivematrix1}:]\label{proof:theo:hermitianpositivematrix1}
Conhecida uma matriz hermitiana $\MATRIX{A} \in \mathbb{R}^{N \times N}$ com  autovalores $\lambda_n$,
e autovetores $\VECTOR{v}_n$, $\forall n \in \{1, 2, ..., N\}$, podemos afirmar que,
\begin{equation}\label{eq:proof:theo:hermitianpositivematrix1:0}
\MATRIX{A}\VECTOR{v}_n=\lambda_n \VECTOR{v}_n,
\end{equation}
\begin{equation}\label{eq:proof:theo:hermitianpositivematrix1:1}
\VECTOR{v}_n^{\ast}\MATRIX{A}\VECTOR{v}_n=
\lambda_n \VECTOR{v}_n^{\ast}\VECTOR{v}_n=
\lambda_n \NORM{\VECTOR{v}_n}^2.
\end{equation}

\begin{itemize}
\item Usando a Eq. (\ref{eq:proof:theo:hermitianpositivematrix1:1}), a 
Definição \ref{def:hermitianapositivematrix0} e o fato que a matriz $\MATRIX{A}$ é definida positiva,
podemos afirmar que
\begin{equation}
\VECTOR{v}_n^{\ast}\MATRIX{A}\VECTOR{v}_n >0
\quad \rightarrow \quad
\lambda_n \NORM{\VECTOR{v}_n}^2 > 0
\quad \rightarrow \quad
\lambda_n  > 0
\end{equation} 
\item Sabendo do Teorema \ref{theo:HermitianDDecomposition0} que 
$\MATRIX{A}=\MATRIX{U}\lambda_{\MATRIX{A}} \MATRIX{U}^{\ast}$,
onde 
$\lambda_{\MATRIX{A}}=\funcdiag\left(\left[\lambda_1\quad \lambda_2 \quad ... \quad \lambda_N \right]^{\transpose}\right)$
e $\MATRIX{U}$ é uma matriz unitária,
podemos afirmar que
\begin{equation}
\VECTOR{x}^{\ast}\MATRIX{A}\VECTOR{x}=
\VECTOR{x}^{\ast}\MATRIX{U}\lambda_{\MATRIX{A}} \MATRIX{U}^{\ast}\VECTOR{x}=
\left(\MATRIX{U}^{\ast}\VECTOR{x}\right)^{\ast}\lambda_{\MATRIX{A}}\left(\MATRIX{U}^{\ast}\VECTOR{x}\right).
\end{equation}
Se definimos $\MATRIX{U}^{\ast}\VECTOR{x}=\VECTOR{y}$, considerando
$\VECTOR{x}$ não nulo e $\MATRIX{U}^{\ast}=\MATRIX{U}^{-1}$ invertível,
podemos afirmar que 
$\VECTOR{y}=\left[y_1\quad y_2 \quad ... \quad y_N \right]^{\transpose}$ é não nulo e que 
\begin{equation}
\VECTOR{x}^{\ast}\MATRIX{A}\VECTOR{x}=
\VECTOR{y}^{\ast}\lambda_{\MATRIX{A}}\VECTOR{y}=\sum_{n=1}^{N}\lambda_n |y_n|^2.
\end{equation}
Se ademais assumimos que $\lambda_n  > 0$, podemos afirmar que 
\begin{equation}
\VECTOR{x}^{\ast}\MATRIX{A}\VECTOR{x}=\sum_{n=1}^{N}\lambda_n |y_n|^2>0.
\end{equation}
Consequentemente podemos afirmar que a matriz hermitiana $\MATRIX{A}$
é definida positiva se seus autovalores $\lambda_n  > 0$.
\end{itemize}
\end{myproofT}

