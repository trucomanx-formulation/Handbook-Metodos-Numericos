\cleardoublepage
\newpage
\thispagestyle{empty}

\vfill
\begin{center}
\textbf{\LARGE  Prefácio }
\end{center}

Este livro foi criado devido a minha necessidade de materializar de forma permanente
um conjunto de ideias, modelos e soluções matemáticas, que recorrentemente
tenho achado útil na minha vida profissional;
especificamente na área da engenharia e no processamento digital de sinais.
Por muito tempo eu caí no ciclo de ver o mesmo problema matemático repetindo-se,
com pequenas variantes na modelagem,  
e iniciei do zero o processo de demonstrar, aplicar e achar uma solução específica a cada problema.
Com o tempo, meu gosto pela programação estruturada e a regra do mínimo esforço,
me motivaram a escrever e organizar minhas ideias de forma modular
para que estas sejam facilmente reutilizáveis; 
esses conceitos são os que o leitor verá nos seguintes capítulos.

Sobre o conteúdo do livro, 
considero que o tema central deste é mostrar alguns dos métodos de solução dos problemas inversos.
Diferentemente de um problema direto 
--- no qual estamos interessados na obtenção dos dados de saída de um sistema
a partir dos dados de entrada e as variáveis de configuração deste --- 
num problema inverso, o sentido do fluxo da informação é invertido e podemos ter, por exemplo,
a necessidade de conhecer as variáveis de configuração do sistema a partir da informação
dos dados de saída e de entrada.
Assim, decorrente do problema inverso, 
outros temas vinculados foram abordados no livro,
por exemplo, os métodos numéricos, a resolução de problemas não lineares, 
a regressão, a classificação, o método de mínimos quadrados,
a álgebra matricial
e a derivada de funções vetoriais de várias variáveis.

Mediante toda a teoria abordada neste livro, 
minha intenção não foi ter apenas uma fonte de informação onde possa debruçar-me quando necessário nas minhas pesquisas,
e sim compartilhar esta informação 
--- em especial no que diz respeito a forma em que esta foi estruturada e ordenada ---
para que auxilie a qualquer pessoa interessada no estudo dos métodos numéricos
para a solução de problemas não lineares e inversos.

\vfill
