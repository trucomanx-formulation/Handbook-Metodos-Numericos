\cleardoublepage
\newpage
\thispagestyle{empty}

\vfill
\begin{center}
\textbf{\LARGE  Prefácio }
\end{center}

Este livro foi criado devido a minha necessidade de plasmar de forma permanente
um conjunto de ideias, modelos e soluções matemáticas, que recorrentemente
tenho usado e encontrado útil na minha vida profissional;
especificamente na área da engenharia e no processamento digital de sinais.
\begin{comment}
Assim, da mesma forma como se disse que 
``quem não conhece sua história está fadado a repeti-la.'' 
Podemos disser que quem não lembra sua teoria, seus problemas matemáticos ou trabalhos resolvidos, 
estruturando eles modularmente, 
está condenado a ter que 
modelar e demostrar todo varias vesses.
\end{comment}
Por muito tempo eu caí no ciclo de ver repetindo-se o mesmo problema matemático,
com pequenas variantes na modelagem,  
e fiz desde zero o processo de demonstrar, aplicar e achar uma solução de forma específica para cada problema.
Com o tempo, meu gosto pela programação estruturada e a regra do mínimo esforço,
me motivaram a escrever e organizar minhas ideias de forma modular; 
esses conceitos são os que o leitor poderá ver nas seguintes seções.

Sobre o conteúdo do livro, 
considero que o tema central deste é mostrar alguns dos métodos de solução dos problemas inversos.
A diferença de um problema direto, 
em que estamos interessados na obtenção dos dados de saída de um sistema
a partir dos dados de entrada e as variáveis de configuração deste. 
Num problema inverso o sentido do fluxo da informação é invertido e podemos ter, por exemplo,
a necessidade de conhecer as variáveis de configuração do sistema a partir da informação
dos dados de saída e os dados de entrada.
Assim, decorrente do problema inverso, 
outros temas vinculados a este foram necessariamente abordados no livro,
por exemplo, os métodos numéricos, a resolução de problemas não lineares, 
a regressão, a classificação, o método de mínimos quadrados,
a álgebra matricial
e a derivada de funções vetoriais de várias variáveis.

Mediante toda a teoria plasmada neste livro, 
minha intenção não é ter apenas uma fonte de informação onde posso debruçar-me quando seja necessário nas minhas pesquisas,
e sim compartilhar esta informação, em especial a forma que esta foi estruturada e ordenada,
para que seja de ajuda a qualquer pessoa interessada no estudo dos métodos numéricos
para a solução de problemas não lineares e inversos.

\vfill
