\chapter*{Lista de Símbolos}

\singlespacing

%\noindent
\section*{Tipos de Dados}
\begin{tabular}{r | p{.45\linewidth} | l}
\hline	
Tipo & Descrição & Formatação \\ \hline
$\mathbf{A}$, $\mathbf{B}$, ..., $\mathbf{X}$, $\mathbf{Y}$, $\mathbf{Z}$& Matriz. & Maiúsculo e negrito \\
$\mathbf{a}$, $\mathbf{b}$, ..., $\mathbf{x}$, $\mathbf{y}$, $\mathbf{z}$ & Vetor (Por falta é asumido que é um vetor coluna) ou conjunto. & Minúsculo e negrito \\
%$A$, $B$, ..., $X$, $Y$, $Z$ & ------- & Maiúsculo \\
$a$, $b$, ..., $x$, $y$, $z$ & Escalar variável. & Minúsculo \\
$A$, $B$, ..., $X$, $Y$, $Z$ & Escalar constante. & Minúsculo \\
$\alpha$, $\beta$, ..., $\chi$, $\psi$, $\omega$ & Escalar variável ou constante. & Letras gregas e minúsculo  \\ \hline
\end{tabular}

\section*{Elementos de matrizes bidimensionais}
\begin{tabular}{r | p{.55\linewidth} | l}
\hline	
Tipo & Descrição & Formatação \\ \hline
$a_{ij}$ & Escalar formado pelo elemento da linha $i$, coluna $j$ da matriz $\mathbf{A}$. & Minúsculo \\ \hline
%$\mathbf{A}_{(i,j)}$& Elemento na linha $i$, coluna $j$ da matriz $\mathbf{A}$ & Maiúsculo e negrito \\ \hline
%$\mathbf{a}_{i}$ & Linha ou coluna $i$-essima da matriz $\mathbf{A}$ (ambíguo) & Minúsculo e negrito \\
$a_{i:}$ & Vetor linha formado pela linha $i$-essima da matriz $\mathbf{A}$.  & Minúsculo \\
%$\mathbf{A}_{(i,:)}$& Vetor formado pela linha $i$-essima da matriz $\mathbf{A}$ & Maiúsculo e negrito \\
$a_{:i}$ & Vetor coluna formado pela coluna $i$-essima da matriz $\mathbf{A}$.  & Minúsculo \\
%$\mathbf{A}_{(:,i)}$& Vetor formado pela coluna $i$-essima da matriz $\mathbf{A}$ & Maiúsculo e negrito \\ \hline
\end{tabular}

\section*{Elementos de vetores ou conjuntos}
\begin{tabular}{r | p{.55\linewidth} | l}
\hline	
Tipo & Descrição & Formatação \\ \hline
$a_{i}$ & Elemento $i$-essimo do vetor ou conjunto  $\mathbf{a}$.& Minúsculo \\
%$\mathbf{a}_{(i)}$ & Elemento $i$-essimo do vetor $\mathbf{a}$ & Minúsculo e negrito \\ \hline
\end{tabular}

\section*{Funções notáveis}
\begin{tabular}{r | p{.70\linewidth} }
\hline	
Função & Descrição \\ \hline
$card(\mathbf{a})$ & Número de elementos, cardinalidade, do vetor ou conjunto $\mathbf{a}$. \\
$card(\mathbf{A})$ & Número de elementos, cardinalidade, da matriz $\mathbf{A}$. \\
\hline
$dim(\mathbf{a},1)$ & Primeira dimensão do vetor $\mathbf{a}$, número de linhas. \\
$dim(\mathbf{a},2)$ & Segunda dimensão do vetor $\mathbf{a}$, número de colunas. \\
$dim(\mathbf{A},1)$ & Primeira dimensão da matriz $\mathbf{A}$, número de linhas. \\
$dim(\mathbf{A},2)$ & Segunda dimensão da matriz $\mathbf{A}$, número de colunas. \\
$dim(\mathbf{A},N)$ & Dimensão $N$ da matriz $\mathbf{A}$, se não tiver retorna $0$. \\
\hline
$inv(\mathbf{A})$ & Inversa da matriz $\mathbf{A}$, é equivalente a escrever $\mathbf{A}^{-1}$. \\
$\mathbf{A}^{-1}$ & Inversa da matriz $\mathbf{A}$, é equivalente a escrever $inv(\mathbf{A})$. \\
\hline
$trans(\mathbf{a})$ & Transposta do vetor $\mathbf{a}$, é equivalente a escrever $\mathbf{a}^{T}$. \\
$\mathbf{a}^{T}$ & Transposta do vetor $\mathbf{a}$, é equivalente a escrever $trans(\mathbf{a})$. \\
$trans(\mathbf{A})$ & Transposta da matriz $\mathbf{A}$, é equivalente a escrever $\mathbf{A}^{T}$. \\
$\mathbf{A}^{T}$ & Transposta da matriz $\mathbf{A}$, é equivalente a escrever $trans(\mathbf{A})$. \\
\hline
$||\mathbf{a}||$ & Módulo do vetor $\mathbf{a}$, é equivalente a escrever $\sqrt{\sum_i a_i^2}$. \\
$||\mathbf{a}||^2$ & Módulo ao quadrado do vetor $\mathbf{a}$, 
é equivalente a escrever $\sum_i a_i^2$ ou $\mathbf{a}^{T}\mathbf{a}$ se $\mathbf{a}$ é um vetor coluna. \\
$||\mathbf{a}||_{\mathbf{B}}^2$ & Módulo ao quadrado ponderado do vetor $\mathbf{a}$, 
é equivalente a escrever $\sum_i b_{ii} a_i^2$ ou $(\sqrt{\mathbf{B}}~\mathbf{a})^{T}\sqrt{\mathbf{B}}~\mathbf{a} \equiv \mathbf{a}^{T}\mathbf{B}\mathbf{a}$ 
se $\mathbf{a}$ é um vetor coluna. Sendo que $\mathbf{B}$ é uma matriz diagonal.\\
\end{tabular}

\onehalfspacing
