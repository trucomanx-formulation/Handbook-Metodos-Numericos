\chapterimage{chapter_head_2.pdf} % Chapter heading image

\chapter{Notação}


\begin{notation}Modo em que os valores escalares, vetoriais ou matriciais são definidos:\\
\begin{tabular}{r | p{.4\linewidth} | l}
\hline	
Tipo & Descrição & Formatação \\ \hline
$\MATRIX{A}$, $\MATRIX{B}$, ..., $\MATRIX{X}$, $\MATRIX{Y}$, $\MATRIX{Z}$& Matriz. & Maiúsculo e negrito \\
$\VECTOR{a}$, $\VECTOR{b}$, ..., $\VECTOR{x}$, $\VECTOR{y}$, $\VECTOR{z}$ & Vetor ou conjunto. & Minúsculo e negrito \\
%$A$, $B$, ..., $X$, $Y$, $Z$ & ------- & Maiúsculo \\
$a$, $b$, ..., $x$, $y$, $z$ & Escalar variável. & Minúsculo \\
$A$, $B$, ..., $X$, $Y$, $Z$ & Escalar constante. & Minúsculo \\
$\alpha$, $\beta$, ..., $\chi$, $\psi$, $\omega$ & Escalar variável ou constante. & Letras gregas e minúsculo  \\ \hline
\end{tabular}
\end{notation}

\begin{notation}Modo em que os elementos das matrizes bidimensionais são definidos:\\
\begin{tabular}{r | p{.6\linewidth} | l}
\hline	
Tipo & Descrição & Formatação \\ \hline
$a_{ij}$ & Escalar formado pelo elemento da linha $i$, coluna $j$ da matriz $\MATRIX{A}$. & Minúsculo \\ \hline
%$\MATRIX{A}_{(i,j)}$& Elemento na linha $i$, coluna $j$ da matriz $\MATRIX{A}$ & Maiúsculo e negrito \\ \hline
%$\VECTOR{a}_{i}$ & Linha ou coluna $i$-essima da matriz $\MATRIX{A}$ (ambíguo) & Minúsculo e negrito \\
$a_{i:}$ & Vetor linha formado pela linha $i$-essima da matriz $\MATRIX{A}$.  & Minúsculo \\
%$\MATRIX{A}_{(i,:)}$& Vetor formado pela linha $i$-essima da matriz $\MATRIX{A}$ & Maiúsculo e negrito \\
$a_{:i}$ & Vetor coluna formado pela coluna $i$-essima da matriz $\MATRIX{A}$.  & Minúsculo \\
%$\MATRIX{A}_{(:,i)}$& Vetor formado pela coluna $i$-essima da matriz $\MATRIX{A}$ & Maiúsculo e negrito \\ \hline
\end{tabular}
\end{notation}

\begin{notation}Modo em que os elementos dos vetores ou conjuntos são definidos:\\
\begin{tabular}{r | p{.6\linewidth} | l}
\hline	
Tipo & Descrição & Formatação \\ \hline
$a_{i}$ & Elemento $i$-essimo do vetor ou conjunto  $\VECTOR{a}$.& Minúsculo \\
%$\VECTOR{a}_{(i)}$ & Elemento $i$-essimo do vetor $\VECTOR{a}$ & Minúsculo e negrito \\ \hline
\end{tabular}
\end{notation}


\begin{notation}Funções notáveis usadas neste livro:\\
\begin{tabular}{r | l }
\hline	
Função & Descrição \\ \hline
$card(\VECTOR{a})$ & Número de elementos, cardinalidade, do vetor ou conjunto $\VECTOR{a}$. \\
$card(\MATRIX{A})$ & Número de elementos, cardinalidade, da matriz $\MATRIX{A}$. \\
\hline
$dim(\VECTOR{a},1)$ & Primeira dimensão do vetor $\VECTOR{a}$, número de linhas. \\
$dim(\VECTOR{a},2)$ & Segunda dimensão do vetor $\VECTOR{a}$, número de colunas. \\
$dim(\MATRIX{A},1)$ & Primeira dimensão da matriz $\MATRIX{A}$, número de linhas. \\
$dim(\MATRIX{A},2)$ & Segunda dimensão da matriz $\MATRIX{A}$, número de colunas. \\
$dim(\MATRIX{A},N)$ & Dimensão $N$ da matriz $\MATRIX{A}$, se não tiver retorna $0$. \\
\hline
$inv(\MATRIX{A})$ & Inversa da matriz $\MATRIX{A}$, é equivalente a escrever $\MATRIX{A}^{-1}$. \\
$\MATRIX{A}^{-1}$ & Inversa da matriz $\MATRIX{A}$, é equivalente a escrever $inv(\MATRIX{A})$. \\
\hline
$trans(\VECTOR{a})$ & Transposta do vetor $\VECTOR{a}$, é equivalente a escrever $\VECTOR{a}^{T}$. \\
$\VECTOR{a}^{T}$ & Transposta do vetor $\VECTOR{a}$, é equivalente a escrever $trans(\VECTOR{a})$. \\
$trans(\MATRIX{A})$ & Transposta da matriz $\MATRIX{A}$, é equivalente a escrever $\MATRIX{A}^{T}$. \\
$\MATRIX{A}^{T}$ & Transposta da matriz $\MATRIX{A}$, é equivalente a escrever $trans(\MATRIX{A})$. \\
\end{tabular}
\end{notation}

